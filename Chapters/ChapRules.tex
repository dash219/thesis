%*****************************************
\chapter{Rules for generating pronunciation variants}\label{ch:rules}
%*****************************************


The pseudocode for each mispronunciation pattern is described in the following subsections. The conventions below apply to all examples:

\begin{enumerate}
 \item phonetic transcriptions are placed between brackets and follow the Listener phoneset convention ]
 (see \autoref{sec:listener-phoneset}), therefore [\textipa{"hElow}] becomes [hh eh l ow];
 \item ortographic forms are placed between angle brackets, like <hello>;
 \item the dollar sign ``\$'' represents word final boundary, whether in phonetic or ortographic transcription, 
 thus [ih d\$] means that the transcription ends in [ih d];
 \item the caret symbol ``\textasciicircum'' denotes word initial boundary, whether in phonetic or 
 ortographic transcription, thus <\textasciicircum st> means that word begins with <\textasciicircum st>.
 \item the pipe symbol ``$\vert$'' inside parentheses describes alternatives among a group of phones or 
 graphemes, e.g. [ah (m$\vert$n$\vert$g)] means that [ah] is imediatelly followed by [m], [n] or [ng].
\end{enumerate}

\subsection{Syllable Simplification}
Rules for syllable simplification were created upon production data reported on the following works:
\citeauthor{Cardoso2011} \citep{Cardoso2011}, \citeauthor{Silveira2012} \citep{Silveira2012}
\citeauthor{Rauber2004} \citep{Rauber2004}, and \citeauthor{Rebello2006} \citep{Rebello2006}. The rules
encompass three major cases of syllable simplification. 

The first one refers to stop consonants in word 
final position, such as the final [\textipa{p}] in ``pop'' which is likely to be produced by learners
together with an epenthetic vowel [\textipa{I}], whence [\textipa{"p\super hApI}]. 

The second case regards written language
influencing the learner's pronunciation, e.g. brazilian \ac{ESL} learners tend to pronounce the word ``name'' with a final 
[\textipa{I}], although such vowel does not occur in the English form: [\textipa{"neIm}]. This happens influenced by written language,
in \ac{BP} ortography, a final ``e'' means that the word should end with the phone [\textipa{I}], thus the \ac{ESL} student
transfer put this knowledge into practice in his/her interlanguage. 

The third and last case of syllable simplification concerns initial clusters formed
by [\textipa{s}] and another consonant, hence initial [\textipa{s}C] clusters. In this case, an epenthetic vowel [\textipa{i}] is appended to the
beginning of the word and the consonantal cluster is broken into two syllables, so that [\textipa{s}C] becomes [\textipa{is.}C].

The pseudocode containing rules for all these contexts is defined in \autoref{lst:rules-syll-simplification}.

\lstinputlisting[label=lst:rules-syll-simplification,caption=Pronunciation rules for generating syllable simplification cases.]%
    {Examples/pseudo-rules-syll-simplif.txt}

\subsection{Consonant Change}\label{sec:consonant-change}
The rules for consonantal change were defined based on the results of
\citeauthor{Reis2006} \citep{Reis2006} and \citeauthor{Trevisol2010} \citep{Trevisol2010}.
\autoref{lst:rules-cons-change} describes the pseudecode for generating the mispronunciations regarding consont changes.

\lstinputlisting[float=H,label=lst:rules-cons-change,caption=Pronunciation rules for generating consonant change cases.]%
    {Examples/pseudo-rules-cons-change.txt}
    
\subsection{Deaspiration of Voiceless Plosives in Initial or Stressed Positions}

With regard to the deaspiration of voiceless plosives, we considered the analyses of the phenomenon as described in following papers: 
\citeauthor{Alves2008} \citep{Alves2008}, \citeauthor{Prestes2012} \citep{Prestes2012}, 
\citeauthor{Scwartzhaupt2014} \citep{Scwartzhaupt2014} and \citeauthor{Zimmer2006} \citep{Zimmer2006}. These analyses establish 
fundementally the same, brazilian \ac{ESL} learners tend to replace the aspirated phones in English 
[\textipa{p\super h}, \textipa{t\super h}, \textipa{k\super h}] by their corresponding deaspirated ones 
[\textipa{p}, \textipa{t}, \textipa{k}], since the latter ones are part of \ac{BP} phonetic inventory, whereas the former ones are not. 

Therefore, for generating the deaspiration cases, it would simple mean to transform all occurrences of
[\textipa{p\super h}, \textipa{t\super h}, \textipa{k\super h}] into [\textipa{p}, \textipa{t}, \textipa{k}]. However,
in \ac{CMUdict}, the difference aspirated voiceless plosives and the deaspirated ones is not pointed out. Both aspirated and 
non-aspirated plosives are annotated with the same
symbol, for instance, ``pit'' is transcribed as [p ih t] and ``spit'' as [s p ih t], despite the fact that the former [p] is aspirated 
and the latter is not. This is also true for aspirated and non-aspirated [t] and [k]. ``Top'' and ``stop'' are transcribed both with a [t]
although they correspond to different sounds, the same happens with the [k] in ``cat'' and ``scat''.

For solving this problem, we chose to transform all [p, t, k] in the corpus into non-aspirated ones, thus [pp, tt, kk] according
to Listener's phoneset convention (see \autoref{sec:listener-phoneset}. After replacing all [p, t, k] in the dictionary by
[pp, tt, kk], we generated their aspirated counterparts, given that their contexts is predictable. In \ac{AmE}, there basically 
two contexts in which voiceless plosives become aspirated: i) when they occur in a stressed sylllable, not after an [\textipa{s}]; 
ii) and when they occur in word initial position, irrespective of the stress \citep{Lisker1985}. We rewrote these contexts 
into rules, then producing all voiceless aspirated in the corpus.
\footnote{Since deaspiration is also conditioned by the stress of the syllable, we had to move back to the original version of \ac{CMUdict}, 
where lexical stress is annotated. In spite of this, for simplicity we are going to leave this workaround out of the pseudocode 
and assume that the dictionary has the stress information.} The pseudocode for the rules is described in \autoref{lst:rules-deaspiration}.

\lstinputlisting[float=H,label=lst:rules-deaspiration,caption=Pronunciation rules for generating plosive deaspiration cases.]%
    {Examples/pseudo-rules-deaspiration.txt}
    
    
\clearpage
\subsection{Devoicing in Word-Final Obstruents}

For terminal devoicing cases, we based our rules on the contexts presented by \citeauthor{Castilho2004} \citep{Castilho2004}.
These rules explicit a phonetic process in which the final [z] becomes devoiced, being produced as [s], in words such as ``Charles'' 
or ``rose''. That is to say these words pronounced as [\textipa{"roUs}] and [\textipa{"tSArls}], instead of [\textipa{"roUz}] 
and [\textipa{"tSArlz}].

\citeauthor{Albuquerque2011} \citep{Albuquerque2011} studied cases of terminal devoicing in plosives, as in [\textipa{"dAg}] 
turning into [\textipa{"dAk}], however their results are inconclusive given the limited sample. 
Besides the study of \citeauthor{Zimmer2012} contradicts \citeauthor{Albuquerque2011} \citep{Albuquerque2011}, for they found 
no statistical significance, in what regards to voicing quality, among the realizations of word final obstruents
between native speakers of English and brazilian \ac{ESL} learners.

For this reason, we decided to discard such phenomenon and deal only with the terminal devoicing of fricatives.
Thereby a more accurate title for this section would be ``devoicing in word-final fricatives'', yet we 
preferred to keep the broader natural class ``obstruents'', since that is how the process is commonly referred to in the literature. 
\autoref{lst:rules-terminal-devoicing} contains the pseudocode for the rewrite rules.
\footnote{It is worth mentioning that the process described in \autoref{sec:consonant-change}, when [\textipa{D}] is realized as
[\textipa{T}] in word-final position is also a case of terminal obstruent devoicing. in spite of this, we did not
find any paper in the literature classifying this phenomenon in such way.}

\lstinputlisting[float=H,label=lst:rules-terminal-devoicing,caption=Pronunciation rules for generating terminal devoicing cases.]%
    {Examples/pseudo-rules-obstruent-devoicing.txt}

\clearpage    
\subsection{Delateralization and rounding of lateral liquids in final position}
In what concerns to the delateralization and rounding of lateral liquids in final position (also known as vocalization of lateral), we based
our rules in the findings of \citeauthor{Baratieri2006} \citep{Baratieri2006} and \citeauthor{Moore2008} \citep{Moore2008}. Both these authors
describe the same phonetic contexts in which the phenomenon takes place. According to them the English lateral [\textipa{l}] is
pronounced as a vowel [\textipa{U}] by many brazilian \ac{ESL} learners, when such consonant occurs in syllable final position before
a consontar, or in word final position. For instance, learners will tend to say ``salt'' as [\textipa{"kIU}] instead of [\textipa{"kIl}].
The pseudocode with the rules for generating lateral vocalization mispronunciations is found in \autoref{lst:rules-lateral-vocalization}.

\lstinputlisting[float=H,label=lst:rules-lateral-vocalization,caption=Pronunciation rules for generating lateral vocalization cases.]%
    {Examples/pseudo-rules-lateral-vocalization.txt}

\clearpage
\subsection{Vocalization of final nasals}

Rules for the vocalization of final nasals were grounded on the phonological processes described in the following works:
\citeauthor{Kluge2007} \citep{Kluge2007}, \citeauthor{Kluge2008} \citep{Kluge2008}, \citeauthor{Kluge2012} \citep{Kluge2012}, 
\citeauthor{Silveira2007} \citep{Silveira2007} and \citeauthor{Silveira2012} \citep{Silveira2012}.
The rules comprise two main cases with respect to the vocalization of final nasals. 

In the first one, the
final nasal is deleted, the previous vowel is nasalized and, if the vowel does not exist in \ac{BP}, it is produced as its
most acoustically similar vowel in \ac{BP}. For instance, in the word ``sandwich'' [\textipa{"s\ae nd.wItS}],
the  [\textipa{\ae n}] is likely
to become [\textipa{\~a}] in the production of brazilian \ac{ESL} learners, with the nasal [\textipa{n}] being deleted, the 
[\textipa{\ae}] vowel being replaced by the \ac{BP} vowel [a] and then becoming nasalized; hence [\textipa{"s\~and.wItS}]. 

The second case addresses L1 spelling patterns interfering in the pronunciation of the brazilian \ac{ESL} learners. This process
is similar to the one described above, the nasal is deleted and the previous is vowel also nasalized, however the vowel quality is 
influenced not perceptual traits, but by the ortographic form of the word. The word ``opium'', for example, although ending
with a schwa [\textipa{@}], [\textipa{"oU.pI.@m}], is usually produced with a [\textipa{\~u}] by the brazilian \ac{ESL} learner, given that an ``u''
appears in the spelling, so [\textipa{"oU.pI.\~u}].

\autoref{lst:rules-nasal-vocalization} contains the pseudocode for generating both cases of final nasal vocalization.

\lstinputlisting[float=H,label=lst:rules-nasal-vocalization,caption=Pronunciation rules for generating nasal vocalization cases.]%
    {Examples/pseudo-rules-nasal-vocalization.txt}

\clearpage
\subsection{Velar consonantal paragoge}

\lstinputlisting[float=H,label=lst:rules-velar-paragoge,caption=Pronunciation rules for generating velar paragoge cases.]%
    {Examples/pseudo-rules-velar-paragoge.txt}

\clearpage
\subsection{Vowel assimilation}

The rules for vowel assimilation were determined based on the results of
\citeauthor{Battistela2010} \citep{Battistela2010}, \citeauthor{Rauber2005} \citep{Rauber2005} and \citeauthor{Rauber2006} \citep{Rauber2006}.
\autoref{lst:rules-vowel-assimilation} contains the pseudecode for generating pronunciations with vowel assimilations from L1 to L2.

\lstinputlisting[float=H,label=lst:rules-vowel-assimilation,caption=Pronunciation rules for generating vowel assimilation cases.]%
    {Examples/pseudo-rules-vowel-assimilation.txt}

\clearpage
\subsection{Interconsonantal epenthesis (-ed and -s morphemes)}




%*****************************************
%*****************************************
%*****************************************
%*****************************************
%*****************************************