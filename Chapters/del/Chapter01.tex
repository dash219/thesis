%************************************************
\chapter{Introduction}\label{ch:introduction}
%************************************************
This bundle for \LaTeX\ has two goals:
\begin{enumerate}
    \item Provide students with an easy-to-use template for their
    Master's
    or PhD thesis. (Though it might also be used by other types of
    authors
    for reports, books, etc.)
    \item Provide a classic, high-quality typographic style that is
    inspired by \citeauthor{bringhurst:2002}'s ``\emph{The Elements of
    Typographic Style}'' \citep{bringhurst:2002}.
    \marginpar{\myTitle \myVersion}
\end{enumerate}
The bundle is configured to run with a \emph{full} 
MiK\TeX\ or \TeX Live\footnote{See the file \texttt{LISTOFFILES} for
needed packages. Furthermore, \texttt{classicthesis} 
works with most other distributions and, thus, with most systems 
\LaTeX\ is available for.} 
installation right away and, therefore, it uses only freely available 
fonts. (Minion fans can easily adjust the style to their needs.)

People interested only in the nice style and not the whole bundle can
now use the style stand-alone via the file \texttt{classicthesis.sty}.
This works now also with ``plain'' \LaTeX.

As of version 3.0, \texttt{classicthesis} can also be easily used with 
\mLyX\footnote{\url{http://www.lyx.org}} thanks to Nicholas Mariette 
and Ivo Pletikosi\'c. The \mLyX\ version of this manual will contain
more information on the details.

This should enable anyone with a basic knowledge of \LaTeXe\ or \mLyX\ to
produce beautiful documents without too much effort. In the end, this
is my overall goal: more beautiful documents, especially theses, as I
am tired of seeing so many ugly ones.

The whole template and the used style is released under the
\textsmaller{GNU} General Public License. 

If you like the style then I would appreciate a postcard:
\begin{center}
 Andr� Miede \\
 Detmolder Stra�e 32 \\
 31737 Rinteln \\
 Germany
\end{center}
The postcards I received so far are available at:
\begin{center}
 \url{http://postcards.miede.de}
\end{center}
\marginpar{A well-balanced line width improves the legibility of
the text. That's what typography is all about, right?}
So far, many theses, some books, and several other publications have 
been typeset successfully with it. If you are interested in some
typographic details behind it, enjoy Robert Bringhurst's wonderful book.
% \citep{bringhurst:2002}.

\paragraph{Important Note:} Some things of this style might look
unusual at first glance, many people feel so in the beginning.
However, all things are intentionally designed to be as they are,
especially these:
\begin{itemize}
    \item No bold fonts are used. Italics or spaced small caps do the
    job quite well.
    \item The size of the text body is intentionally shaped like it
    is. It supports both legibility and allows a reasonable amount of
    information to be on a page. And, no: the lines are not too short.
    \item The tables intentionally do not use vertical or double
    rules. See the documentation for the \texttt{booktabs} package for
    a nice discussion of this topic.\footnote{To be found online at \\
    \url{http://www.ctan.org/tex-archive/macros/latex/contrib/booktabs/}.}
    \item And last but not least, to provide the reader with a way
    easier access to page numbers in the table of contents, the page
    numbers are right behind the titles. Yes, they are \emph{not}
    neatly aligned at the right side and they are \emph{not} connected
    with dots that help the eye to bridge a distance that is not
    necessary. If you are still not convinced: is your reader
    interested in the page number or does she want to sum the numbers
    up?
\end{itemize}
Therefore, please do not break the beauty of the style by changing
these things unless you really know what you are doing! Please.





