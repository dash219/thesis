%*****************************************
\chapter{Conclusions}\label{ch:conclusions}
%*****************************************

\section{Overall Conclusions}\label{sec:overall-conclusions}

Resultados em reconhecimento de fala, em muitas vezes, s\~ao
explorat\'orios. As l\'inguas s\~ao sistemas din\^amicos, variando dado o
espa\c{c}o, o tempo, os grupos sociais, as situa\c{c}\~oes comunicativas e o
pr\'oprio falante. Tendo em vista que cada l\'ingua possui aspectos
particulares, n\~ao se pode assegurar, no PLN, que m\'etodos j\'a testados
para outros idiomas, sejam diretamente transplantados para o PB e tenham
funcionamento e desempenho similar. Esta disserta\c{c}\~ao busca propor um
m\'etodo de elabora\c{c}\~ao de um sistema de reconhecimento de pron\'uncia para
aprendizes de ingl\^es, falantes nativos do PB. Um sistema desse tipo
ainda n\~ao foi elaborado para o PB e os resultados, portanto, s\~ao
tentativos. Como se discutiu, tal sistema \'e de utilidade, tendo em vista
a baixa profici\^encia em ingl\^es dos brasileiros, demonstrada recentemente
nos \'indices da GlobalEnglish (2012) e da Education First (2013).

A literatura pertinente da \'area foi revisada e m\'etodos que se mostraram
promissores foram selecionados para integrar o projeto. Pretende-se
elaborador um reconhecedor de pron\'uncia que seja capaz de tratar nove
erros de pron\'uncia, provendo feedback ao usu\'ario sobre a qualidade de
sua pron\'uncia. Os erros foram selecionados com base nos trabalhos de
Zimmer (2004), Godoy (2005), Zimmer et al. (2009) e Crist\'ofaro-Silva
(2012), assumindo-se, como pron\'uncia padr\~ao, o General American (GA). A
abordagem de interl\'ingua foi selecionada para a arquitetura do
reconhecedor. O modelo ac\'ustico \'e, assim, alimentado com dados de fala
tanto de nativos, quanto de n\~ao-nativos, aprendizes de ingl\^es. No caso,
para os dados de nativos, ser\'a utilizado o TIMIT; e para os de
n\~ao-nativos, o COBAI e um corpus de leitura de senten\c{c}as foneticamente
balanceadas, ainda a ser compilado. As senten\c{c}as ser\~ao extra\'idas de um
corpus de aprendizes, o COMAprend. Um script em Python ser\'a utilizado
para realizar a convers\~ao grafema-fone das senten\c{c}as, tendo por base a
pron\'uncia can\^onica das palavras registrada no CMU Pronouncing
Dictionary. Na abordagem interlingual, tamb\'em, o modelo de pron\'uncia
deve ser alimentado com as variantes de pron\'uncia do aprendiz, de modo a
compor os chamados dicion\'arios multipron\'uncia. Para isso, pretende-se
utilizar o CMU Pronouncing Dictionary como base e adicionar as hip\'oteses
de pron\'uncia dos aprendizes por meio de regras transformacionais. O
modelo de l\'ingua ser\'a constitu\'ido por trigramas e gerado a partir da
Simple English Wikipedia., de modo a apresentar um sintaxe pr\'oxima à
produ\c{c}\~ao do aprendiz.

Um prot\'otipo foi elaborado de modo a avaliar a viabilidade do m\'etodo ora
proposto. O cronograma de execu\c{c}\~ao do projeto est\'a dentro do prazo, e as
bibliotecas e os softwares necess\'arios para sua execu\c{c}\~ao (HTK, Julius,
Adintool, Audacity, Praat, SoX) j\'a foram testados na elabora\c{c}\~ao do
prot\'otipo. Uma por\c{c}\~ao do COBAI (\textasciitilde{}3h40min) foi
segmentada, alinhada e analisada e utilizada para estimar o modelo
ac\'ustico. Tendo em vista o grande de n\'umero de arquivos do COBAI que
teve de ser desconsiderado (apenas \textasciitilde{}1h30min do que foi
segmentado pode ser utilizado), optou-se por se realizar a coleta de um
corpus de leitura de senten\c{c}as foneticamente balanceadas especificamente
para o desenvolvimento do projeto. De modo a simular, no prot\'otipo, os
dados que se obter\~ao com esse corpus, um corpus de erros induzidos
(\textasciitilde{}2h20min) foi gravado, segmentado, transcrito e
analisado. Tal corpus tamb\'em foi utilizado na estima\c{c}\~ao do modelo
ac\'ustico do prot\'otipo, juntamente com os dados do COBAI. O m\'etodo de
coleta e anota\c{c}\~ao do corpus real, de leitura de senten\c{c}as foneticamente
balanceadas, ser\'a definido em visita t\'ecnica à Universidade de Coimbra,
sob supervis\~ao da Profa. Sara Candeias, no per\'iodo de 28 de janeiro a 28
de fevereiro de 2014. O dicion\'ario-base do modelo de pron\'uncia, o CMU
Pronouncing Dictionary, j\'a foi testado no prot\'otipo, tendo sido
adicionadas variantes de pron\'uncia para um dos erros selecionados: a
simplifica\c{c}\~ao sil\'abica. Foi observado que houve um grande aumento no
n\'umero de entradas no modelo de pron\'uncia com a adi\c{c}\~ao das variantes de
pron\'uncia: o dicion\'ario cresceu de 1.855 palavras para 7.597. Sendo
assim, \'e poss\'ivel, ao se coligir todas as regras para os nove tipos de
erros, que o dicion\'ario cres\c{c}a fortemente, tornando o reconhecimento
confuso, bem como consumindo tempo e recursos computacionais. Caso isso
ocorra, uma solu\c{c}\~ao poss\'ivel seria criar dicion\'arios espec\'ificos para
cada tipo de erro, ou solicitar a um especialista que cerceie o
dicion\'ario, eliminando as variantes que ocorrem com menor frequ\^encia. Os
resultados iniciais obtidos com o prot\'otipo no reconhecimento
mostraram-se promissores.

O conte\'udo do projeto tem sido publicado na web{[}31{]} de modo a
dar-lhe visibilidade e angariar poss\'iveis colaboradores. Pretende-se,
tamb\'em, ao final do projeto, disponibilizar um pequeno sistema de treino
de pron\'uncia, como a prova de conceito para uso do reconhecedor de
pron\'uncia. Uma interface vem sendo desenvolvida para disponibilizar o
prot\'otipo na web, as Figura 19 e Figura 20 trazem algumas telas de
exemplo dessa interface.

                                [pic]

Figura 19: Interface do prot\'otipo na web - vis\~ao geral do site e tela de
captura do \'audio com espectro de frequ\^encia.

\textbar{} {[}i{]} \textbar{} {[}ii{]} \textbar{} \textbar{}{[}pic{]}
\textbar{}{[}pic{]} \textbar{}

Figura 20: Interface do prot\'otipo na web - {[}i{]} palavra reconhecida
com transcri\c{c}\~ao em formato IPA e em alfabeto adaptado; {[}ii{]} tela com
texto de feedback sobre a pron\'uncia do aprendiz, ap\'os ele reiterar no
erro.

H\'a tamb\'em a possibilidade de se realizar visita t\'ecnica ao Laborat\'orio
de Processamento de Sinais (LaPS) da Universidade Federal do Par\'a
(UFPA), na \'epoca de avalia\c{c}\~ao dos resultados finais da disserta\c{c}\~ao, sob
supervis\~ao do Prof.~Aldebaro Klautau, co-orientador da pesquisa.

\section{Limitations}\label{sec:limitations}
\section{Further Work}\label{sec:further-work}


%*****************************************
%*****************************************
%*****************************************
%*****************************************
%*****************************************
