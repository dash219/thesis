%*****************************************
\chapter{Conclusions}\label{ch:conclusions}
%*****************************************

\section*{Overall Conclusions}\label{sec:overall-conclusions}

This thesis sought to be an initial step in using deep linguistic knowledge to develop a \ac{CAPT} for Brazilian-accented English. Back in 2013, when this project started, one of the core challenges of researches in \ac{CAPT} was that \ac{ASR} systems were not precise enough in terms of phone recognition \cite{Witt2012}, what posed serious problems for \ac{CAPT} since phones are necessarily the basic signal that pronunciation training systems must rely on. Our first hypothesis was that by using as much phonetic knowledge in all stages of the pipeline for a \ac{CAPT}, especially in the \ac{ASR}, by training acoustic models that would be representantive of the phones in Brazilian-accented English and also by enriching the pronunciation dictionary with relevant mispronunciations, we would be able to provide a more accurate results for thi use-case, thus pushing the state of the art forward.

However as time went by, this initial project has proven to be undoable for a Master's Course not only with respect to the time or resources available, but also in terms of complexity. The project was too ambitious and results in \ac{ASR} are often exploratory. The truth is that \ac{ASR} is a huge interdisciplary area which, at present time, is still not solved. Even large IT companies which have at their disposal the best algorithms, computer power and speech corpora in the world still report \ac{WER} results around 12\% for conversational speech  \cite{Huang2014}. This means that short sentences with 4 words are fully recognized only 59\% of the times. 

Due to the scarcity of resources and the complexity of the task, replicating for Brazilian-accented English other methods that were successfully applied to other languages was unfeasiable. There was not enough data for training acoustic models, specialized language models or dictionaries which included entries with common mispronunciations. Our approach was then more conservative and focused on resources for Natural Language Processing and Pronunciation Evaluation that can help a future development of a \ac{CAPT} system for Brazilian-accented English. More especifically, our focus was on tasks for text-to-speech, spelling correction, corpus building and automatic pronunciation assessment -- our approach in all of them was to include enrich the models with phonetic knowledge in order to increase their performance. As it was shown in the papers presented, improvements were made and we were able to push the state of the art one step further in several occasions\footnote{The resources are available at the project website. Due to copyright reasons, the corpora used for training the acoustic models cannot be made available.  \emph{(http://nilc.icmc.usp.br/listener)}}:

\paragraph*{Text-to-speech}
  \begin{enumerate}
    \item The hybrid approach that we proposed to text-to-speech, which makes use of manual rules and machine learning, has proven to be quite efficient. Not only the results are in the state of the art, but also the approach is the quite flexible and scalable. The architecture of Aeiouado allows one to easily work on improving the accuracy/recall of the system by embedding more phonetic knowledge through reviewing the transcription rules, or by enalrging the test set, for instance, providing new examples for training.
    \item Supra-segmental information has shown to be useful feature for the machine learning classifier. As can be seen by the results in the paper \cite{Mendonca2014}, the model was able to learn difference between pretonic/tonic vs. postonic vowels. The f1-measure for [\ipa{U, @, I}] was 1.00, 0.99 and 0.97. 
    \item Part-of-speech were able to help the model to learn the difference heterophonic homograph pairs to a certain extent. The G2P was able to correctly infer heterophonic homograph pairs that were not seen during training, such as "ab[\textipa{o}]rto" (abortion) and "ab[\textipa{O}]rto" (I abort), or "ap[\textipa{o}]sto" (apposition) and "ap[\textipa{O}]sto" (I bet).
  \end{enumerate}

\paragraph*{Spelling correction}
  \begin{enumerate}
    \setcounter{enumi}{4}
    \item The hypothesis that phonetic knowledge could be used to improve the results of the speller due to phonologically-motivated errors was correct. The baseline system which had no phonetic module was able to correct 48.2\% of non-contextual phonologically motivated errors in the corpus. In comparison, one of the methods which considered the distance between phonetic transcriptions was  able to achieve 87.1\% accuracy in the same task.
    \item Our assumption about the types of mispellings also held. Around 18\% of the mispelling errors in user-generated content were phonologically-motivated. We compiled, annotated a released to the public a corpus for spelling correction tasks with 38,128 tokens, 4,083 of which containing mispelled words. It is worth noticing that there were no open corpora for this task in Portuguese, prior to this.
  \end{enumerate}

\paragraph*{Corpus building}
  \begin{enumerate}
    \setcounter{enumi}{5}
    \item Our hypothesis that greedy algorithms would be a suitable way for extracting phonetically-rich sentences from a corpora was supported. The results showed that the greedy strategy was capable of extracting sentences in a much more uniform way, while comparing to a random selection. For instance, the method was able to extract 854 new distinct triphones for a sample of 250 sentences, with almost twice the type/token ration of the random sample -- 0.61 vs. 0.32, respectively.
  \end{enumerate}

\paragraph*{Automatic pronunciation assessment}
  \begin{enumerate}
    \setcounter{enumi}{6}
    %\item Multipronunciation dictionaries with hand-written rules for generating variants are a reliable source of pronunciation information for pronunciation assessment.
    %\item Acoustic models trained on phonetically-rich speech corpora are able to provide more accurate phone models than those trained on balanced corpora.
    %\item Context free grammars can be adapted for forced-alignment recognition to list all pronunciation variants of a given word without hindering the performance of the \ac{ASR}.
    %\item Combined acoustic models (trained over native corpora + interlingua data) have phone models which are arobust enough to perform recognition in all languages used for training.
    %\item The additional pronunciation variants does not hurt the performance.
    \item TBD
  \end{enumerate}

\section*{Limitations and Further Work}\label{sec:limitations-further}

Although the research has reached its partial aims in building tools and resources for Natural Language Processing and Pronunciation Evaluation, there were still some unavoidable limitations. We conclude this thesis by discussing the limitations and providing ideas for future research.

\paragraph*{Text-to-speech}
Despite using part-of-speech information to capture the difference in heterophone homograph pairs, this did provide all context that mid vowels need. The worst results were related to the transcription of mid vowels [\ipa{E, e, O, o}]. Particularly, the model was very confused about mid-low, [\ipa{E}] showed an F1-score 0.66 and [\ipa{O}] of 0.71. In future works, it might be interesting to see if any other features could be used to improve the model performance in distinguishing the difference between [\ipa{e,o}] and [\ipa{E,O}], respectively. It can also be interesting to see if the errors are somehow related to vowel harmony (p\ipa{E}r\ipa{E}r\ipa{E}ca) or to suffixation (p\ipa{E} > p[\ipa{E}]zinho). One could also check whether training the models with more data would suffice.

\paragraph*{Spelling correction}
A considerable part of the typos that users make are phonologically-motivated, therefore phonetic transcription can be used to improve the coverage of spelling correction systems.

\paragraph*{Corpus building}
The method we proposed was able to extract sentences with much more uniform triphones as the type/token ration confirms, however, since the method basically favours rare triphones in each iteration, some of the sentences that were extracted had very awkward reading or uncommon words. Considering that the method is useful especially for preparing prompts to build speech recognition corpora, this poses a problem: if the prompt is not understood, the voice donor might hesitate or read the sentence in a way that is not natural. In the future, it might be interesting to develop some sort of filter during each iteration of the algorithm, in to remove these sentences with issues, while keeping the triphone balance. In addition to this, one of the main reasons for building the algorithm was to test whether phonetically-rich corpora would be more benefitial to phone recognition than phonetically-balanced corpora. Our hypothesis was that  since the phones are sampled more equally, acoustic models would be able to better estimate the phone properties, thus leading to improvements in tasks which demand robust phone models, such as phone recognition or forced-alignment. But due to time constraints this hypothesis could not be tested. If supported, this may be quite interesting for application which require very accurate phone models, such as \ac{CAPT} or speaker verification.

\paragraph*{Automatic pronunciation assessment}
Because of the time limit, the evaluation was conducted only on a small speaker-dependent corpus, recorded 


  %Multipronunciation dictionaries with hand-written rules for generating variants are a reliable source of pronunciation information for pronunciation assessment.
  %Acoustic models trained on phonetically-rich speech corpora are able to provide more accurate phone models than those trained on balanced corpora.
  %Context free grammars can be adapted for forced-alignment recognition to list all pronunciation variants of a given word without hindering the performance of the \ac{ASR}.
  %Combined acoustic models (trained over native corpora + interlingua data) have phone models which are arobust enough to perform recognition in all languages used for training.
  %The additional pronunciation variants does not hurt the performance.


%*****************************************
%*****************************************
%*****************************************
%*****************************************
%*****************************************
