%************************************************
\chapter{Introduction}\label{ch:introduction}
%************************************************

\section*{Setting and Motivation}

According to the International Monetary Found (IMF) \cite{IMF2015}, in 2015, Brazil as the seventh largest economy in the world with a GDP of US\$ 2.34 trillions. A survey by The Economist (2013) says that, since 2009, the growth of BRICS accounts for 55\% of the entire world economy growth. The current economic scenario is extremely favourable for Brazil to increase its global influence; however with regard to the ability to communicate globally, Brazil occupies a much more modest position. 

In 2015, Brazil ranked 41\textsuperscript{st} out of 70 countries in the English Proficiency Index (EF-EPI) \cite{EF2015}, classified among countries with low English proficiency, with 51.05 points. Scandinavian countries led the very high proficiency rankings, with Sweden (70.94) in the first position, Denmark (70.05) in third the spot and Norway (67.83) in fourth. Brazil performance was close to several other Latin America countries, such as Peru (52.46), Chile (51.88), Ecuador (51.67), Uruguay (50.25) and Colombia (46.54). The only exception in Latin America was Argentina that, despite the recent great depression was ranked 15\textsuperscript{th}, being classified as high proficiency, with a score of 60.26.

The \ac{EF-EPI} bands are aligned to the \ac{CEFR}, which is a guideline proposed by the Council of Europe to describe achievements of learners of foreign languages across the European Union. The \ac{CEFR} reference levels are described in \autoref{tab:cefr-levels}. \ac{EF-EPI} bands are mapped into \ac{CEFR} reference levels as follows: the very high proficiency band corresponds to \ac{CEFR} level B2; very low proficiency to A2; high, moderate and low proficiency bands to B1 with different punctuations. In case, Brazil's low proficiency rank is analogous to the \ac{CEFR} B1 level. 

\begin{table}[!htpb]
\newcolumntype{A}{>{\centering\arraybackslash}m{.12\textwidth}}
\newcolumntype{B}{>{\centering\arraybackslash}m{.12\textwidth}}
\newcolumntype{C}{>{\arraybackslash}m{.64\textwidth}}
\caption{CEFR reference levels.}
\scriptsize
\begin{center}
\begin{tabular}{ABC}
\hline
\textbf{ Group} & \textbf{Level} & \textbf{\centering Description} \\ \hline
\multicolumn{ 1}{A}{\textbf{Basic User
(A)}} & \multicolumn{ 1}{B}{\textbf{Beginner (A1)}} & Can understand and use familiar everyday expressions and very basic phrases aimed at the satisfaction of needs of a concrete type. \\ 
\multicolumn{ 1}{A}{} & \multicolumn{ 1}{B}{} & Can introduce him/herself and others and can ask and answer questions about personal details such as where he/she lives, people he/she knows and things he/she has. \\ 
\multicolumn{ 1}{A}{} & \multicolumn{ 1}{B}{} & Can interact in a simple way provided the other person talks slowly and clearly and is prepared to help. \\ \cline{ 2- 3}
\multicolumn{ 1}{A}{} & \multicolumn{ 1}{B}{\textbf{Elementary (A2)}} & Can understand sentences and frequently used expressions related to areas of most immediate relevance (e.g. very basic personal and family information, shopping, local geography, employment). \\ 
\multicolumn{ 1}{A}{} & \multicolumn{ 1}{B}{} & Can communicate in simple and routine tasks requiring a simple and direct exchange of information on familiar and routine matters. \\ 
\multicolumn{ 1}{A}{} & \multicolumn{ 1}{B}{} & Can describe in simple terms aspects of his/her background, immediate environment and matters in areas of immediate need. \\ \hline
\multicolumn{ 1}{A}{\textbf{Independent User 
(B)}} & \multicolumn{1}{B}{\textbf{Intermediate (B1)}} & Can understand the main points of clear standard input on familiar matters regularly encountered in work, school, leisure, etc. \\ 
\multicolumn{ 1}{A}{} & \multicolumn{ 1}{B}{} & Can deal with most situations likely to arise while traveling in an area where the language is spoken. \\ 
\multicolumn{ 1}{A}{} & \multicolumn{ 1}{B}{} & Can produce simple connected text on topics that are familiar or of personal interest. \\
\multicolumn{ 1}{A}{} & \multicolumn{ 1}{B}{} & Can describe experiences and events, dreams, hopes and ambitions and briefly give reasons and explanations for opinions and plans. \\ \cline{ 2- 3}
\multicolumn{ 1}{A}{} & \multicolumn{ 1}{B}{\textbf{Upper intermediate (B2)}} & Can understand the main ideas of complex text on both concrete and abstract topics, including technical discussions in his/her field of specialization. \\ 
\multicolumn{ 1}{A}{} & \multicolumn{ 1}{B}{} & Can interact with a degree of fluency and spontaneity that makes regular interaction with native speakers quite possible without strain for either party. \\ 
\multicolumn{ 1}{A}{} & \multicolumn{ 1}{B}{} & Can produce clear, detailed text on a wide range of subjects and explain a viewpoint on a topical issue giving the advantages and disadvantages of various options. \\ \hline
\multicolumn{ 1}{A}{\textbf{Proficient User
(C)}} & \multicolumn{ 1}{B}{\textbf{Advanced (C1)}} & Can understand a wide range of demanding, longer texts, and recognize implicit meaning. \\ 
\multicolumn{ 1}{A}{} & \multicolumn{ 1}{B}{} & Can express ideas fluently and spontaneously without much obvious searching for express \\ 
\multicolumn{ 1}{A}{} & \multicolumn{ 1}{B}{} & Can use language flexibly and effectively for social, academic and professional purposes. \\ 
\multicolumn{ 1}{A}{} & \multicolumn{ 1}{B}{} & Can produce clear, well-structured, detailed text on complex subjects, showing controlled use of organizational patterns, connectors and cohesive devices. \\ \cline{ 2- 3}
\multicolumn{ 1}{A}{} & \multicolumn{ 1}{B}{\textbf{Proficiency (C2)}} & Can understand with ease virtually everything heard or read. \\ 
\multicolumn{ 1}{A}{} & \multicolumn{ 1}{B}{} & Can summarize information from different spoken and written sources, reconstructing arguments and accounts in a coherent presentation. \\ 
\multicolumn{ 1}{B}{} & \multicolumn{ 1}{B}{} & Can express him/herself spontaneously, very fluently and precisely, differentiating finer shades of meaning even in the most complex situations. \\ \hline
\end{tabular}
\end{center}
\label{tab:cefr-levels}
\end{table}

As one might notice,  the B1 level describe someones who is usually able to understand familiar matters, deal with traveling situations, describe personal experiences and plans, and produce simple texts about subjects of personal interest. Needless to say, this is a very restricted communicative competence, which limits English usage primarily to the personal domain. 

With respect of Business English proficiency, Brazil performance is even more concerning. On the 
Business English Index (BEI) of 2013 \cite{BEI2013}, Brazil reached the 71\textsuperscript{st} position out of 77 countries analyzed. We attained a score of 3.27 points, in a scale from 1 to 10, being placed at the ``Beginner'' range, the lowest range considered by the index. We were close to countries such as El Salvador (3.24), Saudi Arabia (3.14) and Honduras (2.92) which up until recently had experienced civil wars or dictatorship governments. BEI describes individuals at the beginner level as those who ``can read and communicate using only simple questions and statements, but can't communicate and understand basic business information during phone calls''. Again, we can see that this is a very limited linguistic competence, that would not allow one not even to perform the most elementary day-to-day task in a company or industry work environment. 

Given this scenario, it is clear that we  need to improve English language proficiency among Brazilians. 
\ac{CALL} systems can be of great help in this scenario, these systems have several benefits, such as \cite{Witt1999}: 

\begin{itemize}
 \item Can provide undivided attention to the user, as opposed to a classrom environment, where teachers need to share attention among all students.
 \item Are cheap and scale well since students need is access to a computer.
 \item Enable asynchronous learning, allowing people with time and place constraints to study at their own pace and schedule.
 \item Permit high degree of individuality regarding the choice of material which is studied.
\end{itemize}

Obviously, there are also disadvantages. The limited interaction and feeback is often mentioned as a major problem in \ac{CALL}. Despite this, some studies have shown that students that used \ac{CALL} for training pronunciation training achieved equivalent results to students that were enroled in traditional classes, with teacher-led pronunciation training \cite{Neri2008, Stenson2013}.

\section*{Gap and objectives} 

This project seeks to be a primary step towards developing a \ac{CAPT} for Brazilian-accented English. The initial plan was to focus on methods for creating \ac{CAPT} systems for Brazilian-accent English by embedding as much phonetic knowledge as possible. As Witt \citep{Witt2012} points out in her survey on \ac{CAPT} from 2012, back then one of the core challenges in the area was that \ac{ASR} systems were not able to provide a reliable signal in terms of phoneme recognition. Our hypothesis was that by integrating phonetic knowledge in all stages of the pipeline for a \ac{CAPT}, one would be able to improve the accuracy of the system and provide more trustworthy results.

However as the project went through, with a broader view of the literature in \ac{CAPT} and a more profound understading of how \ac{ASR} works, such plan has shown to be unfeasible in the time given for a Master's Course. Due to the scarcity of resources, it was not possible, out of the blue, to replicate for Brazilian-accented English several of the existing methods that were applied to other languages in the literature. There were no specialized speech corpora for Brazilian-accented English, no language models trained from learners' texts and no dictionaries with pronunciation variants.

Instead of this, we decided to focus on investigating and building resources for Natural Language Processing and Pronunciation Evaluation that, in the future, may somehow help the development of \ac{CAPT} systems for Brazilian-accented English. Particularly we worked with tools and resources for text-to-speech, spelling correction, corpus building and automatic pronunciation assessment. In all cases, we attempted to integrate and embed phonetic knowledge in the pipeline.

\section*{Research Questions}

\paragraph*{Text-to-speech}
  \begin{enumerate}
    \item A hybrid-approach, which include both manual rules and machine learning, can be used to achieve state of the art results for \ac{G2P} conversion in Brazilian Portuguese.
    \item Supra-segmental information, such as lexical stress and syllable boundary are useful for improving the accuracy of a \ac{G2P} for \ac{BP}, especially with regard to the vowel transcription.
    \item Part-of-speech tags are able to provide enough information for a hybrid \ac{G2P} to distinguish between heterophonic homograph pairs, such as "gov[\textipa{e}]rno" (government) "gov[\textipa{E}]rno" (I govern).
  \end{enumerate}

\paragraph*{Spelling correction}
  \begin{enumerate}
    \setcounter{enumi}{4}
    \item A considerable part of the typos that users make are phonologically-motivated, therefore phonetic transcription can be used to improve the coverage of spelling correction systems.
  \end{enumerate}

\paragraph*{Corpus building}
  \begin{enumerate}
    \setcounter{enumi}{5}
    \item Greedy algorithms can help to produce richer speech corpora by making local optimal decisions, through analyzing the triphone sequences of the sentences in a corpus and extracting those which keep the triphone distribution as uniform as possible in each iteration.
  \end{enumerate}

\paragraph*{Automatic pronunciation assessment}
  \begin{enumerate}
    \setcounter{enumi}{6}
    \item Multipronunciation dictionaries with hand-written rules for generating variants are a reliable source of pronunciation information for pronunciation assessment.
    \item Acoustic models trained on phonetically-rich speech corpora are able to provide more accurate phone models than those trained on balanced corpora.
    \item Context free grammars can be adapted for forced-alignment recognition to list all pronunciation variants of a given word without hindering the performance of the \ac{ASR}.
    \item Combined acoustic models (trained over native corpora + interlingua data) have phone models which are arobust enough to perform recognition in all languages used for training.
    \item The additional pronunciation variants does not hurt the performance.
  \end{enumerate}

\section*{Contributions of the Thesis} 

Within this work, we have investigated and developed a set of tools and resources which integrate phonetic knowledge -- and benefit from it. Some of these tools were created and tested in tasks related to processing Brazilian-accented English or Brazilian Portuguese, but their architecture and methods are certainly scalable to other languages or scenarios. The full list of contributions is provided below\footnote{All files, resources and scripts developed are available at the project website. Due to copyright reasons, the corpora used for training the acoustic models cannot be made available.  \emph{(http://nilc.icmc.usp.br/listener)}}:

\begin{enumerate}
 \item \emph{Aeiouad\^o G2P}: A grapheme-to-phoneme converter for \ac{BP} which uses a hybrid approach, based on both handcrafted rules and machine learning method, as decribed in \citeauthor{Mendonca2014} \cite{Mendonca2014}. \emph{Aeiouad\^o dictionary}: A large machine readable dictionary for \ac{BP}, compiled from a word list extracted from the Portuguese Wikipedia, which was preprocessed in order to filter loanwords, acronyms, scientific names and other spurious data, and then transcribed with Aeiouad\^o G2P).
 \item A phonetic speller for user-generated content in \ac{BP}, based on machine learning, which takes advantage of Aeiouad\^o G2P to group phonetically related words, as described in \citeauthor{Mendonca2015} \cite{Mendonca2015}; 
 \item A method for the extraction of phonetically-rich sentences, i.e. sentences with a high variety of triphones distributed in a uniform fashion, which employs a greedy algorithm for comparing triphone distributions among sentences, as described in \citeauthor{Mendonca2014b} \cite{Mendonca2014b};
 \item \emph{Listener}:A prototype system for automatic speech recognition and evaluation of Brazilian-accented English, which makes use of forced alignment, \ac{HMM}/\ac{GMM} acoustic models, context free grammars and multipronunciation dictionaries \cite{Mendonca2016};
\end{enumerate}

\section*{Thesis Structure}

This Master's thesis is organized into four chapters and follows the structure of a sandwich thesis, consisting of a collection of published or in-press articles. Chapter ~\ref{ch:foundations} presents the theoretical fundations, with an introduction to phonetics and phonology. Chapter ~\ref{ch:articles} contains all articles that were published during the scope of this work or which are in-press. Section ~\ref{sec:aeiouado} presents the Aeiouad\^o's dictionary and \ac{G2P} converter, which was built uppon a hybrid strategy for converting graphemes into phones, with manual transcription rules, as well as machine learning. Section ~\ref{sec:speller} presents a use-case of Aeiouad\^o, namely a phonetic-speller which employs the transcriptions generated by the grapheme-to-phoneme converter. Section ~\ref{sec:phon-rich} proposes a method for the extraction of phonetically-rich sentences which can be used for building more representative speech corpora. Section ~\ref{sec:listener} describes a prototype system for non-native speech recognition and evaluation of Brazilian-accented English, which makes use of the tools and resources developed in this thesis. Finally in Chapter ~\ref{ch:conclusions}, we present the overall conclusions, some limitations we found, together with the next steps for future work. A glossary can be found at the back of the thesis in order to help the reader with uncommon terms or specialized jargon.