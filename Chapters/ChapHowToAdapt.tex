%************************************************
\chapter{How to Adapt a Speech Recognition to Non-Native Data}\label{ch:how-to-adapt}
%************************************************
Lorem ipsum quod dolor sit amet.

\section{PB}


2 Adapta\c{c}\~ao a Dados de N\~ao-nativos

\'E ineg\'avel que as tecnologias de reconhecimento de fala, mesmo as do
estado da arte, apresentam problemas. Por tal raz\~ao, muitos
pesquisadores mostram- se c\'eticos quanto à efici\^encia do reconhecimento
de fala de n\~ao-nativos. Entretanto, as cr\'iticas que, geralmente, s\~ao
imputadas a sistemas de reconhecimento de fala de n\~ao-nativos, conforme
apontam Neri et al. (2003), s\~ao fruto da falta de familiaridade com o
design de reconhecedores de fala. De fato, caso se tente utilizar um
reconhecedor de fala, projetado para nativos, com n\~ao-nativos, o
desempenho ser\'a baixo, tendo em vista que o reconhecedor n\~ao est\'a
preparado para os padr\~oes ac\'ustico-articulat\'orios que o n\~ao-nativo
produzir\'a. Entretanto, h\'a diversos m\'etodos para se adaptar um sistema de
RAF a dados de n\~ao-nativos. Conforme apontam Strik e Cucchiarini (1999),
varia\c{c}\~oes de pron\'uncia de falantes n\~ao-nativos podem ser adicionadas a
qualquer n\'ivel do reconhecedor: no modelo ac\'ustico, no modelo de l\'ingua
ou no modelo de pron\'uncia. Tais modelos s\~ao apresentados nas tr\^es se\c{c}\~oes
seguintes.

3 Adapta\c{c}\~ao do Modelo Ac\'ustico (MA)

No que concerne ao modelo ac\'ustico do reconhecedor, tr\^es m\'etodos t\^em
sido, comumente, empregados para tratar dados de fala de n\~ao-nativos:
(i) adapta\c{c}\~ao ao falante; (ii) constru\c{c}\~ao de modelos bil\'ingues; e (iii)
utiliza\c{c}\~ao de modelos combinados, ou de interl\'ingua (Wang, et al.,
2003). Tais m\'etodos distinguem-se quanto à origem dos dados ac\'usticos
utilizados para treinar o modelo. Na adapta\c{c}\~ao ao falante, s\~ao
utilizados apenas dados de fala de falantes nativos da l\'ingua alvo. Na
constru\c{c}\~ao de modelos bil\'ingues, empregam-se no treinamento do modelo
ac\'ustico dados de fala de falantes nativos tanto da l\'ingua alvo, quanto
da l\'ingua base. Por fim, nos modelos combinados, ou de interl\'ingua, o
modelo ac\'ustico \'e treinado tendo em vista dados de falantes nativos da
l\'ingua alvo e, tamb\'em, de aprendizes. A Figura 12 resume as tr\^es
abordagens dispon\'iveis para adaptar o modelo ac\'ustico de um sistema de
RAF.

                                [pic]

Figura 12: M\'etodos para se adaptar o Modelo Ac\'ustico (MA) do
reconhecedor a dados de n\~ao-nativos.

A fim de melhor esclarecer a distin\c{c}\~ao que h\'a entre os tr\^es m\'etodos,
consideremos a cria\c{c}\~ao de um sistema de reconhecimento de pron\'uncia que
tenha por fim reconhecer o ingl\^es americano falado por falantes nativos
de PB. Na t\'ecnica de adapta\c{c}\~ao ao falante, para esse reconhecedor,
apenas dados de falantes nativos de ingl\^es seriam utilizados no
treinamento do modelo. Na abordagem bil\'ingue, o modelo ac\'ustico seria
composto de forma dual, possuindo dados de fala de ambas as l\'inguas:
tanto do ingl\^es americano, quanto do portugu\^es brasileiro. Na utiliza\c{c}\~ao
de modelos combinados ou de interl\'ingua, o modelo ac\'ustico seria
alimentado com dados nativos da l\'ingua alvo, no caso, americanos falando
ingl\^es, e tamb\'em com dados de aprendizes, isto \'e, brasileiros falando
ingl\^es.

Diversos m\'etodos e algoritmos podem ser empregados para realizar uma das
tr\^es abordagens de adapta\c{c}\~ao, a exemplo de: Maximum Likelihood Linear
Regression (MLLR) + Maximum A Posteriori (MAP) + Identifica\c{c}\~ao de fones
mais informativos (Oh et al., 2006), Phonetic Decision Tree (PDT) (Chen
\& Cheng, 2012), Polyphone Decision Tree Specialization (PDTS) (Wang et
al., 2003), Eigenvoices + MLLR (Tan \& Besacier, 2007), Phoneset comum +
Modelo multil\'ingue (Fischer et al., 2002). Nesta disserta\c{c}\~ao, propomos a
utiliza\c{c}\~ao de modelos combinados. De tal modo, ne

4 Adapta\c{c}\~ao no Modelo de Pron\'uncia (MP)

Em sistemas de RAF, os dicion\'arios de pron\'uncia constituem o m\'odulo que
cont\'em as palavras do l\'exico do reconhecedor, juntamente com sua forma
fon\'etica transcrita. Trata-se, em verdade, do componente do reconhecedor
que faz a ponte entre as unidades ac\'usticas subpalavras presentes no
modelo ac\'ustico e as poss\'iveis sequ\^encias de palavras especificadas no
modelo de l\'ingua. O Quadro 5 ilustra a estrutura do CMUdict{[}15{]}
(Weide, 1998), um dicion\'ario de pron\'uncia de refer\^encia para o ingl\^es
americano.

              Quadro 5: Exemplo de entradas no CMUdict.

                                [pic]

Como se observa, um dicion\'ario constitui uma lista de palavras, que
cont\'em formas ortogr\'aficas e fon\'eticas pareadas, al\'em de um
identificador. A fun\c{c}\~ao do identificador \'e possibilitar a distin\c{c}\~ao de
palavras hom\'ografas heter\'ofonas, isto \'e, palavras que possuem mesma
grafia mas pron\'uncia distinta, como gov{[}e{]}rno (nome) e gov{[}?{]}rno
(verbo); bem como a distin\c{c}\~ao das pron\'uncias variantes de uma palavra.

No que diz respeito ao RAF de n\~ao-nativos, a adapta\c{c}\~ao que se costuma
fazer ao dicion\'ario de pron\'uncia \'e a adi\c{c}\~ao das formas variantes de
pron\'uncia do n\~ao-nativo, de modo a construir os chamados dicion\'arios
multipron\'uncia. A constru\c{c}\~ao de tais dicion\'arios pode se dar por meio de
duas abordagens: (i) baseada em conhecimento ou (ii) baseada em dados
(Strik \& Cucchiarini, 1999). Ambas possuem seus pr\'os e contras.

Na abordagem baseada em conhecimento, variantes de pron\'uncia s\~ao
inseridas no dicion\'ario do reconhecedor por meio de regras geradas por
um especialista - um linguista. A l\'ingua constitui uma heterogeneidade
ordenada, isto \'e, a varia\c{c}\~ao n\~ao \'e um processo que ocorre
aleatoriamente, h\'a diversos fatores, sejam eles estruturais, sejam
sociais, que condicionam a varia\c{c}\~ao lingu\'istica (Weinreich, et al.,
2006). Portanto, cabe ao especialista explicitar a estrutura que subjaz
à varia\c{c}\~ao lingu\'istica, inferindo as regras fonol\'ogicas que melhor
descrevem as variantes. Tais regras, em geral, assumem o formato:

                            $A > B / C_D$,

em que A representa o elemento a ser modificado, B o elemento j\'a
modificado, e C\_D o contexto de aplica\c{c}\~ao da regra, sendo C o contexto
estrutural à esquerda e D o contexto à direita (Crystal, 2008). Regras
fonol\'ogicas, portanto, s\~ao capazes de descrever a varia\c{c}\~ao fon\'etica em
uma escala segmental (Wester, 2003). Considere-se, como exemplo, o caso
da ep\^entese de vogais altas anteriores n\~ao-arredondadas {[}i{]} em
s\'ilabas contendo oclusivas em coda, por brasileiros aprendizes de
ingl\^es. Por transfer\^encia de padr\~oes fonot\'aticos de L1 para L2,
brasileiros tendem a realizar ep\^entese em palavras que apresentam
oclusivas em posi\c{c}\~ao coda, que s\~ao proibidas, originalmente, na
fonologia do portugu\^es (Collischonn, 2003; Zimmer \& Alves, 2006). Sendo
assim, acabam por pronunciar ``book'' como boo{[}ki{]} em vez de
boo{[}k{]} e ``trip'' como tri{[}pi{]} em vez de tri{[}p{]}. Tal fato
pode ser capturado pela regra:

                      $COCLUS > COCLUS+[i] / _\$$,

em que COCLUS representa uma consoante oclusiva e ``\$'' a fronteira
sil\'abica. Na constru\c{c}\~ao de dicion\'arios multipron\'uncia baseados em
conhecimento, o especialista deve arrolar um conjunto suficiente de
regras, de forma a acomodar as diversas varia\c{c}\~oes de pron\'uncia que um
falante n\~ao-nativo apresenta.

Trata-se, portanto,  de  uma  tarefa  custosa,  quer  financeira  quer
temporalmente, uma vez que pressup\~oe um extenso levantamento e an\'alise
de dados lingu\'isticos ou uma consulta dicion\'arios transcritos j\'a
compilados. Al\'em disso, apesar de os dados obtidos com um especialista
serem fi\'aveis, frequentemente, o levantamento feito \'e incompleto, j\'a que
diversos fen\^omenos de varia\c{c}\~ao lingu\'istica ainda est\~ao para ser
estudados, n\~ao havendo literatura dispon\'ivel (Wester, 2003). Em outras
palavras, regras desenvolvidas por um especialista possuem alta precis\~ao
(precision), mas n\~ao necessariamente alta cobertura (recall). Ademais, a
abordagem baseada em conhecimento \'e dependente de l\'ingua ou, em certos
casos, de um dialeto; regras desenvolvidas para uma l\'ingua ou dialeto
n\~ao s\~ao necessariamente aplic\'aveis a outras l\'inguas ou dialetos.

A abordagem baseada em dados para a constru\c{c}\~ao de dicion\'arios
multipron\'uncia pode ser classificada como direta ou indireta (Kim, Oh,
\& Kiem, 2007). Na direta, padr\~oes de varia\c{c}\~ao existentes em um conjunto
de teste s\~ao analisados e utilizados, imediatamente, para gerar as
palavras variantes. J\'a na indireta, busca-se inferir regras que possam
ser aplicadas na gera\c{c}\~ao de uma ou mais variantes para uma palavra. A
abordagem direta, portanto, at\'em-se aos padr\~oes de varia\c{c}\~ao que ocorrem
no conjunto de teste, enquanto a indireta \'e capaz de prever padr\~oes que
n\~ao vieram a ocorrer.

Um dos problemas da utiliza\c{c}\~ao de dicion\'arios multipron\'uncia \'e o aumento
da incerteza do reconhecedor. Com a adi\c{c}\~ao de variantes ao dicion\'ario de
pron\'uncia, aumenta-se a cardinalidade do espa\c{c}o amostral de palavras que
o reconhecedor deve percorrer na busca e, por conseguinte, aumenta-se
sua confus\~ao. Em s\'intese, ao se adicionar muitas formas variantes ao
dicion\'ario de pron\'uncia, a confus\~ao inserida no modelo pode
contrabalancear os ganhos com uma forma fon\'etica mais precisa para as
palavras (Compernolle, 2001). Dicion\'arios multipron\'uncia, de tal forma,
devem procurar o ponto ideal entre o n\'umero de formas variantes
adicionadas e as formas can\^onicas. Kim et al. (2007) prop\~oem uma m\'etrica
para avaliar a confus\~ao de dicion\'arios multipron\'uncia, baseada no n\'umero
de variantes de pron\'uncia de cada palavra, nos fones que a comp\~oem e na
dist\^ancia de Levenshtein entres as palavras. Tal m\'etrica, chamada
Confusability Measure (CM), \'e definida formalmente como: seja {[}pic{]}
um dicion\'ario multipron\'uncia, composto por {[}pic{]} palavras, de
maneira que cada palavra {[}pic{]} possua {[}pic{]} variantes de
pron\'uncia, sendo {[}pic{]} a j-\'esima variante de pron\'uncia pertencente à
i- \'esima palavra, ent\~ao:

{[}pic{]}

onde {[}pic{]} \'e a dist\^ancia de Leveshtein entre {[}pic{]} e {[}pic{]},
e {[}pic{]} \'e o n\'umero de fones da variante de pron\'uncia {[}pic{]},
normalizado pelo n\'umero total de fones de todas as pron\'uncias de
{[}pic{]}, tal que:

                                [pic]

onde {[}pic{]} \'e definido como o n\'umero de fones da pron\'uncia {[}pic{]}.
O objetivo da CM \'e sistematizar a constru\c{c}\~ao de dicion\'arios
multipron\'uncia, estabelecendo um crit\'erio objetivo para a sele\c{c}\~ao das
palavras que devem compor o modelo. Palavras distintas, mas com contexto
fon\'etico similar s\~ao favorecidas pela m\'etrica, podendo-se selecion\'a-las
ao estabelecer um threshold.

Em muitas vezes, os ganhos em WER obtidos com a constru\c{c}\~ao de modelos de
pron\'uncia s\~ao baixos em raz\~ao da adi\c{c}\~ao ``cega'' de palavras ao
dicion\'ario. Jurafsky et al. (2001) demonstraram, por exemplo, que o
relaxamento de vogais no ingl\^es e a substitui\c{c}\~ao de segmentos s\~ao,
automaticamente, capturados por modelos ac\'usticos baseados em trifones,
n\~ao sendo necess\'ario, portanto, adicionar variantes ao dicion\'ario.
Outros tipos de varia\c{c}\~ao, como o apagamento de s\'ilabas, n\~ao s\~ao bem
modelados em trifones, de modo que \'e necess\'ario trat\'a-los no dicion\'ario
de pron\'uncia (Jurafsky, et al., 2001).

Diversas t\'ecnicas podem ser utilizadas para elaborar um dicion\'ario
multipron\'uncia, a exemplo de: Gaussian Densities Across Phonetic Models
(Sara\c{c}lar \& Khudanpur, 2000), Group Delay Based Segmentation (Brunet \&
Murthy, 2012), Phones Adaptation and Pronunciation Generalization (Ahmed
\& Ping, 2011), Automatic Generation of Accented Variants + MLLR
(Goronzy \& Eisele, 2003), Multi-span Linguistic Parse Tables (Mertens
et al., 2011), Decision Trees (Byrne et al., 1998), Pronunciation
Mixture Model (McGraw et al., 2008). A seguir, discutimos, em mais
detalhe, tr\^es desses trabalhos, notadamente: Sara\c{c}lar e Khudanpur
(2000), Matsunaga et al. (2003) e Kim et al. (2007).

Sara\c{c}lar \& Khudanpur (2000) prop\~oem uma t\'ecnica, a que chamam state
level pronunciation model (SLPM), que estima as pron\'uncias variantes de
uma palavra sem a inser\c{c}\~ao de variantes no dicion\'ario de pron\'uncia. Os
autores elaboraram um m\'etodo que permite que a representa\c{c}\~ao de um
fonema /f/, por exemplo, seja, no modelo ac\'ustico, diretamente mapeada
nos estados HMM de suas variantes {[}f1{]} e {[}f2{]}, criando assim um
modelo misto. Tal t\'ecnica \'e capaz de gerar variantes de pron\'uncia
automaticamente, mas pressup\~oe a utiliza\c{c}\~ao de dados anotados de forma
manual para bootstrap do modelo. O state level pronunciation model
(SLPM) foi testado com dados do ingl\^es americano, em uma por\c{c}\~ao do
corpus Switchboard (\textasciitilde{}4 horas, \textasciitilde{}100k
fones). O incremento em WER reportado foi de 1,7\% (valor absoluto), tal
valor corresponde à melhoria no reconhecedor dos valores de WER sem um
modelo de pron\'uncia (39,4\%) e com um modelo de pron\'uncia do tipo state
level pronunciation model (SLPM) (37,7\%).

  Matsunaga et al. (2003) utilizam dicion\'arios multipron\'uncia e  modelos

ac\'usticos combinados para tratar o ingl\^es falado por japoneses. Os
autores conduziram testes utilizando cinco m\'etodos: (i) l\'exico do ingl\^es
+ modelo ac\'ustico nativo; (ii) l\'exico com influ\^encia do japon\^es + modelo
ac\'ustico japon\^es; (iii) compara\c{c}\~ao de ambos os m\'etodos; (iv) l\'exico
japon\^es e ingl\^es + modelo ac\'ustico combinado; (v) l\'exico combinado +
modelo ac\'ustico combinado + penalidade para altern\^ancia de palavras
entrel\'inguas.

  Kim et al. (2007) propuseram um m\'etodo autom\'atico  de  cria\c{c}\~ao  de  um

dicion\'ario multipron\'uncia, atrav\'es de uma abordagem indireta baseada em
dados. De modo a criar um sistema de RAF n\~ao-nativo para coreanos
aprendizes de ingl\^es, os autores expandiram as formas can\^onicas do
reconhecedor e, ent\~ao, utilizaram uma medida de confus\~ao (CM), baseada
na dist\^ancia de Levenshtein, para atribuir um valor a cada variante de
pron\'uncia e excluir aquelas que apresentavam baixa confus\~ao. As
variantes foram geradas atrav\'es do seguinte procedimento: 1º) as
senten\c{c}as n\~ao- nativas foram reconhecidas atrav\'es de um reconhecedor de
fones; 2º) a sa\'ida do reconhecedor foi alinhada com a forma de pron\'uncia
can\^onica esperada atrav\'es de um algoritmo de programa\c{c}\~ao din\^amica
(algoritmo de Viterbi); 3º) padr\~oes de pron\'uncia variantes foram obtidos
a partir dos dados alinhados; 4º) regras de varia\c{c}\~ao de pron\'uncia s\~ao
derivadas dos padr\~oes de varia\c{c}\~ao, por meio de \'arvores de decis\~ao
(algoritmo C4.5); 5º) as regras de varia\c{c}\~ao s\~ao aplicadas ao restante do
l\'exico, de modo a gerar poss\'iveis candidatas a variantes de pron\'uncia. A
medida de confus\~ao (CM) \'e ent\~ao utilizada, de modo a excluir do modelo
de pron\'uncia as variantes que apresentam baixa confus\~ao. O corpus
utilizado no experimento foi uma por\c{c}\~ao do Wall Street Journal Database
(WSJ0) (\textasciitilde{}7.138 senten\c{c}as). A melhoria na taxa de WER
reportada foi de 1,34\% (valor absoluto), que corresponde a uma varia\c{c}\~ao
de 19,92\% a 18,58\%. O baseline utilizado foi uma por\c{c}\~ao de 340
palavras do CMU Pronouncing Dictionary.

5 Adapta\c{c}\~ao no Modelo de L\'ingua (ML)

{[}Esta se\c{c}\~ao do trabalho ser\'a realizada posteriormente, ap\'os elaborados
os modelos ac\'ustico e de pron\'uncia.{]}

\section{Adaptation Methods for the Acoustic Model}
Lorem ipsum quod dolor sit amet.

\section{Adaptation Methods for the Pronunciation Model}
Lorem ipsum quod dolor sit amet.

\section{Adaptation Methods for the Languaeg Model}
Lorem ipsum quod dolor sit amet.


%*****************************************
%*****************************************
%*****************************************
%*****************************************
%*****************************************
