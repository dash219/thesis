%*****************************************
\chapter{Copy of the articles}\label{ch:articles}
%*****************************************

This chapter presents a copy of all papers which were originally produced within the period of the Master's course. All papers are based on the premise that linguistic knowledge can be used to push the state-of-the-art of NLP tasks one step further. Since our focus was on speech and on pronunciation training, this means phonetic knowledge.


\begin{itemize}
\item \textbf{Section ~\ref{sec:aeiouado}:} \bibentry{Mendonca2014}
\item \textbf{Section ~\ref{sec:speller}:} \bibentry{Mendonca2014b}
\item \textbf{Section ~\ref{sec:phon-rich}:} \bibentry{Mendonca2015}
\item \textbf{Section ~\ref{sec:listener}:} \bibentry{Mendonca2016}
\end{itemize}


\cleardoublepage
%*****************************************

\invisiblesection{Using a hybrid approach to build a pronunciation dictionary for Brazilian Portuguese}
\includepdf[pages={1-},scale=0.90, pagecommand={}]{Chapters/PaperAeiouado.pdf}\label{sec:aeiouado}
%%*****************************************
\chapter{Aieouad\^o's dictionary and G2P converter}\label{ch:aieouado-g2p}
%*****************************************

\footnote{This section contains is an extended version of the paper}
Aeiouad\^o is a product and an original contribution of this Master's research. It was developed by \citeauthor{Mendonca2014} \cite{Mendonca2014}, after noticing that no prior work applied machine learning algorithms to grapheme to phoneme conversion in \ac{BP}.

To the best of our knowledge, all previous efforts to solve the G2P problem in \ac{BP} used only a rule-based approach, and as such, they require complex linguistic knowledge to code rules and are difficult to evaluate. Unfortunately, it is a common practice in rule-based G2P (and a wrong one!) to develop the transcription rules based on a corpus and to evaluate them on the same corpus. This usally leads to overfitting, so that the results which are reported (some of them which approach 100\%) are hardly those which are found when such systems are applied to other data or tested in a different context. In addition to this, most of the G2P systems for \ac{BP} are proprietary or licensed under commercial terms, therefore their use is restricted.

We tried to overcome this gap by developing a hybrid G2P converter, which makes use of both rules and machine learning algorithms. The converter was evaluated with traditional machine learning metrics, such as precision, recall and F1-measure; therefore, its reported metrics are trustworthy. This converter was used to build a wide pronunciation dictionary, based on wordlist extracted from Wikipedia articles. Both the dictionary and the G2P converter are distributed under permissive licenses, respectivelly \ac{CC} and \ac{BSD}. Therefore, these resources aim at promoting the development of novel speech technologies for Brazilian Portuguese. It is interesting to notice that, despite being 6\textsuperscript{th} most spoken language in the world \cite{Ethnologue2013}, with about 200 million speakers, speech recognition and speech synthesis for Brazilian Portuguese are far from the current state of the art, specially what regard to open tools and resources \cite{Neto2011}. 

Aeiouad\^o's dictionary and G2P converter are based on the dialect of the city of S\~ao Paulo. They were designed primarily for Speech Technologies, such as \ac{ASR} an \ac{TTS}; but might also be used by linguists, speech therapists, lexicographers, students of Brazilian Portuguese as a second language, and whoever is interested in the sound structure of \ac{BP}.

The G2P converter makes use of a hybrid approach for grapheme to phoneme conversion, based on both manual transcription rules and machine learning algorithms. Hybrid approaches in grapheme to phoneme conversion have been applied successfully to other languages \cite{Damper1998, Polyakova2006, Teixeira2006, Veiga2013}. They have the benefit of taking advantage from both knowledge-based and data-driven methods. We proposed a method in which the phonetic transcription of a given word is obtained through a two-step procedure. Its primary word list derives from the Portuguese Wikipedia dump of 23\textsuperscript{rd} January 2014. We decided to use Wikipedia as the primary word list for the dictionary for many reasons: i) given its encyclopedia nature, it covers wide-ranging topics, providing words from both general knowledge and specialized jargon; ii) it contains around 168,8 million word tokens, being robust enough for the task; iii) it makes uses of crowdsourcing, lessening author's bias; iv) its articles are distributed through Creative Commons License. Wikipedia articles were transformed into plain text, tokenized and word types were extracted.

The dictionary makes use of a hybrid approach for grapheme to phoneme conversion, based on both manual transcription rules and machine learning algorithms, and aims at promoting the development of novel speech technologies for Brazilian Portuguese. Hybrid approaches in grapheme to phoneme conversion have been applied successfully to other languages \cite{Damper1998}\cite{Polyakova2006}\cite{Teixeira2006}\cite{Veiga2013}. They have the benefit of taking advantage from both knowledge-based and data-driven methods. We propose a method in which the phonetic transcription of a given word is obtained through a two-step procedure. Its primary word list derives from the Portuguese Wikipedia dump of 23\textsuperscript{rd} January 2014. We decided to use Wikipedia as the primary word list for the dictionary for many reasons: i) given its encyclopedia nature, it covers wide-ranging topics, providing words from both general knowledge and specialized jargon; ii) it contains around 168,8 million word tokens, being robust enough for the task; iii) it makes uses of crowdsourcing, lessening author's bias; iv) its articles are distributed through Creative Commons License. Wikipedia articles were transformed into plain text, tokenized and word types were extracted.

We developed a language identifier in order to detect loanwords among data. It is a known fact that when languages interact, linguistic exchanges inevitably occur. One particular type of linguistic exchange is of great concern while building a pronunciation dictionary, namely, non-assimilated loanwords \cite{Bussmann96}. Non-assimilated loanwords stand for lexical borrowings in which the borrowed word is incorporated from one language into another straightforwardly, without any translation or orthographic adaptation. These words represent a problem to grapheme-to-phoneme (G2P) conversion since they show orthographic patterns which are not predicted in advance by rules or which are too deviant to be captured by machine learning algorithms. Many algorithms have been proposed to address Language Identification (LID) from text \cite{Bergsma2012, Bilcu2006, Dolf2012, Zampieri2012}. Since our goal is to detect the language of single words, we employed n-gram character  models in the identifier, given its previous success in dealing with short sequences of characters. 

Brazilian Portuguese Phonology can be regarded as syllable and stress-driven \cite{Cristofaro2005}. In fact, many phonological processes in Brazilian Portuguese are related to or conditioned by syllable structure and stress position \cite{Girelli1990}. Vowel harmony occurs in pretonic context \cite{Bisol1989}, posttonic syllables show a limited vowel inventory \cite{Cristofaro2005}, nasalization occurs when stress syllables are followed by nasal consonants \cite{Quicoli1990}, epenthesis' processes are triggered by the occurrence of non-allowed consonants in coda position \cite{Delatorre2005} and so on and so forth. Therefore, detecting syllable boundaries and stress is of crucial importance for G2P systems, in order to achieve correct transcriptions. Several algorithms have been proposed to deal with the syllabification in Brazilian Portuguese. However most of them were not extensively evaluated nor were made publicly available \cite{Oliveira2005, Nhenhem2012, Neto2011, Rocha2013}. For this reason, we implemented our own syllabification algorithm, 
based directly on the rules of the last Portuguese Language Orthographic Agreement \cite{Acordo2009}. 

Word types recognized as belonging to Brazilian Portuguese by the language identifier were transcribed in a two-step process: i) words are submitted to a set of transcription rules, in which predictable graphemes (mostly consonants) are transcribed; ii) a machine learning classifier is used to predict the transcription of the remaining graphemes (mostly vowels). All the data were subsequently revised. Figure 1 summarizes the method.

\begin{figure}[t]
\centerline{ \includegraphics[width=8cm]{./gfx/aeiouado-flowchart-mod.pdf}}
\caption{{\it System architecture for building the pronunciation dictionary.}}
\label{g2p-architecture}
\end{figure}

We used the Portuguese Wikipedia's dump of 23\textsuperscript{rd} January 2014 as the primary word list for the pronunciation dictionary. In order to obtain plain text from the articles, we employed WikiExtractor \cite{Wikiextractor2013}; it strips all the  MediaWiki markups and metadata forms. Afterwards, texts were tokenized and unique words types extracted. The Portuguese Wikipedia has about 168,8 million word tokens and 9,7 million types, distributed among  820,000 articles. With the purpose of avoiding misspellings, URLs and other spurious data, only words with frequency higher than 10, which showed neither digits nor punctuation marks were selected. 

A Language Identifier module was developed in order to detect loanwords in the pronunciation dictionary. The Identifier consists of a Linear Support Vector Machine Classifier \cite{Steinwart2008} and was implemented in Python, through Scikit-learn \cite{Scikit2011}. It was trained on a corpus made of the 200,000, containing 100,000 Brazilian Portuguese words and 20,000 words of each of the following languages: English, French, German, Italian and Spanish. All of these words were collected through web crawling News' sites and were not revised. We selected these languages because they are the major donors of loanwords to Brazilian Portuguese \cite{Alves2001}. From these words we extracted features such as initial and final bi- and trigraphs;  number of accented graphs, vowel-consonant ratio; average mono-, bi- and trigraphs probability; and used them to estimate the classifier. Further details can be found in the website of the Project\footnote{http://nilc.icmc.usp.br/listener/aeiouado}. After training, we applied the classifier to the Wikipedia word list with the purpose of identifying loanwords among data. The identified loanwords were then separated from the rest of words for later revision, i.e. they were not submitted to automatic transcription.

Our syllabification algorithm follows a rule-approach and is based straightforwardly on the syllabification rules described in the Portuguese Language Orthographic Agreement \cite{Acordo2009}. Given space limitations, rules were omitted from this paper as they can be found in the website of the project, along with all the resources developed for the dictionary. As for the stress marker, once the syllable structure is known in Brazilian Portuguese, one can predict where stress falls. Stress falls:

\begin{enumerate}
 \item on the antepenultimate syllable if it has an accented vowel $<$\'a,\^a,\'e,\^e,\'i,\'o,\^o,\'u$>$;
 \item on the ultimate syllable if it contains the accented vowels $<$\'a,\'e,\'o$>$ or $<$i,u$>$; or if it ends with one of the following consonants $<$r,x,n,l,z$>$;
 \item on the penultimate syllable otherwise.
\end{enumerate}

The transcriber is based on a hybrid approach, making use of manual transcription rules and an automatic classifier, which builds Decision Trees. Initially, transcription rules are applied to the words. The rules covers not all possible graphemes to phoneme relations, but only those which are predictable by context. The output of the rules is what we called the intermediary transcription form. After obtaining it, a machine learning classifier is applied in order to predict the transcription of the remaining graphemes. Figure 2 gives an example of the transcription process.

\begin{figure}[!ht]
\centerline{ \includegraphics[width=8cm]{./gfx/aeiouado-transcript-ex.pdf}}
\caption{{\it Example of the transcription procedure -- in grey: graphemes yet to be transcribed; in black: graphemes already transcribed.}}
\label{transcExample}
\end{figure}

The rules' phase has two main goals: guarantee the correct transcription of certain predictable graphemes (mostly consonants) and also ensure the alignment between graphemes and phones for the classifier. They were set in order to avoid overlapping and order conflicts. Long sequences of graphemes, such as triphthongs, contextual diphthongs and general diphthongs are transcribed first (e.g.  $<$x-ce$> \rightarrow $\textipa{[-se]}). Then graphemes involving phones that undergo phonological processes are transcribed (e.g. $<$ti$> \rightarrow $\textipa{[tSi]}, $<$di$> \rightarrow $\textipa{[dZi]}). After that, several contextual and general monophones are transcribed (e.g. $<$\#x$> \rightarrow $\textipa{[S]}, $<$\#e-x$> \rightarrow $\textipa{[}\#\textipa{e-z]}). 

On what regards to the classifier, it was developed primarily to deal with the transcription of vowels. In Brazilian Portuguese, vowels have a very irregular behavior, specially the mid ones. Therefore the relations between the vowels' graphemes and their corresponding phonemes are hard to predict beforehand through rules. Consider, for instance, the words ``teto'' \emph{(roof)} and ``gueto'' \emph{(ghetto)}; both are nouns and share basically the same orthographic environment. However the former is pronounced with an open ``e'' \textipa{["tE.tU]} and the latter with a closed one \textipa{["ge.tU]}. The classifier employs Decision Trees, through an optimised version of the CART (Classification and Regression Trees) algorithm and was implemented in Python, by means of the Scikit-learn library \cite{Scikit2011}. 

The algorithm was trained over a corpus of 3,500 words phonetically transcribed and manually revised, with a total of 39,934 instances of phones. The feature extraction happened in the following way. After reviewing the data, we obtained the intermediary transcription form for each of these words and aligned them with the manual transcription. Then, we split the intermediary transcription form into its corresponding phones and, for each phone, we extracted the following information: 

\begin{enumerate}
 \item the phone itself; 
 \item 8 previous phones; 
 \item 8 following phones; 
 \item the distance between the phone and the tonic syllable; 
 \item word class -- parts of speech; 
 \item the manually transcribed phone. 
\end{enumerate}

We considered a window of 8 phones in order deal with vowel harmony phenomena. By establishing a window with such length, one  can assure that pretonic phones will be able to reach the transcription of the vowels in the stressed syllable. The classifier was applied to all 108,389 words categorized as BP words by the Language Identifier module, all of them were cross-checked by two linguists with experience in Phonetics and Phonology.

The Portuguese Wikipedia has about 168,8 million word tokens and 9,7 million types, distributed among 820k articles. After applying the filters to the data, i.e. words with frequency higher than 10, with no digits nor punctuation marks, we ended up with circa 238k word types, representing 151,9 million tokens. Table 1 describes the data.

\begin{table} [t,h]
\caption{\label{wikipedia} {\it Portuguese Wikipedia Summary -- Dumped on 23\textsuperscript{rd} January 2014.}}
\vspace{2mm}
\centerline{
\begin{tabular}{ccc}
\hline \bf & \bf Word Tokens  & \bf Word Types\\\hline
Wikipedia & 168,823,100 & 9,688,039 \\
Selected  & 151,911,350 & 238,012 \\
\textbf{\% Used} &90.0 & 2.4 \\
\hline
\end{tabular}}
\end{table}

The selected words covers 90,0\% of the Wikipedia content. Although the number of selected word types seems too small at first glance, one of the reasons is that 7,901,277 of the discarded words were numbers (81,5\%). The remaining discarded words contained misspellings (\emph{dirijem-se} -- it should be \emph{dirigem-se}), used a non-Roman alphabet ($\lambda$\emph{\'o}$\gamma \omega$), were proper names (\emph{Stolichno}, \emph{Z\'e-pereira}), scientific names (\emph{Aegyptophitecus}), abbreviations or acronyms (LCD, HDMI). 

As for the language identifier, we trained and evaluated it with the 200,000 words multilingual corpus. The corpus consists of 100,000 Brazilian Portuguese words and 20,000 words from each of the following languages: English, French, German, Italian and Spanish. All of these words were collected through web crawling News' sites and were not revised. The results obtained for the identifier, through 5-fold cross validation are described in Table 2. 

\begin{table} [t,h]
\caption{\label{langIdentEval} {\it Results from the Language Identifier module -- Training Phase.}}
\vspace{2mm}
\centerline{
\begin{tabular}{ccccc}
\hline
 & \textbf{Precision} & \textbf{Recall} & \textbf{F1-score} & \textbf{Support} \\ \hline
BP words & 0.85 & 0.89 & 0.87 & 100,000 \\ 
Foreign Words & 0.88 & 0.84 & 0.86 & 100,000 \\
\bf Avg/Total & 0.86 & 0.86 & 0.86 & 200,000 \\ \hline
\end{tabular}}
\end{table}

The language identifier showed an average F1-score of 0.86. Although such result is not as good as we expected -- some authors reported 99\% by using similar methods with trigrams probability, the relatively low F1-score can be explained given the nature of the data. In most language identifiers, the input  consists of texts or several sentences, in other words, there is much more data available for the classifier. Since we are working with single words, the confusion of the model is higher and the results are, consequently, worse. Additionally, because the word list used to train the identifier was not revised, there is noise among the data. After training and evaluating the classifier, we applied it to the selected word list derived from the Wikipedia, in order to detect loanwords. Table 3 describes the results gathered.

\begin{table} [H]
\caption{\label{langIdentWiki} {\it Results from the Language Identifier module -- Wikipedia word list.}}
\vspace{2mm}
\centerline{
\begin{tabular}{cc}
\hline
 & \textbf{Wikipedia word list}\\ \hline
BP words & 108,370 (46\%)\\ 
Foreign Words & 129,642 (54\%)\\
\bf Total & 238,012 \\ \hline
\end{tabular}}
\end{table}

As one can observe, although we established a frequency filter to avoid spurious words, many loanwords still remain. More than half of the word list selected from Wikipedia consists of foreign words. Notwithstanding that, the list of Brazilian Portuguese words is still of considerable size. For instance, the CMUdict \cite{CMUdict1998}, a reference pronunciation dictionary for the English language, has about 125,000 word types.

Concerning the syllabification algorithm and the stress marker, we did not evaluate them in isolation, but together with the transcriber since the rules for each of these modules are intertwined. That is to say the transcription rules are strictly dependent on the stress marker module and the syllable identifier. Besides, the Decision Tree Classifier is built upon the output of the transcription rules, so it is entirely dependent on it. The Decision Tree Classifier was trained over a corpus of 3,500 cross-checked transcribed words, containing 39,934 instances of phones. We analyzed its performance through 5-fold cross validation, the results for each individual phone are summarized in Table 4.

\begin{table} [H]
\caption{\label{transcriberEval} {\it Results from the Transcriber -- Performance per phone.}}
\vspace{2mm}
\centerline{
\begin{tabular}{ccccc}
\hline
 & \bf Precision & \bf Recall & \bf F1-score & \bf Support \\ \hline
\emph{syl. boundary} & 1.00 & 1.00 & 1.00 & 9099 \\ 
\emph{stress} & 1.00 & 1.00 & 1.00 & 3507 \\ 
\textipa{p} & 1.00 & 1.00 & 1.00 & 760 \\ 
\textipa{b} & 1.00 & 1.00 & 1.00 & 357 \\ 
\textipa{t} & 0.99 & 0.99 & 0.99 & 1135 \\ 
\textipa{d} & 0.99 & 0.99 & 0.99 & 1148 \\ 
\textipa{k} & 0.99 & 0.99 & 0.99 & 978 \\ 
\textipa{g} & 1.00 & 1.00 & 1.00 & 298 \\ 
\textipa{tS} & 0.98 & 0.98 & 0.97 & 450 \\ 
\textipa{dZ} & 0.96 & 0.96 & 0.96 & 243 \\ 
\textipa{m} & 1.00 & 1.00 & 1.00 & 668 \\ 
\textipa{n} & 1.00 & 1.00 & 1.00 & 556 \\ 
\textipa{\textltailn} & 1.00 & 1.00 & 1.00 & 69 \\ 
\textipa{f} & 1.00 & 1.00 & 1.00 & 311 \\ 
\textipa{v} & 1.00 & 1.00 & 1.00 & 531 \\ 
\textipa{s} & 0.98 & 0.98 & 0.98 & 2309 \\ 
\textipa{z} & 0.93 & 0.94 & 0.93 & 416 \\ 
\textipa{S} & 0.84 & 0.84 & 0.84 & 138 \\ 
\textipa{k.s} & 0.72 & 0.64 & 0.66 & 41 \\ 
\textipa{Z} & 1.00 & 1.00 & 1.00 & 196 \\ 
\textipa{l} & 1.00 & 1.00 & 1.00 & 682 \\ 
\textipa{L} & 1.00 & 1.00 & 1.00 & 58 \\ 
\textipa{R} & 1.00 & 1.00 & 1.00 & 1388 \\ 
\textipa{h} & 0.98 & 0.99 & 0.99 & 737 \\ 
\textipa{H} & 0.97 & 0.92 & 0.94 & 169 \\ 
\textipa{w} & 0.97 & 0.98 & 0.97 & 441 \\ 
\textipa{\~w} & 0.98 & 0.99 & 0.99 & 309 \\ 
\textipa{j} & 0.97 & 0.95 & 0.96 & 223 \\ 
\textipa{\~j} & 0.95 & 1.00 & 0.98 & 110 \\ 
\textipa{a} & 1.00 & 1.00 & 0.99 & 2316 \\ 
\textipa{@} & 0.99 & 0.99 & 0.99 & 1093 \\ 
\textipa{E} & 0.65 & 0.68 & 0.66 & 275 \\ 
\textipa{e} & 0.93 & 0.91 & 0.92 & 1779 \\ 
\textipa{i} & 0.98 & 0.99 & 0.98 & 2073 \\ 
\textipa{I} & 0.97 & 0.97 & 0.97 & 365 \\ 
\textipa{O} & 0.69 & 0.75 & 0.71 & 220 \\ 
\textipa{o} & 0.93 & 0.92 & 0.93 & 1112 \\ 
\textipa{u} & 0.96 & 0.96 & 0.96 & 488 \\ 
\textipa{U} & 1.00 & 1.00 & 1.00 & 1033 \\ 
\textipa{\~a} & 1.00 & 1.00 & 1.00 & 719 \\ 
\textipa{\~e} & 0.96 & 0.97 & 0.97 & 497 \\ 
\textipa{\~i} & 0.99 & 0.99 & 0.99 & 274 \\ 
\textipa{\~o} & 0.97 & 0.96 & 0.97 & 299 \\ 
\textipa{\~u} & 0.94 & 0.92 & 0.93 & 64 \\ 
Avg/Total & 0.98 & 0.98 & 0.98 & 39934 \\ \hline
\end{tabular}}
\end{table}

As it can be seen, the method achieved very good results, with a F1-score of 0.98. Many segments were transcribed with 100\% accuracy, most of them were consonants. As it was expected, the worst results are related to mid vowels \textipa{[E, e, O, o]}, specially mid-low vowels, \textipa{[E]} showed a F1-score 0.66 and \textipa{[O]} of 0.71. It can be the case that since the grapheme context is the same for \textipa{[E, e]} and \textipa{[O, o]}, the DecisionTree classifier generalizes, in some cases, to the most frequent phone, that is the mid-high vowels \textipa{[e,o]}. The transcriber also had problems with the \textipa{[k.s]} (F1-score: 0.66) and \textipa{[S]} (F1-score: 0.84). This result was also expected, both these phones are related to the grapheme $<$x$>$ which, in Brazilian Portuguese, shows a very irregular behavior. In fact, $<$x$>$ can be pronounced as \textipa{[S, s, z, k.s]}, depending on the  word:  ``bruxa'' \emph{(witch)} \textipa{[S]}, ``pr\'oximo'' \emph{(near)} \textipa{[s]};  ``exame'' \emph{(test)} \textipa{[z]} and ``axila'' \emph{(armpit)} \textipa{[k.s]}.

We presented the method we employed in building a pronunciation dictionary for Brazilian Portuguese. High F1-score values were achieved while transcribing most of the graphemes in Brazilian Portuguese and the dictionary can be considered robust enough for Large Vocabulary Continuous Speech Recognition (LVCSR) and Speech Synthesis. Although the rules we developed are language-specific, the architecture we used for compiling the dictionary, by using transcription rules and machine learning classifiers, can be successfully replicated in other languages. In addition, the entire dictionary, all scripts, algorithms and corpora were made publicly available.\footnote{\url{http://nilc.icmc.usp.br/aeiouado}}\footnote{\url{https://github.com/gustavoauma/aeiouado_g2p}}

\clearpage


\begin{abstract}
This paper describes the method employed to build a machine-readable 
pronunciation 
dictionary for Brazilian Portuguese. The dictionary makes use of a hybrid approach for
converting graphemes into phonemes, based on both manual transcription rules and machine
learning algorithms. It makes use of a word list compiled from the Portuguese
Wikipedia dump. Wikipedia articles were transformed into plain text, tokenized and 
word types were extracted. A language identification tool was developed to detect 
loanwords among data. Words' syllable boundaries and stress were identified.
The transcription task was carried out in a two-step process: i) words are submitted
to a set of transcription rules, in which predictable graphemes (mostly consonants)
are transcribed; ii) a machine learning classifier is used to predict the transcription of the 
remaining graphemes (mostly vowels). The method was evaluated through 5-fold cross-validation; 
results show a F1-score of 0.98. The dictionary and all the resources used to build it 
were made publicly available.

\end{abstract}
\noindent{\bf Index Terms}: pronunciation dictionary, grapheme to phoneme conversion, text to speech


%
\section{Introduction}


In many day-to-day situations, people can now interact with machines and computers through the
most natural human way of communication: speech. Speech Technologies are present in GPS navigation devices, dictation
systems in text editors, voice-guided browsers for the vision-impaired, mobile phones and many other applications \cite{Godwin2009}. 
However, for many languages, there is a dire shortage of resources for building speech technology systems. 
Brazilian Portuguese can be considered one of these languages. Despite being 6\textsuperscript{th} most
spoken language in the world \cite{Ethnologue2013}, with about 200 million speakers, speech recognition and speech 
synthesis for Brazilian Portuguese are far from the current state of the art \cite{Neto2011}. In this paper, we describe the method 
employed in building a publicly available pronunciation dictionary for Brazilian Portuguese which tries to
diminish this scarcity. 

The dictionary makes use of a hybrid approach for grapheme to phoneme conversion, 
based on both manual transcription rules and machine learning algorithms, and aims at promoting the development 
of novel speech technologies for Brazilian Portuguese. Hybrid approaches in grapheme to phoneme conversion have been applied 
successfully to other languages \cite{Damper1998}\cite{Polyakova2006}\cite{Teixeira2006}\cite{Veiga2013}. They have the benefit of taking advantage from both knowledge-based and 
data-driven methods. We propose a method in which the phonetic transcription of a given word is obtained through
a two-step procedure. Its primary word list derives from the Portuguese Wikipedia 
dump of 23\textsuperscript{rd} January 2014. We decided to use Wikipedia as the primary word list 
for the dictionary for many 
reasons: i) given its encyclopedia nature, it covers wide-ranging topics, providing words from both 
general knowledge and 
specialized jargon; ii) it contains around 168,8 million word tokens, being robust enough for the task; iii) it 
makes uses of crowdsourcing, lessening author's bias; iv) its articles are distributed through Creative Commons 
License. Wikipedia articles were transformed into plain text, tokenized and word types were extracted.

We developed a language identifier in order to detect loanwords among data. 
It is a known fact that when languages interact, linguistic exchanges inevitably occur. 
One particular type of linguistic exchange is of great concern while building a pronunciation
dictionary, namely, non-assimilated loanwords \cite{Bussmann96}. Non-assimilated loanwords stand
for lexical borrowings in which the borrowed word is incorporated from one language into another
straightforwardly, without any translation or orthographic adaptation. These words represent a problem to 
grapheme-to-phoneme (G2P) conversion since they show orthographic patterns which are not predicted in advance by rules or
which are too deviant to be captured by machine learning algorithms. Many algorithms
have been proposed to address Language Identification (LID) from text \cite{Bergsma2012}\cite{Bilcu2006}\cite{Dolf2012}\cite{Zampieri2012}. Since our goal is to detect the language of single words, we employed n-gram character 
models in the identifier, given its previous success in dealing with short sequences of characters. 

Brazilian Portuguese Phonology can be regarded as syllable and stress-driven \cite{Cristofaro2005}. In fact,
many phonological processes in Brazilian Portuguese are related to or conditioned by 
syllable structure and stress position \cite{Girelli1990}. Vowel harmony occurs in pretonic context 
\cite{Bisol1989}, posttonic syllables show a limited vowel inventory \cite{Cristofaro2005}, nasalization occurs
when stress syllables are followed by nasal consonants \cite{Quicoli1990}, epenthesis' processes are triggered
by the occurrence of non-allowed consonants in coda position \cite{Delatorre2005} and so on and so forth. Therefore, 
detecting syllable boundaries and stress is of crucial importance for G2P systems, in order to achieve 
correct transcriptions. Several algorithms have been proposed to deal with the syllabification in Brazilian
Portuguese. However most of them were not extensively evaluated nor were made publicly available \cite{Oliveira2005}
\cite{Nhenhem2012} \cite{Neto2011} \cite{Rocha2013}. For this reason, we implemented our own syllabification algorithm, 
based directly on the rules of the last Portuguese Language Orthographic Agreement \cite{Acordo2009}. 

Word types recognized as belonging to Brazilian Portuguese by the language identifier were transcribed in a two-step process:
i) words are submitted to a set of transcription rules, in which predictable graphemes (mostly consonants)
are transcribed; ii) a machine learning classifier is used to predict the transcription of the remaining graphemes (mostly vowels). 
All the data were subsequently revised. Figure 1 summarizes the method.

\begin{figure}[t]
\centerline{ \includegraphics[width=8cm]{./gfx/aeiouado-flowchart-mod.pdf}}
\caption{{\it System architecture for building the pronunciation dictionary.}}
\label{g2p-architecture}
\end{figure}



\section{Method}

\subsection{Primary Word List}

We used the Portuguese Wikipedia's dump of 23\textsuperscript{rd} January 2014 as the primary word
list for the pronunciation dictionary. In order to obtain plain text from the articles, we employed WikiExtractor \cite{Wikiextractor2013};
it strips all the  MediaWiki markups and metadata forms. Afterwards, texts were tokenized and unique words types extracted. 
The Portuguese Wikipedia has about 168,8 million word tokens and 9,7 million types, distributed among  820,000 articles.
With the purpose of avoiding misspellings, URLs and other spurious data, only words with frequency higher than 10, 
which showed neither digits nor punctuation marks were selected. 

\subsection{Language Identifier}


A Language Identifier module was developed in order to detect loanwords in the pronunciation dictionary.
The Identifier consists of a Linear Support Vector Machine Classifier \cite{Steinwart2008} and was implemented in Python, 
through Scikit-learn \cite{Scikit2011}. It was trained on a corpus made of the 200,000, containing 100,000 Brazilian Portuguese
words and 20,000 words of each of the following languages: English, French, German, Italian and Spanish. All of these words were 
collected through web crawling News' sites and were not revised.
We selected these languages because they are the major donors of loanwords to Brazilian
Portuguese \cite{Alves2001}. From these words we extracted features such as initial and final bi- and trigraphs; 
number of accented graphs, vowel-consonant ratio; average mono-, bi- and trigraphs probability; and used 
them to estimate the classifier. Further details can be found in the website of the Project\footnote{http://nilc.icmc.usp.br/listener/aeiouado}. After training,
we applied the classifier to the Wikipedia word list with the purpose of identifying
loanwords among data. The identified loanwords were then separated from the rest of words for later 
revision, i.e. they were not submitted to automatic transcription.

\subsection{Syllabification algorithm and stress marker}

Our syllabification algorithm follows a rule-approach and is based straightforwardly on the syllabification
rules described in the Portuguese Language Orthographic Agreement \cite{Acordo2009}. Given space limitations,
rules were omitted from this paper as they can be found in the website of the project, along with all the
resources developed for the dictionary. As for the stress marker, once the syllable structure is known
in Brazilian Portuguese, one can predict where stress falls. Stress falls:

\begin{enumerate}
 \item on the antepenultimate syllable if it has an accented vowel $<$\'a,\^a,\'e,\^e,\'i,\'o,\^o,\'u$>$;
 \item on the ultimate syllable if it contains the accented vowels $<$\'a,\'e,\'o$>$ or $<$i,u$>$; or if it ends with one of the following consonants $<$r,x,n,l,z$>$;
 \item on the penultimate syllable otherwise.
\end{enumerate}


\subsection{Transcriber}
The transcriber is based on a hybrid approach, making use of manual transcription rules and an automatic classifier, 
which builds Decision Trees. Initially, transcription rules are applied to the words. 
The rules covers not all possible graphemes
to phoneme relations, but only those which are predictable by context. The output of the rules is what we called the 
intermediary transcription form. After obtaining it, a machine learning classifier is applied in order to
predict the transcription of the remaining graphemes. Figure 2 gives an example of the transcription process.

\begin{figure}[!ht]
\centerline{ \includegraphics[width=8cm]{./gfx/aeiouado-transcript-ex.pdf}}
\caption{{\it Example of the transcription procedure -- in grey: graphemes yet to be transcribed; in black: graphemes already transcribed.}}
\label{transcExample}
\end{figure}

The rules' phase has two main goals: guarantee the correct transcription of certain predictable graphemes (mostly consonants)
and also ensure the alignment between graphemes and phones for the classifier. They were set in order to avoid
overlapping and order conflicts. Long sequences of graphemes, such as triphthongs, contextual diphthongs and general 
diphthongs are transcribed first (e.g.  $<$x-ce$> \rightarrow $\textipa{[-se]}). 
Then graphemes involving phones that undergo phonological processes are transcribed (e.g. $<$ti$> \rightarrow $\textipa{[tSi]},
$<$di$> \rightarrow $\textipa{[dZi]}). After that, several contextual and general monophones are transcribed 
(e.g. $<$\#x$> \rightarrow $\textipa{[S]}, $<$\#e-x$> \rightarrow $\textipa{[}\#\textipa{e-z]}). 

On what regards to the classifier, it was developed primarily to deal with the transcription of vowels. In Brazilian
Portuguese, vowels have a very irregular behavior, specially the mid ones. Therefore the relations between the vowels' 
graphemes and their corresponding phonemes are hard to predict beforehand through rules. Consider, for instance, the words 
``teto'' \emph{(roof)} and ``gueto'' \emph{(ghetto)}; both are nouns and share basically the same orthographic environment.
However the former is pronounced with an open ``e'' \textipa{["tE.tU]} and the latter with a closed one \textipa{["ge.tU]}.
The classifier employs Decision Trees, through an optimised version of the CART (Classification and Regression Trees)
algorithm and was implemented in Python, by means of the Scikit-learn library \cite{Scikit2011}. 

The algorithm was trained over a corpus of 3,500 words phonetically transcribed and manually revised, with a total
of 39,934 instances of phones. The feature extraction happened in the following way. After reviewing the data, we obtained the intermediary 
transcription form for each of these words and aligned them with the manual transcription. Then, we split
the intermediary transcription form into its corresponding phones and, for each phone, we extracted the following 
information: i) the phone itself; ii) 8 previous phones; iii) 8 following phones; iv) the distance between 
the phone and the tonic syllable; v) word class -- parts of speech; v) the manually transcribed phone. We considered
a window of 8 phones in order deal with vowel harmony phenomena. By establishing a window with such length, one 
can assure that pretonic phones will be able to reach the transcription of the vowels in the stressed syllable.
The classifier was applied to all 108,389 words categorized as BP words by the Language Identifier module,
all of them were cross-checked by two linguists with experience in Phonetics and Phonology.

\section{Results}

The Portuguese Wikipedia has about 168,8 million word tokens and 9,7 million types, distributed among 820k articles.
After applying the filters to the data, i.e. words with frequency higher than 10, with no digits nor punctuation marks,
we ended up with circa 238k word types, representing 151,9 million tokens. Table 1 describes the data.

\begin{table} [t,h]
\caption{\label{wikipedia} {\it Portuguese Wikipedia Summary -- Dumped on 23\textsuperscript{rd} January 2014.}}
\vspace{2mm}
\centerline{
\begin{tabular}{|ccc|}
\hline \bf & \bf Word Tokens  & \bf Word Types\\\hline
Wikipedia & 168,823,100 & 9,688,039 \\
Selected  & 151,911,350 & 238,012 \\
\textbf{\% Used} &90.0 & 2.4 \\
\hline
\end{tabular}}
\end{table}

The selected words covers 90,0\% of the Wikipedia content. Although the number of selected word types seems too small at first glance,
one of the reasons is that 
7,901,277 of the discarded words were numbers (81,5\%). The remaining discarded words contained misspellings (\emph{dirijem-se} --
it should be \emph{dirigem-se}), used a non-Roman alphabet ($\lambda$\emph{\'o}$\gamma \omega$), were proper names (\emph{Stolichno}, \emph{Z\'e-pereira}), 
scientific names (\emph{Aegyptophitecus}), abbreviations or acronyms (LCD, HDMI). 

As for the language identifier, we trained and evaluated it with the 200,000 words multilingual corpus. The corpus consists of
100,000 Brazilian Portuguese words and 20,000 words from each of the following languages: English, French, German, Italian and Spanish. 
All of these words were collected through web crawling News' sites and were not revised.
The results obtained for the identifier, through 5-fold cross validation are described in Table 2. 

\begin{table} [t,h]
\caption{\label{langIdentEval} {\it Results from the Language Identifier module -- Training Phase.}}
\vspace{2mm}
\centerline{
\begin{tabular}{|ccccc|}
\hline
 & \textbf{Precision} & \textbf{Recall} & \textbf{F1-score} & \textbf{Support} \\ \hline
BP words & 0.85 & 0.89 & 0.87 & 100,000 \\ 
Foreign Words & 0.88 & 0.84 & 0.86 & 100,000 \\
\bf Avg/Total & 0.86 & 0.86 & 0.86 & 200,000 \\ \hline
\end{tabular}}
\end{table}

The classifier showed an average F1-score of 0.86. Although such result is not as good as we expected -- some authors
reported 99\% by using similar methods with trigrams probability, the relatively low F1-score can be explained given
the nature of the data. In most language identifiers, the input  consists of texts or several sentences, in other 
words, there is much more data available for the classifier. Since we are working with single words, the confusion
of the model is higher and the results are, consequently, worse. Additionally, because the word list used to train the identifier
was not revised, there is noise among the data. After training and evaluating the classifier,
we applied it to the selected word list derived from the Wikipedia, in order to detect loanwords. Table 3 
describes the results gathered.

\begin{table} [t,h]
\caption{\label{langIdentWiki} {\it Results from the Language Identifier module -- Wikipedia word list.}}
\vspace{2mm}
\centerline{
\begin{tabular}{|cc|}
\hline
 & \textbf{Wikipedia word list}\\ \hline
BP words & 108,370 (46\%)\\ 
Foreign Words & 129,642 (54\%)\\
\bf Total & 238,012 \\ \hline
\end{tabular}}
\end{table}

As one can observe, although we established a frequency filter to avoid spurious words, many loanwords
still remain. More than half of the word list selected from Wikipedia consists of foreign words. Notwithstanding 
that, the list of Brazilian Portuguese words is still of considerable size. For instance, the CMUdict \cite{CMUdict1998}, a reference
pronunciation dictionary for the English language, has about 125,000 word types.

Concerning the syllabification algorithm and the stress marker, we did not evaluate them in isolation, but together with 
the transcriber since the rules for each of these modules are intertwined.
That is to say the transcription rules are strictly dependent on the stress marker module and the syllable identifier. Besides,
the Decision Tree Classifier is built upon the output of the transcription rules, so it is entirely dependent on it.
The Decision Tree Classifier was trained over a corpus of 3,500 cross-checked transcribed words, containing 39,934 instances of phones.
We analyzed its performance through 5-fold cross validation, the results for each individual phone are summarized in Table 4.

\begin{table} [t,!h]
\caption{\label{transcriberEval} {\it Results from the Transcriber -- Training Phase.}}
\vspace{2mm}
\centerline{
\begin{tabular}{|ccccc|}
\hline
 & \bf Precision & \bf Recall & \bf F1-score & \bf Support \\ \hline
\emph{syl. boundary} & 1.00 & 1.00 & 1.00 & 9099 \\ 
\emph{stress} & 1.00 & 1.00 & 1.00 & 3507 \\ 
\textipa{p} & 1.00 & 1.00 & 1.00 & 760 \\ 
\textipa{b} & 1.00 & 1.00 & 1.00 & 357 \\ 
\textipa{t} & 0.99 & 0.99 & 0.99 & 1135 \\ 
\textipa{d} & 0.99 & 0.99 & 0.99 & 1148 \\ 
\textipa{k} & 0.99 & 0.99 & 0.99 & 978 \\ 
\textipa{g} & 1.00 & 1.00 & 1.00 & 298 \\ 
\textipa{tS} & 0.98 & 0.98 & 0.97 & 450 \\ 
\textipa{dZ} & 0.96 & 0.96 & 0.96 & 243 \\ 
\textipa{m} & 1.00 & 1.00 & 1.00 & 668 \\ 
\textipa{n} & 1.00 & 1.00 & 1.00 & 556 \\ 
\textipa{\textltailn} & 1.00 & 1.00 & 1.00 & 69 \\ 
\textipa{f} & 1.00 & 1.00 & 1.00 & 311 \\ 
\textipa{v} & 1.00 & 1.00 & 1.00 & 531 \\ 
\textipa{s} & 0.98 & 0.98 & 0.98 & 2309 \\ 
\textipa{z} & 0.93 & 0.94 & 0.93 & 416 \\ 
\textipa{S} & 0.84 & 0.84 & 0.84 & 138 \\ 
\textipa{k.s} & 0.72 & 0.64 & 0.66 & 41 \\ 
\textipa{Z} & 1.00 & 1.00 & 1.00 & 196 \\ 
\textipa{l} & 1.00 & 1.00 & 1.00 & 682 \\ 
\textipa{L} & 1.00 & 1.00 & 1.00 & 58 \\ 
\textipa{R} & 1.00 & 1.00 & 1.00 & 1388 \\ 
\textipa{h} & 0.98 & 0.99 & 0.99 & 737 \\ 
\textipa{H} & 0.97 & 0.92 & 0.94 & 169 \\ 
\textipa{w} & 0.97 & 0.98 & 0.97 & 441 \\ 
\textipa{\~w} & 0.98 & 0.99 & 0.99 & 309 \\ 
\textipa{j} & 0.97 & 0.95 & 0.96 & 223 \\ 
\textipa{\~j} & 0.95 & 1.00 & 0.98 & 110 \\ 
\textipa{a} & 1.00 & 1.00 & 0.99 & 2316 \\ 
\textipa{@} & 0.99 & 0.99 & 0.99 & 1093 \\ 
\textipa{E} & 0.65 & 0.68 & 0.66 & 275 \\ 
\textipa{e} & 0.93 & 0.91 & 0.92 & 1779 \\ 
\textipa{i} & 0.98 & 0.99 & 0.98 & 2073 \\ 
\textipa{I} & 0.97 & 0.97 & 0.97 & 365 \\ 
\textipa{O} & 0.69 & 0.75 & 0.71 & 220 \\ 
\textipa{o} & 0.93 & 0.92 & 0.93 & 1112 \\ 
\textipa{u} & 0.96 & 0.96 & 0.96 & 488 \\ 
\textipa{U} & 1.00 & 1.00 & 1.00 & 1033 \\ 
\textipa{\~a} & 1.00 & 1.00 & 1.00 & 719 \\ 
\textipa{\~e} & 0.96 & 0.97 & 0.97 & 497 \\ 
\textipa{\~i} & 0.99 & 0.99 & 0.99 & 274 \\ 
\textipa{\~o} & 0.97 & 0.96 & 0.97 & 299 \\ 
\textipa{\~u} & 0.94 & 0.92 & 0.93 & 64 \\ 
Avg/Total & 0.98 & 0.98 & 0.98 & 39934 \\ \hline
\end{tabular}}
\end{table}

As it can be seen, the method achieved very good results, with a F1-score of 0.98. Many segments were transcribed
with 100\% accuracy, most of them were consonants. As it was expected, the worst results are related to mid vowels 
\textipa{[E, e, O, o]}, specially mid-low vowels, \textipa{[E]} showed a F1-score 0.66 and \textipa{[O]} of 0.71. It 
can be the case that since the grapheme context is the same for \textipa{[E, e]} and \textipa{[O, o]}, the Decision
Tree classifier generalizes, in some cases, to the most frequent phone, that is the mid-high vowels \textipa{[e,o]}.
The transcriber also had problems with the \textipa{[k.s]} (F1-score: 0.66) and \textipa{[S]} (F1-score: 0.84). This 
result was also expected, both these phones are related to the grapheme $<$x$>$ which, in Brazilian Portuguese,
shows a very irregular behavior. In fact, $<$x$>$ can be pronounced as \textipa{[S, s, z, k.s]}, depending on the 
word:  ``bruxa'' \emph{(witch)} \textipa{[S]}, ``pr\'oximo'' \emph{(near)} \textipa{[s]};  ``exame'' \emph{(test)}
\textipa{[z]} and ``axila'' \emph{(armpit)} \textipa{[k.s]}.

\section{Final Remarks}

We presented the method we employed in building a pronunciation dictionary for Brazilian Portuguese. High F1-score 
values were achieved while transcribing most of the graphemes in Brazilian Portuguese and the dictionary can be
considered robust enough for Large Vocabulary Continuous Speech Recognition (LVCSR) and Speech Synthesis. Although 
the rules we developed are language-specific, the architecture we used for compiling the dictionary, by using transcription
rules and machine learning classifiers, can be successfully replicated in other languages. In addition, the entire dictionary,
all scripts, algorithms and corpora were made publicly available.

\section{Acknowledgements}
Part of the results presented in this paper were obtained through research activity in the project titled 
``Semantic Processing of Brazilian Portuguese Texts'', sponsored by \emph{Samsung Eletr\^onica da Amaz\^onia Ltda.} under 
the terms of Brazilian federal law number 8.248/91.




\paragraph{Further developments}

Since the publishing date of our paper (\citeauthor{Mendonca2014}~\cite{Mendonca2014}), Aeiouad\^o \ac{G2P} has been improved. Recently, we increased the training database to XXX words (XXX phone tokens). Moreover we are now using extra morphological information in order to determine the grapheme's transcription, mostly to solve the problems with the mid vowels \textipa{[E, e, O, o]}. 

Previously we were using only the parts of speech as the source of morphological information. We assumed that by just providing the words' parts of speech, the Decision Tree Classifier would be able to learn and differ pairs of heterophonic homographs, such as ``jogo'' \textsc{noun} \emph{(game)} and ``jogo'' \textsc{verb} \emph{(I play)}, or ``governo'' \textsc{noun} \emph{(government)} and ``governo'' \textsc{verb} \emph{(I rule)}. However, given the poor performance of the conversor in discerning between \textipa{[E]} vs. \textipa{[e]}, and \textipa{[O]} vs. \textipa{[o]}; we decided to refine the morphological features. 

We adapted the training database to the Unitex-PB dictionary \cite{Muniz2004}, which follows  formalismo DELA (Dictionnarie Electronique du LADL)

There is a huge lexicon with,

This type of alternation is very productive in \ac{BP}.  reported to have collected 1,812 pairs of heterophonic homographs in dictionaries and , although only 226 occurred in corpus).

\citeauthor{}, Pandu , although in a corpus analysis, only 226 \cite{Shulby2013}.

However this is not the case. Previously, only the word class information was used to determine. We thought that would be enough to 


%*****************************************
%*****************************************
%*****************************************
%*****************************************
%*****************************************
\cleardoublepage

\invisiblesection{Evaluating phonetic spellers for user-generated content in Brazilian Portuguese}
\includepdf[pages={1-},scale=1.0, pagecommand={}]{Chapters/PaperSpeller.pdf}\label{sec:speller}
%%*****************************************
\chapter{Phonetic-based Speller}\label{ch:speller}
%*****************************************

\section*{Abstract}
\footnote{This chapter contains an extended version of the recently submitted paper: \bibentry{Mendonca2015}}
Recently, spell checking (or spelling correction systems) has regained attention 
due to the need of normalizing user-generated content (UGC) on the web. UGC presents 
new challenges to spellers, as its register is much more informal and contains much
more variability than traditional spelling correction systems can handle. This chapter proposes 
two new approaches to deal with spelling correction of UGC in Brazilian Portuguese (BP), 
both of which take into account phonetic errors. The first approach is based on three phonetic
modules running in a pipeline. The second one is based on machine learning, with soft decision 
making, and considers context-sensitive misspellings. We compared our methods with 
others on a human annotated UGC corpus of reviews of products. The machine learning approach surpassed 
all other methods, with 78.0\% correction rate, very low false positive (0.7\%) and false 
negative rate (21.9\%). 

\section{Introduction}

Spell checking is a very well-known and studied task of natural language processing (NLP), being present in applications used by the general public, including word processors and search engines. Most of the methods of spell checking are based on large dictionaries to detect non-words, mainly related to typographic errors caused by key adjacency or fast key stroking. 
%Such kind of error was the basis of the first spell checkers, dating back to Damerau \cite{Damerau1964}, which addressed the problem by analyzing the edit distance of the words. 
Currently, with the recent boom of mobile devices, with small touchscreens and tiny keyboards, one can miss the keystrokes, hitting adjacent keys on the keyboard, thus spell checking has regained attention \cite{Duan2011}.   

Dictionary-based approaches can be ineffective when the task is to detect and correct spelling mistakes which coincidentally correspond to an existing word (real-word errors). Different from non-word errors, real-word errors are context dependent. Several approaches have been proposed to deal with these errors: mixed trigram models \cite{Fossati2007}, confusion sets \cite{Fossati2008}, improvements on the trigram-based noisy-channel model \cite{Mays1991} and \cite{WilcoxOHearn2008}, use of GoogleWeb 1T 3-gram data set and a normalized and modified version of the Longest Common Subsequence string matching algorithm \cite{Islam2009}, a graph-based method using contextual and PoS features and the double metaphone algorithm to represent phonetic similarity \cite{Sonmez2014}. As an example, although MS Word (from 2007 version to on) claims to include a contextual spelling checker, an independent evaluation of it found high precision but low recall in a sample of 1400 errors \cite{Hirst2008}.

Errors due to phonetic similarity also impose difficulties to spell checkers. They occur when a writer knows well the pronunciation of a word but does not know how to spell it. This kind of error requires new approaches to combine phonetic models and models for correcting typographic and/or real-word errors. In \cite{Zampieri2014}, for example, the authors use a linear combination of two measures -- the Levenshtein distance between
two strings and the Levenshtein distance between the Soundex \cite{soundex} code of two strings.
Phonetic errors usually occur due to inconsistent spellings rules, ambiguous word breaking and fast introduction of new words, mainly related to technology jargon, affecting both native and non-native speakers of a language \cite{Duan2011}. 

\cite{Baptista2011} detail these problems in a typology of spelling errors in the scenery of written language acquisition: errors produced by lack of understanding of  letter-to-sound correspondences in written language, by fails in the transcription of the oral language, by breaking rules based on phonology or morphology and by inconsistent spelling rules. Due to these several kinds of error this task is still difficult.

Some applications require interactive spelling correction (e.g. typing a text or a search query),  whereas others require fully automatic correction, (e.g. corpus normalization). While in the first kind of applications the spell checker presents several suggestions to the user, the second one requires a spell checker that elects the better suggestion (first hit accuracy), as there is no user to take the decision.

In the last decade, some researchers have revisited spell checking issues motivated by web applications, such as search query engines and sentiment analysis tools based on natural language processing (NLP) of UGC, e.g. Twitter data or product reviews.
The search engine Google has turned popular the facility of auto-completing, where suggestions to prefixes of a query are offered; this is still a process of interactive spelling correction. There is also another kind of facility where a unique suggestion is given after the query is typed, with the link of the documents recovered for it \cite{Duan2011}, \cite{Cucerzan2004}. This facility demands a high precision automatic spelling correction. 

Normalization of UGC has received great attention also because the performance of NLP tools (e.g. taggers, parsers and named entity recognizers) is greatly decreased when applied to UGC. Besides misspelled words, this kind of text presents a long list of problems, such as acronyms and proper names with inconsistent capitalization, abbreviations introduced by chat-speak style, slang terms mimicking the spoken language, loanwords from English as technical jargon, as well as problems related to ungrammatical language and lack of punctuation \cite{Duran2014,DeClercq2013,Han2013,Andrade2012}.
UGC normalization also requires automatic spelling correction, i.e., there is a need to automatically select the word that will more likely correct the misspelled word, not relying on a list of candidates for human selection. 

In \cite{Andrade2012} the authors propose a spell checker for Brazilian Portuguese (BP) to work on the top of Web text collectors. They have tested their method on news portals and on informal texts collected from Twitter in BP. However, they do not inform the error correction rate of the system. Furthermore, while their focus is on the response time of the application, they do not address real-word errors.

This chapter presents two new spell checking methods for UGC in BP. The first of them deals with phonetically motivated errors, a recurrent problem in UGC not addressed by traditional spell checkers. The second one deals additionally with real-word errors.
We present a comparison of these methods with a baseline system and JaSpell over a new and large benchmark corpus for this task. The corpus contains product reviews with 38,128 tokens and 4,083 annotated errors. Such corpus is also a contribution of our study\footnote{The small benchmark of 120 tokens used in \cite{Martins2004} and \cite{Ahmed2009} 
is not representative of our scenario.}. 

This chapter is structured as follows. In Section~\ref{sec:speller-method} we describe our methods, the setup of the experiments and the corpus we compiled. In Section~\ref{sec:speller-results} we present the results. In Section~\ref{sec:speller-related} we discuss related work on spelling correction of phonetic and real-word errors. To conclude, the final remarks are outlined in Section~\ref{sec:speller-final-remarks}.

\section{Experimental Settings and Methods}\label{sec:speller-method}

In this Section we present the four methods compared in our evaluation. Two of them are used by existing spellers, one is taken as baseline and the other is taken as benchmark. The remaining two are novel methods developed within the project reported herein. After describing in detail the novel methods, we present the corpus specifically developed to evaluate BP spellers, as well as the evaluation metrics.
The first one is the open source spell checker JaSpell for Portuguese (Baseline method). The second is acombination of phonetic rules and Soundex applied to candidates generated by 1 and 2 edit distance (Benchmark method). The third (nome1) and the fourth (nome2) are the two novel methods presented by this paper. The (nome1) is an improvement of the Benchmark method, a pipeline of three phonetic modules applied to candidates generated by 1 and 2 edit distance and by phonetic similarity - a manually built set of phonetic-based rules, a grapheme-to-phoneme converter, and the Soundex method adapted to BP (called here as Grapheme-to-Phoneme method). The (nome2) is a context-sensitive speller, based on machine learning, applied to candidates generated by 1 and 2 edit distance, by phonetic similarity and by word combinations of diacritics (called here as Machine Learning).
 
If none of these phonetic modules succeed, the second layer chooses the best candidate according to its edit-distance (the lower, the better) and to its frequency in large BP corpus (the bigger, the better); 
In both methods (Grapheme-to-Phoneme and Rules\&Soundex), if none of the phonetic modules succeed, a second layer chooses the best candidate according to its edit-distance (the lower, the better) and to its frequency in large BP corpus (the bigger, the better).

The aim of our experiments is to evaluate the effectiveness in the correction of common misspellings in UGC in BP of different phonetic-based methods applied on a corpus of product reviews. In this research we do not focus on errors related to acronyms, proper names, abbreviations, internet slang, technical jargon or loanwords in English. Instead, our goal is to assess the methods with regard to  phonetic-motivated errors and a special group of real-word errors in UGC due to the absence of diacritics.

\subsection{Method I - Baseline}

We use as a baseline the open source Java Spelling Checking Package, JaSpell\footnote {http://jaspell.sourceforge.net/}. JaSpell can be  considered a strong baseline and it is  employed at the tumba! Portuguese Web search engine to support interactive spelling checking of user queries. 
JaSpell classifies the candidates for a misspelled word according to the word frequency in a large corpus together with other heuristics, such as keyboard proximity or phonetic keys, provided by the Double Metaphone algorithm \cite{2000double} for the English language. At the time this speller was developed there was no version of these rules for the Portuguese language\footnote{Currently, a BP version of the phonetic rules can be found at http://sourceforge.net/projects/metaphoneptbr/}. 

\subsection{Method II - Benchmark}

The method presented in \cite{Avanco2014} is taken as benchmark. It combines phonetic knowledge in the form of a set of rules and the algorithm Soundex. It was inspired by the analysis of errors of the same corpus of products' reviews \cite{Hartmann2014} that inspired our proposals. Furthermore, as such method aims to be used for normalizing web texts, it performs automatic spelling correction. 

To increase the accuracy of the first hit, this method relies in some ranking heuristics. The strategies developed by the authors consider the phonetic proximity between the input wrong word and the candidates to substitute it. If the typed word does not belong to the lexicon, a set of candidates is generated by applying one and two edit distances from the original word and the words in the lexicon. Then a set of phonetic rules for Brazilian Portuguese codifies letters and digraphs which have similar sounds in a specific code. If necessary, the next step performs the algorithm Soundex, slightly modified for BP. Finally, if none of these phonetic-based algorithms is able to suggest a correction, the candidate with the highest frequency in a reference corpus among the ones with the least edition-distance is suggested. The lexicon used is the Unitex-PB\footnote{http://www.nilc.icmc.usp.br/nilc/projects/unitex-pb/web/} and the frequency list was taken from Corpus Brasileiro\footnote{http://corpusbrasileiro.pucsp.br/cb/}. The pseudocode for the algorithm can be found in Algorithm~\ref{alg:pseudocode-method2}.

\begin{algorithm}
\scriptsize
\caption{Method II -- Benchmark}\label{alg:pseudocode-method2}
\begin{algorithmic}[1]
\Procedure{CorrectWord}{w}
\If {w in Unitex}
\State \Return w
\Else
\State $w\_trans \gets pt\_rules(w)$ \Comment{apply PT-Rules to word}
\State $w\_soundex \gets soundex(w)$ \Comment{get word Soundex code}
\State $sugs \gets lev(Unitex,2)$ \Comment{get words with edit dist. 1 or 2}
\State \emph{\textbf{Look for rule transcription match}}:
\For{sug in sugs}
\State $sug\_trans \gets pt\_rules(sug)$
\If {$w\_trans = sug\_trans$} 
\State \Return sug
\EndIf
\EndFor
\State \emph{\textbf{Look for Soundex code match}}:
\For{sug in sugs}
\State $sug\_soundex \gets soundex(sug)$
\If {$w\_soundex = sug\_soundex$} 
\State \Return sug
\EndIf
\EndFor    
\State \Return{most frequent suggestion} 
\EndIf
\EndProcedure
\end{algorithmic}
\end{algorithm}


\subsection{Method III - Grapheme-to-Phoneme based Method (GPM)}

By testing the benchmark method, we noticed that many of the wrong corrections were related to a gap between the application of phonetic rules and 
the Soundex module. The letter-to-sound rules were developed specially
for the spelling correction, therefore, they are very accurate for the task but have a low recall, since many words do not possess the misspelling patterns 
which they try to model. In contrast, the transcriptions generated by the adapted
Soundex algorithm are too broad and many phonetically different words are given the same code. For instance, the words "perto" (\emph{near}) and "forte" (\emph{strong}) are both
transcribed with the Soundex code ''1630'', in spite of being very distinct phonetically: 
"perto" corresponds to \textipa{['pEh.tU]}, and "forte" to \textipa{['fOh.tSI]}.

To fill this gap we propose the use of a general-purpose grapheme-to-phoneme converter to be executed prior to the Soundex module. We selected Aeiouado's grapheme-to-phoneme converter \cite{Mendonca2014} 
for this purpose, since it consists of the state of the art in grapheme-to-phoneme transcription for Brazilian Portuguese. 
Aeiouado employs a hybrid approach for converting graphemes into phonemes, based on both manual transcription rules and machine learning algorithms. The transcription task is carried out in two stages: i) words are submitted to a set of transcription rules, in which predictable graphemes (mostly consonants) are transcribed; ii) a decision tree classifier is used to predict the transcription of the remaining graphemes (mostly vowels). The method achieved an average F1-score of 0.98 regarding to phone transcription. 
Since the converter was developed primarily for text-to-speech and automatic speech recognition, its transcriptions are too much detailed for spelling correction purposes. Only few words share exactly the same phone sequence. Therefore, we had to broaden the transcription, by grouping the mid-high and mid-low vowels together, by deleting the difference between nasal and oral vowels, by considering the many rhotic sounds into a single one, etc.

The usage of the grapheme-to-phoneme converter is a bit different from a simple pipeline. According to Toutanova \cite{Toutanova2002}, phonetic-based errors usually need larger edit distances to be detected. For instance, the word "durex" (\emph{sellotape}) and one of its misspelled forms "dur\'equis" have an edit distance of 5 units, despite having very similar or equal phonetic forms: \textipa{[du'rEks]} $\sim$ \textipa{[du'rEkIs]}. Therefore, instead of simply increasing the edit distance, which would imply in having a larger number of candidates to filter, we decided to do the reverse process. We transcribed the Unitex-PB dictionary and stored it into a database, with the transcriptions as keys. Thus, in order to obtain words which are phonetic similar words, we transcribe the input word and look it up in the database. Considering the "dur\'equis" example, we would first transcribe it as \textipa{[du'rE.kIs]}, and then check if there are any words in the database with this transcription. In this case, it would return "durex", the expected form.

The only difference of GPM in comparison with Method II lies in the G2P transcription match, which takes place prior to Soundex. The pseudocode for the algorithm can be found in Algorithm~ \ref{alg:pseudocode-method3}. 

\begin{algorithm}
\scriptsize
\caption{Method III -- GPM}\label{alg:pseudocode-method3}
\begin{algorithmic}[1]
\Procedure{CorrectWord}{w}
\If {w in Unitex}
\State \Return w
\Else
\State $w\_trans \gets pt\_rules(w)$ \Comment{apply PT-Rules to word}
\State $w\_phono \gets g2p(w)$ \Comment{apply G2P converter to word}
\State $w\_soundex \gets soundex(w)$ \Comment{get word soundex code}
\State $sugs \gets lev(Unitex,2)$ \Comment{get words with edit dist. 1 or 2}
\State \emph{\textbf{Look for rule transcription match}}:
\For{sug in sugs}
\State $sug\_trans \gets pt\_rules(sug)$
\If {$w\_trans = sug\_trans$} 
\State \Return sug
\EndIf
\EndFor
\State \emph{\textbf{Look for G2P transcription match in the database}}:
\If {any word that match $w\_phono$} 
\State \Return word
\EndIf
\State \emph{\textbf{Look for soundex code match}}:
\For{sug in sugs}
\State $sug\_soundex \gets soundex(sug)$
\If {$w\_soundex = sug\_soundex$} 
\State \Return sug
\EndIf
\EndFor    
\State \Return{most frequent suggestion} 
\EndIf
\EndProcedure
\end{algorithmic}
\end{algorithm}

In spite of being better than the baseline because they tackle phonetic-motivated errors, Method II and GPM have a limitation: they do not correct real word errors. The following method is intended to overcome this shortcoming by using context information.

\subsection{Method IV -- GPM in a Machine Learning framework (GPM-ML)}

Method IV has the advantage of bringing together many approaches to spelling correction into a machine learning framework. The architecture of the method is described in Figure~\ref{fig:sys3-architecture}. 

\begin{figure}[h!]
  \centering
    \includegraphics[width=0.7\textwidth]{gfx/speller_architecture.pdf}
\caption{\label{fig:sys3-architecture} \it Architecture of the GPM-ML}
\end{figure}
The method is based on three main steps: (i) candidate word generation, (ii) feature extraction and (iii) candidate selection. 
The word generation phase encompasses three modules which produce a large number of suggestions, considering the following aspects: orthographic, phonetic and diacritic similarities.
For producing suggestions which are typographically similar, the Levenshtein distance is used. For each input word, we select all words in a dictionary which diverge from the input by at most 2 units. For instance, suppose the user intended to write "mesa" (\emph{table}),  
but missed a keystroke and typed "meda" instead. The Levenshtein module would generate a number of suggestions including an edit distance of 1 or 2, such as "medo" (\emph{fear}), "meta" (\emph{goal}), "moda" (\emph{fashion}), "nada" (\emph{nothing}), "mexe" (\emph{he/she moves}) etc. For computational efficiency, we stored the dictionary in a trie structure, in order to make it quickly searchable. A revised version of the Unitex-PB was employed as our reference dictionary (\emph{circa} 550,000 words)\footnote{The dictionary is available upon request.}.

As for phonetic similarity, the Aeiouado's grapheme-to-phoneme converter \cite{Mendonca2014} was used to group phonetically related words. We transcribed the Unitex-PB word list  phonetically and stored all word transcriptions along with their orthographic form into a database, exactly as we did for GPM. Thus for generating suggestions which are phonetically similar to the word typed by the user, we obtain its phonetic transcription and look it up in the database.

The diacritic module is responsible for generating words which are similar to the word typed by the user with respect to diacritic symbols. This module was proposed since we observed that most of the misspellings in the corpus were caused by a lack or misuse of diacritics. BP has
five types of diacritics: accute (\'{}), cedilla (\c{c}), circumflex (\^{}), grave (\`{}) and tilde (\~{}). The diacritics often indicate different vowel quality, timbre or stress. However, these symbols are rarely used in UGC, and the reader uses the the context to  disambiguate the intended word. In order to allow the speller to deal with this problem, the diacritic model generates, given a word input, all possible word combinations of diacritics. Once more, the Unitex-PB is used as reference. 

After word generation, the feature extraction phase takes place. This phase is responsible  for extracting relevant information from the list of words generated in the previous step. The aim is to allow the classifier to compare these words with the one typed by the user, in such a way that the classifier is able to choose to keep the typed word or to replace it with one of the generated suggestions. 

As misspelling errors may be of different nature (such as typographical, phonological or related to diacritics), we try to select features that encompass all these phenomena. For each word suggestion produced in the word generation phase, we extract 14 features:

\begin{enumerate}
\setlength{\itemsep}{1pt}
\item \textsc{typedOrGen}: whether the word was typed by the user or was produced in the word generation phase;
\item \textsc{isTypo}: 1 if the word was generated by the typographical module; 0 otherwise;
\item \textsc{isPhone}: 1 if the word was generated by the phonetic module; 0 otherwise;
\item \textsc{isDiac}: 1 if the word was generated by the diacritic module; 0 otherwise;
\item \textsc{typedProb}: the unigram probability of the word typed;
\item \textsc{genUniProb}: the unigram probability of the word suggestion;
\item \textsc{typedTriProb}: the trigram probability of the word typed;
\item \textsc{genTriProb}: the trigram probability of the word suggestion;
\item \textsc{typoLevDist}: the levenshtein distance between the typed word and the suggestion;
\item \textsc{insKeyDist}: the sum of the key insertion distances;
\item \textsc{delKeyDist}: the sum of the key deletion distances;
\item \textsc{replKeyDist}: the sum of the key replacement distances;
\item \textsc{keyDists}: the sum of all previous three types of key distances;
\item \textsc{phoneLevDist}: the levenshtein distance between of the phonetic transcription of the typed word and of the suggestion. 
\end{enumerate}

The probabilities come from a language model trained over a subset of the Corpus Brasileiro (\emph{circa} 10 million tokens). Good-Turing smoothing is used to estimate the probability of unseen trigrams.
After feature extraction, the word selection phase comes into play. It consists of a Decision Tree Classifier which was trained over the dataset presented in Section~\ref{sec:dataset}, with the features we discussed. 
The classifier was implemented through scikit-learn \cite{Pedregosa2011} and comprises an optimized version of the CART algorithm. Several other classification algorithms were tested, but since our features contain both nominal and numerical data, and since some of them are dependent, the Decision Tree Classifier achieved the best performance.

\subsection{Dataset}\label{sec:dataset}

The evaluation corpus was compiled specially for this research and is composed of a set of  annotated product reviews, written by users on Buscap\'e\footnote{http://www.buscape.com.br/}, a Brazilian price comparison search engine. All misspelled words were marked, the correct expected form was suggested and the misspelling category was indicated.

We used snowball sampling to obtain a reasonable amount of data with incorrect orthography. A  list of ortographical errors with frequency greater than 3 in the  the corpus of product reviews compiled by \cite{Hartmann2014}
gathered by Avan\c{c}o ~\cite{Avanco2014} 
was used to pre-select, from the same corpus, sentences with at least
one incorrect word. Among those, 1,699 sentences were randomly
selected to compose the corpus (38,128 tokens). All these sentences
were annotated by two linguists with prior experience in corpus annotation. The inter-rater agreement for the error detection task is described in Table~\ref{tab:kappa}.

\begin{table}[ht!]
\centering
\caption{\it Inter-rater agreement for the error detection task}
\begin{tabular}{|lllll|}
\hline
 &  & \multicolumn{ 2}{c}{\textbf{Annot. B}} &  \\ \cline{3-4}
 &  & \textbf{Correct} & \textbf{Wrong} & \textbf{Total} \\ \hline
\textbf{Annot. A} & \textbf{Correct} & 33,988 & 512 & 34,500 \\
 & \textbf{Wrong} & 76 & 3,559 & 3,635 \\ \hline
\multicolumn{ 2}{|r}{\textbf{Total}}  & 34,064 & 4,071 & 38,135 \\ \hline
\end{tabular}
\label{tab:kappa}
\end{table}

The agreement was evaluated by means of the kappa test \cite{Carletta1996}. The $\kappa$
value for the error detection task was $0.915$ which stands for good reliability or almost perfect 
agreement \cite{Landis1977}. The final version of the corpus used 
to evaluate all methods was achieved by submitting both annotations to an adjudication
phase, in which all discrepancies were resolved. We noticed that most annotation 
problems consisted of whether or not to correct abbreviations, loanwords, proper nouns, internet slang, and technical jargon. In order to enrich the annotation and the evaluation procedure, we classified the misspellings into five categories:

\begin{enumerate}
\setlength{\itemsep}{1pt}
\item \textsc{typo}: misspellings which encompass a typographical problem (character insertion, deletion, replacement or transposition), usually related to key adjacency or fast typing; e.g. "obrsevei" instead of "observei" (\emph{I noticed}) and "memso" instead of "mesmo" (\emph{same}).
\item \textsc{phono}: cognitive misspellings produced by lack of understanding of letter-to-sound correspondences, e.g. "esselente" for "excelente" (\emph{excelent}), since both "ss" and "xc", in this context, sound like [s]. 
\item \textsc{diac}: this class identifies misspellings which are related to the inserting, deleting or replacing diacritics in a given word, e.g. "organizacao" instead of "organiza\c{c}\~ao" (\emph{organization}).
\item \textsc{int\_slang}: use of internet slang or emoticons, such as "vc" instead of "voc\^e" (\emph{you}), "kkkkkk" (to indicate laughter) or ":-)".
\item \textsc{other}: other types of errors that do not belong to any of the above classes, such as abbreviations, loanwords, proper nouns, technical jargon; e.g. "aprox" for "aproximadamente" (\emph{approximately}).
\end{enumerate}

The distribution of each of these categories of errors can be found in Table \ref{tab:error-dist}. The difference between the total number of counts in  Table \ref{tab:kappa} and \ref{tab:error-dist} is caused by 
spurious orthographies which were reconsidered or removed in the adjudication phase. In addition to the five categories previously listed, we also classified the misspellings into either contextual or non-contextual; i.e. if the misspelled word corresponds to another existing word in the dictionary, it is considered a contextual error (or real-word error). For instance, if the intended word was ``est\'a'' (\emph{he/she/it is}), but the user typed ``esta'', without the acute accent, it is classified as a contextual error, since ``esta'' is also a word in Brazilian Portuguese which means \emph{this} \textsc{fem}. 

The corpus has been made publicly available\footnote{Link omitted for blind review.} and intends to be a benchmark for future research in spelling correction for user generated content in BP.

\begin{table}[ht!]
\centering
\caption{\label{tab:error-dist} {\it Error distribution in corpus by category}}
\begin{tabular}{|cccc|}
\hline
\multicolumn{ 2}{|c}{\textbf{Misspelling type}} & \textbf{Counts} & \textbf{\% Total} \\ \hline
\textsc{typo} & - & 1,027 & 25.2 \\
\textsc{phono} & Contextual & 49 & 1.2 \\ 
 & Non-contextual & 683 & 16.7 \\ 
\textsc{diac} & Contextual & 411 & 10.1 \\ 
 & Non-contextual & 1,626 & 39.8 \\ 
\textsc{int\_slang} & - & 201 & 4.9 \\
\textsc{other} & - & 86 & 2.1 \\ \hline
\multicolumn{ 2}{|l}{\textbf{Total/Avg}} & 4,083 & 100.0 \\ \hline
\end{tabular}
\end{table}

\subsection{Evaluation Metrics}
Four performance measures are used to evaluate the spellers. The \emph{Detection rate} is the ratio between the number of errors detected and the total number of errors. The \emph{Correction rate} stands for the ratio between the number of corrected errors and the total number of errors. \emph{False positive rate} is the ratio between the number of
false positives (correct words that are wrongly detected as errors) and the total number of correct words. The \emph{False negative rate} consists of the ratio between the number of
false negatives (wrong words that are detected as correct) and the total number of errors.
In addition, the correction hit rates are evaluated by misspelling categories. In the analysis, we do not take into account the "int\_slang" and "other" categories, since both show a very irregular behavior and constitute specific types of spelling correction.

\section{Discussion}\label{sec:speller-results}

In Table~\ref{tab:sys-comparison}, we summarize all methods' results. As one can observe, the GPM-ML achieved the best overall performance, with the best results in at least three rates: detection, correction and false positive. Both methods we proposed in this paper, GPM and GPM-ML, performed better than the baseline in all metrics. However, GPM did not show any improvement in comparison to the benchmark. In fact, the addition of the grapheme-to-phoneme converter decreased the performance in what concerns to the correction rate. By analyzing the output of GPM, we noticed that there seems to be some overlapping information between the phonetic rules and the grapheme-to-phoneme module. Apparently, the phonetic rules were able to cover all cases which could be solved by adding the grapheme-to-phoneme converter. Therefore our hypothesis was not supported. 


\begin{table}[!ht]
\centering
\caption{\label{tab:sys-comparison} {\it Comparison of the Methods}}
\begin{tabular}{|ccccc|}
\hline
 & \multicolumn{4}{c|}{\textbf{Rate}} \\ \cline{2-5}
\textbf{Method} & \textbf{Detection} & \textbf{Correction} & \textbf{FP} & \textbf{FN} \\ \hline
Baseline JaSpell & 74.0\% & 44.7\% & 5.9\% & 26.0\% \\
Benchmark Rules\&Soundex & 83.4\% & 68.6\% & 1.7\% & \textbf{16.6\%} \\
GPM & 83.4\% & 68.2\% & 1.7\% & \textbf{16.6\%} \\
GPM-ML  & \textbf{84.9\%} & \textbf{78.1\%} & \textbf{0.7\%} & 21.9\% \\ \hline
\end{tabular}
\end{table}


All methods showed a low rate of false positives, the best value was found in GPM-ML (0.7\%). The false positive rate is very important for spelling correction purposes and is related to the reliability of the speller. 
If a user knows that a certain word is correct, but the speller says the opposite, the user is very likely to gradually lose confidence on the speller and, as a consequence, he/she stops using it.
In the following we discuss the correction hit rates by misspelling categories. Table~\ref{tab:corr-rate-by-speller} presents a comparison among all methods.

{\setlength{\tabcolsep}{0.6em}
\begin{table}[htbp]
\centering
\caption{\it Comparison of Correction Rates}
\begin{center}
\begin{tabular}{|cccccc|}
\hline
\multicolumn{1}{|c}{} & \multicolumn{1}{c}{} & \multicolumn{ 4}{c|}{\textbf{Correction rate by method}} \\ \cline{ 3- 6}
\multicolumn{1}{|c}{\textbf{Misspelling type}} & \multicolumn{1}{c}{\textbf{Errors}} & \multicolumn{1}{c}{\bf I} & \multicolumn{1}{c}{\bf II} & \multicolumn{1}{c}{\bf III} & \multicolumn{1}{c|}{\bf IV} \\ \hline
Typo & 1,027 & 28.3\% & \textbf{56.3\%} & 53.0\% & 55.4\% \\ 
Phono Contextual & 49 & 0.0\% & 0.0\% & 0.0\% & \textbf{8.1\%} \\ 
Phono Non-contextual & 683 & 48.2\% & 85.1\% & \textbf{87.1\%} & 81.1\% \\ 
Diac Contextual & 411 & 9.2\% & 26.5\% & 26.5\% & \textbf{64.5\%} \\ 
Diac Non-contextual & 1626 & 64.0\% & 82.2\% & 82.4\% & \textbf{96.6\%} \\ \hline
\textbf{Total/Weighted Avg} & 3,796 & 44.7\% & 68.6\% & 68.1\% & \textbf{78.0\%} \\ \hline

\end{tabular}
\end{center}
\label{tab:corr-rate-by-speller}
\end{table}
}

The baseline JaSpell (Method I) presented an average correction rate of 44.7\%. Its best results comprise non-contextual diacritic misspellings with a rate of 64.0\%. Its worst result is found in contextual phonological errors, not a single case of this type of error was corrected by the speller. The typographical
misspelling were also very troublesome for the baseline method, with a correction 
hit rate of 28.3\%. These results indicate that the method is not suitable for real world applications which deal with user generated content. It is important to notice that the JaSpell was not developed specifically for this text domain, so its performance is much influenced by this fact.

The benchmark Rules\&Soundex (Method II) achieved a correction rate of 68.6\%, a relative gain of 53.4\% in comparison to the baseline. The best results are, once more, related to the non-contextual diacritic misspellings (82.2\%), which stand for the major class. The best improvements compared to the baseline appear in phonological errors that are influenced by context (85.1\%), with a relative increase of 76.6\%. These results are coherent with the results reported by \cite{Avanco2014}, since they claim that the method focuses on phonetically motivated misspellings.
As already mentioned, GPM (Method III) did not show any gain in comparison with the benchmark. As can be noticed, the grapheme-to-phoneme converter had a small positive impact in what regards to the phonological errors, raising the correction rate of non-contextual phonological misspellings from 85.1\% to 87.1\% (2.3\% gain). 

GPM-ML (Method IV) achieved the best performance among all methods in what regards to correction hit rate (78.0\%). Some misspelling categories showed a very high correction rate, such as non-contextual diacritic errors (96.6\%) and non-contextual phonological errors (81.1\%). The trigram Language Model proved to be effective for capturing some contextual misspellings, as can be seen by the contextual diacritic correction rate (64.5\%). However, the method was not able to properly infer contextual phonological misspellings (8.1\%). We hypothesize that this result might be caused by the few number of contextual phonological instances in the corpus used for training (there were only 49 cases of contextual phonological misspellings). Such a small number of cases is not adequate for ensuring good performance by machine learning techniques. No significant improvement was found with respect to typographical errors (55.4\%) in comparison to the other previous methods.


\section{Related Work}\label{sec:speller-related}

The first approaches to spelling correction date back to Damerau \cite{Damerau1964} and address the problem by analyzing the edit distance of the words. He proposes a speller based on a reference dictionary and on an algorithm to check for out-of-vocabulary (OOV) words. The method assumes that words which are not found in the dictionary have at most one error, which was caused by a letter insertion, deletion, substitution or transposition. OOV words are then compared to the words from the dictionary. The one error threshold was established to avoid high computational cost.
An improved error model for spelling correction, which works for letter sequences of lengths up to 5 and is also able to deal with phonetic errors was proposed by \cite{Brill2000}. It embeds a noisy channel model for spell checking based on string to string edits. This model depends on the probabilistic modeling of sub-string transformations. 
As texts present several kinds of misspellings, no single method will cover all of them, therefore it is very natural to combine methods which supplement each other. This approach was pursued by \cite{Toutanova2002} who included information on pronunciation to the model of typographical errors correction. \cite{Toutanova2002} and also \cite{Berkel1988} took the pronunciation of the misspelled words into account by using the technology of grapheme-to-phoneme converters. The later proposed the use of triphone analysis as a new correction strategy to combine phonemic transcription with trigram analysis, since they performed better than either grapheme-to-phoneme conversion or trigram analysis alone, in their evaluation. 
Our GPM method also combines models to correct typographical errors by using information on edition distance, information on pronunciation provided by a set of phonetic rules, on a grapheme-to-phoneme converter and finally on the output of the Soundex method. In this two-layer method these modules are put in sequence, as we take advantage of the high precision of the phonetic rules before trying the converter; typographical errors are corrected in the last pass of the process. We understand that the probabilistic classification framework used by \cite{Toutanova2002} is very interesting and would provide better results to our two-layer method. Therefore, we decided to take advantage of a machine learning approach to decide how to correct a word, by using candidates generated by one and two edit distance, phonetic similarity and word combinations of diacritics. In our GPM-ML proposal, we adapted the output of a grapheme-to-phoneme converter which was developed for automatic speech recognition, and used it together with a keyboard model and a language model to provide features for a decision tree classifier.  We had to broaden the transcriptions in order to deal with real-word errors related to diacritics, since the transcriptions are too much detailed for spelling correction purposes. With this new proposal one can deal with a special group of real-word errors caused by the presence or absence of diacritics, besides phonetic and typographic errors.

\section{Final Remarks}\label{sec:speller-final-remarks}

We compared four spelling correction methods for UGC in BP, two of which consist of novel approaches and were proposed in this paper. The Method III (GPM) consisted of an upscale version of the benchmark method. In comparison to benchmark, it contained an additional module with a grapheme-to-phoneme converter. The grapheme-to-phoneme converter was intended to provide the speller with transcriptions that were not so fine-grained or specific as those generated by the phonetic rules and also not so coarse-grained as those created by Soundex. However, it didn't work as well as expected.  The Machine Learning version of GPM, the GPM-ML, however, presented a good overall performance, as it is the unique that addresses the problem of real word errors, and surpass all other methods in most situations.  It reached 78.0\% in correction rate, with very low false positive (0.7\%) and false negative (21.9\%), thus establishing the new state of the art in spelling correction for UGC in BP. As for future work, we intend to improve GPM-ML by expanding the training database, by testing other language models as well as new phone conventions. In addition, we plan to more fully evaluate it into different testing corpora. We also envisage, in due course, the development of an internet slang module.

%*****************************************
%*****************************************
%*****************************************
%*****************************************
%*****************************************

\invisiblesection{A Method for the Extraction of Phonetically-Rich Triphone Sentences}
\includepdf[pages={1-},scale=0.85, pagecommand={}]{Chapters/PaperPhonRich.pdf}\label{sec:phon-rich}
%\include{Chapters/ChapPhonRich}
\cleardoublepage

\invisiblesection{Listener: A prototype system for automatic speech recognition and evaluation of
Brazilian-accented English}
\includepdf[pages={1-},scale=0.85, pagecommand={}]{Chapters/Listener/bmc_article.pdf}\label{sec:listener}
%
%*****************************************
\chapter{Listener}\label{ch:listener}
%*****************************************

O reconhecedor de pron\'uncia ora proposto ser\'a implementado a partir do
motor de reconhecimento de fala Julius (Lee \& Kawahara, 2009). Nove
erros de pron\'uncia foram selecionados para serem tratados pelo Listener,
assumindo-se, como pron\'uncia padr\~ao, o General American (GA). O modelo
ac\'ustico ser\'a compilado a partir de tr\^es corpora. Um de falantes nativos
de ingl\^es: TIMIT Acoustic-Phonetic Continuous Speech Corpus{[}16{]}; e
outros dois de aprendizes: COBAI - Corpus Oral Brasileiro de Aprendizes
de Ingl\^es{[}17{]} e um corpus coletado especialmente para este trabalho,
composto por leitura de senten\c{c}as foneticamente balanceadas. O
dicion\'ario a ser empregado \'e o CMU Pronouncing Dictionary, ao qual ser\~ao
acrescentaremos as hip\'oteses de pron\'uncia dos aprendizes, por meio de
regras. O modelo de l\'ingua ser\'a gerado a partir da Simple English
Wikipedia em conjunto com um corpus de textos escritos por aprendizes de
ingl\^es, o COMAprend, um dos tr\^es corpus do projeto COMET da Faculdade de
Filosofia, Letras e Ci\^encias Humanas da Universidade de S\~ao Paulo. A
efici\^encia do reconhecedor ser\'a avaliada por meio de medidas de Word
Error Rate (WER), Character Error Rate (CER) e matrizes de confus\~ao,
aplicadas por meio de ten-fold cross validation sobre os dados dos
corpora coligidos. De modo a verificar a viabilidade do m\'etodo ora
proposto, um prot\'otipo do sistema foi elaborado e avaliado, detalhes s\~ao
apresentados na Se\c{c}\~ao .

4 MATERIAIS E M\'ETODOS

1 Levantamento dos Desvios de Pron\'uncia

Na classifica\c{c}\~ao dos erros de pron\'uncia, deu-se prioridade,
especialmente, aos erros de pron\'uncia que afetam a compreens\~ao e que s\~ao
apresentados em trabalhos que consideram, no ensino da pron\'uncia do
ingl\^es, a transfer\^encia de padr\~oes sonoros de L1 para L2.

A listagem dos erros de pron\'uncia a serem considerados pelo Listener foi
obtida a partir da consulta aos trabalhos de Zimmer (2004), Godoy
(2005), Zimmer et al. (2009) e Crist\'ofaro-Silva (2012). Tais trabalhos
analisam aspectos de transfer\^encia de L1 para L2 e estabelecem um m\'etodo
de ensino de pron\'uncia que leva em conta essa transfer\^encia, a fim de
otimizar o aprendizado pelo aluno. De tal maneira, centra-se o estudo no
ensino dos padr\~oes de pron\'uncia que devem ser enfatizados para o falante
do PB, a fim de melhor garantir a compreens\~ao de sua pron\'uncia e sua
efici\^encia comunicativa. No reconhecedor, optou-se por utilizar os nove
tipos de erros elencados em Zimmer et al. (2009), por se tratar da
investiga\c{c}\~ao mais abrangente sobre o assunto. Os desvios de pron\'uncia
selecionados est\~ao sintetizados no Quadro 5.

   Quadro 5: Desvios de pron\'uncias a ser analisados pelo Listener.

                                [pic]

2 Simplifica\c{c}\~ao sil\'abica

Um conjunto de 30 regras foi definido para gerar as variantes de
pron\'uncia envolvendo casos de simplifica\c{c}\~ao sil\'abica. As regras se
baseiam nos exemplos citados por Rauber e Baptista (2004), Rebello e
Baptista (2007), Zimmer (2009) e Silveira (2012). A discuss\~ao dos
contextos consta na Se\c{c}\~ao 2.1.1.4.1. O pseudoc\'odigo com a implementa\c{c}\~ao
das regras est\'a descrito na Figura 13.

Figura 13: Pseudoc\'odigo com regras para gera\c{c}\~ao das variantes de
pron\'uncia envolvendo simplifica\c{c}\~ao sil\'abica.

Como se observa, as regras de simplifica\c{c}\~ao sil\'abica buscam cobrir,
majoritariamente, quatro situa\c{c}\~oes: (i) oclusivas em posicao final de
palavra; (ii) palavras foneticamente terminadas em consoantes, mas com
\textless{}-e\textgreater{} final na forma escrita; (iii) clusters
consonantais em in\'icio de palavra do tipo /sC(C)/; (iv) palavras
iniciadas por /s ao/, com na forma ortogr\'afica.

3 O Motor de Reconhecimento de Fala Julius

Neste trabalho, propomos a utiliza\c{c}\~ao do motor de reconhecimento de fala
Julius (Lee \& Kawahara, 2009), como a base do reconhecedor de pron\'uncia
a ser desenvolvido. Julius \'e uma engine de alto desempenho e de c\'odigo
aberto para a constru\c{c}\~ao de sistemas de reconhecimento de fala. Ele
incorpora grande parte das t\'ecnicas do estado da arte em reconhecimento
de fala e executa Reconhecimento de Fala Cont\'inuo com Grande Vocabul\'ario
(LVCSR). Sua arquitetura est\'a sintetizada na Figura 14.

                                [pic]

Figura 14: Arquitetura do motor de reconhecimento de fala Julius (Lee \&
Kawahara, 2009).

Como se observa, ele suporta a entrada de dados de \'audio vindo de
microfone, de arquivos j\'a gravados, ou de streaming via internet. A
sa\'ida \'e composta ou por dados de textos, a exemplo de ditado, ou por uma
determinada a\c{c}\~ao solicitada ao Julius. No caso do reconhecedor de
pron\'uncia, a sa\'ida ser\'a constitu\'ida pela transcri\c{c}\~ao da fala e da
avalia\c{c}\~ao de sua pron\'uncia.

  Para  se  construir  um  reconhecedor  fala  atrav\'es  do  Julius,  s\~ao

necess\'arios: um modelo ac\'ustico, um dicion\'ario de pron\'uncia e um modelo
de l\'ingua. Mais detalhes sobre a elabora\c{c}\~ao desses modelos, de maneira a
possibilitar o desenvolvimento do reconhecedor de pron\'uncia, s\~ao
fornecidos a seguir.

4 Elabora\c{c}\~ao do Modelo Ac\'ustico

O modelo ac\'ustico proposto ser\'a elaborado atrav\'es de HMM e definido para
trifones. Julius prov\^e suporte a modelos ac\'usticos de HMM obtidos a
partir do HTK Hidden Markov Model Toolkit{[}18{]}, disponibilizado pelo
Speech Vision and Robotics Group, da Universidade de Cambridge. A
abordagem utilizada na elabora\c{c}\~ao do modelo ac\'ustico \'e a de interl\'ingua.
Portanto, ele ser\'a estimado a partir de dois tipos de corpora de fala:
um de falante nativos do ingl\^es: TIMIT Acoustic-Phonetic Continuous
Speech Corpus{[}19{]}; outros dois de falantes nativos do PB, aprendizes
de ingl\^es como L2: COBAI - Corpus Oral Brasileiro de Aprendizes de
Ingl\^es{[}20{]} - e um corpus coletado especificamente para o Listener,
composto por leitura de senten\c{c}as foneticamente balanceadas, isto \'e,
senten\c{c}as contendo fones de acordo com sua frequ\^encia de ocorr\^encia em
uma dada l\'ingua. A constru\c{c}\~ao de um modelo ac\'ustico interlingual busca
contornar a dificuldade de reconhecer a fala de n\~ao-nativos, atrav\'es da
inser\c{c}\~ao de informa\c{c}\~ao da pron\'uncia n\~ao-nativa no processo de
treinamento do modelo ac\'ustico, por meio de dados com a pron\'uncia dos
aprendizes (Wang, Schultz, \& Waibel, 2003).

5 Corpus de Nativos: TIMIT Acoustic-Phonetic Continuous Speech Corpus

Optou-se pela utiliza\c{c}\~ao do TIMIT (Garofolo, et al., 1993) como o corpus
de falantes nativos ingl\^es por se tratar de: um corpus bem modelado,
robusto, foneticamente rico, amplamente utilizado e testado na \'area de
reconhecimento de fala, em cerca de duas d\'ecadas de pesquisa, al\'em de
cobrir os dialetos majorit\'arios do ingl\^es americano (Lopes \& Perdig\~ao,
2011). O corpus TIMIT foi elaborado, conjuntamente, pelo Instituto de
Tecnologia de Massachusetts (MIT), SRI Internacional e Texas Instruments
Inc. (TI) com o prop\'osito fornecer dados para a realiza\c{c}\~ao de estudos de
fon\'etica ac\'ustica do ingl\^es, bem como para o desenvolvimento de sistemas
autom\'aticos de reconhecimento de fala. Ele cont\'em grava\c{c}\~oes de cerca de
630 falantes, dos oito principais dialetos do ingl\^es americano. As
grava\c{c}\~oes foram elaboradas a partir da leitura de dez senten\c{c}as criadas
artificialmente, de modo a capturar ambientes fon\'eticos relevantes. O
TIMIT foi verificado manualmente e est\'a transcrito ortogr\'afica e
foneticamente, adicionalmente, foi feito o alinhamento temporal entre o
arquivo de \'audio e as transcri\c{c}\~oes. Os arquivos est\~ao separados em
senten\c{c}a, amostrados a 16kHz com 16 bits por amostra. O TIMIT est\'a
dispon\'ivel para venda no Linguistic Data Consortium (LDC) e ser\'a
adquirido pelo N\'ucleo Interinstitucional de Lingu\'istica Computacional
(NILC).

6 Corpus de N\~ao-nativos: Corpus Oral Brasileiro de Aprendizes de Ingl\^es
(COBAI)

O COBAI (Mello, Avila, Neder-Neto, \& Orfano, 2012) constitui a primeira
iniciativa brasileira que busca compilar e distribuir, de forma aberta,
um corpus de fala anotado de aprendizes de ingl\^es, falantes nativos do
PB. O COBAI integra o Louvain International Database of Spoken English
Interlanguage (LINDSEI) e vem sendo organizado pelo Laborat\'orio de
Estudos Emp\'iricos e Experimentais da Linguagem (LEEL), da Faculdade de
Letras, da Universidade Federal de Minas Gerais (UFMG). O prop\'osito do
LINDSEI \'e a disponibiliza\c{c}\~ao de corpora de fala de aprendizes de ingl\^es,
com diferentes backgrounds de l\'ingua nativa. O COBAI segue as diretrizes
de transcri\c{c}\~ao do LINDSEI, que utiliza padr\~oes XML na anota\c{c}\~ao. A
transcri\c{c}\~ao \'e do tipo ortogr\'afica e agrega informa\c{c}\~oes de: troca de
turno, sobreposi\c{c}\~ao de fala, pausas, hesita\c{c}\~oes, formas reduzidas e
algumas indica\c{c}\~oes fon\'eticas e pros\'odicas. Atualmente, cerca de 60\% do
corpus est\'a anotado. O corpus consiste em 50 grava\c{c}\~oes de 15 minutos,
que incorporam uma narrativa, uma entrevista e uma descri\c{c}\~ao. Os
arquivos est\~ao separados e amostrados a 44kHz com 16 bits por amostra.
Todas as grava\c{c}\~oes foram feitas com falantes nativos do PB, aprendizes
de ingl\^es. O grau de conhecimento da l\'ingua inglesa dos participantes \'e
variado, havendo desde aprendizes com baixa profici\^encia at\'e indiv\'iduos
proficientes.

Uma contribui\c{c}\~ao colateral deste projeto ser\'a a finaliza\c{c}\~ao da
transcri\c{c}\~ao ortogr\'afica do COBAI e a anota\c{c}\~ao, no corpus, dos nove tipos
de erros que ser\~ao analisados pelo Listener. De modo a finalizar a
transcri\c{c}\~ao ortogr\'afica do COBAI, pretende-se utilizar um reconhecedor
de fala, no caso, o Dragon Naturally Speaking v.12 Premium, de modo a
obter uma transcri\c{c}\~ao ortogr\'afica inicial. A partir disso, realizaremos
a revis\~ao da transcri\c{c}\~ao, no intuito de corrigir os erros e adequar o
formato ao que \'e proposto pelo LINDSEI. Para a transcri\c{c}\~ao dos desvios
de pron\'uncia, prop\~oe- se o seguinte m\'etodo: ap\'os ter-se obtido a
transcri\c{c}\~ao ortogr\'afica de todo o corpus, ser\'a criado um script em
Python para percorrer cada palavra transcrita ortograficamente,
conferi-la no CMUdict e extrair a transcri\c{c}\~ao fon\'etica que l\'a est\'a
registrada. De tal forma, obteremos uma vers\~ao do COBAI transcrito com a
pron\'uncia can\^onica do General American (GA), que est\'a registrada no
CMUdict. A seguir, ser\'a realizada a revis\~ao do das transcri\c{c}\~oes
fon\'eticas, corrigindo-as quando os aprendizes cometerem algum dos nove
tipos de erros que o Listener avaliar\'a.

7 Corpus de N\~ao-nativos: Corpus de Leitura de Senten\c{c}as Foneticamente
Balanceadas por Aprendizes

A inten\c{c}\~ao inicial do projeto era utilizar apenas o COBAI como corpus
fala de n\~ao-nativos. Por\'em, ap\'os ter-se desenvolvido o prot\'otipo do
reconhecedor, conforme ser\'a descrito na Se\c{c}\~ao , foi poss\'ivel observar
que muitos dados do COBAI ter\~ao de ser desconsiderados e, de tal forma,
somente o COBAI n\~ao ser\'a suficiente para fornecer o n\'umero de horas
necess\'ario para a estima\c{c}\~ao de um bom modelo ac\'ustico. Por isso,
decidiu-se criar um corpus espec\'ifico para o desenvolvimento deste
trabalho.

O prop\'osito \'e compilar um corpus de aprendizes, em situa\c{c}\~ao de leitura
de frases pr\'e-definidas, foneticamente balanceadas, o qual seja gravado
em um ambiente com isolamento ac\'ustico. Os objetivos, com isso, s\~ao
tr\^es: i) assegurar a boa qualidade do \'audio, mantendo baixa a rela\c{c}\~ao
sinal-ru\'ido; ii) garantir que todas as combina\c{c}\~oes de trifones estejam
presentes na base de dados do modelo ac\'ustico, uma vez que as senten\c{c}as
ser\~ao foneticamente balanceadas; e iii) facilitar a tarefa de
transcri\c{c}\~ao, j\'a que a dura\c{c}\~ao da pesquisa de mestrado \'e curta. Prop\~oe-se
seguir as diretrizes de compila\c{c}\~ao e anota\c{c}\~ao de corpora, tal como
descrito por Hovy e Lavid (2010).

Pretende-se que os detalhes do m\'etodo de compila\c{c}\~ao e anota\c{c}\~ao do corpus
sejam definidos em visita t\'ecnica a ser realizada na Universidade de
Coimbra, sob supervis\~ao da Profa. Sara Candeias, no per\'iodo de 28 de
janeiro a 28 de fevereiro de 2014. O plano de tarefas proposto para a
visita inclui:

a defini\c{c}\~ao do tamanho do corpus a ser compilado;

a defini\c{c}\~ao do n\'ivel de detalhe da transcri\c{c}\~ao fon\'etica (qual o
invent\'ario de fones e xenofones utilizar);

a defini\c{c}\~ao do tipo de hesita\c{c}\~ao ou disflu\^encia a ser anotada;

a discuss\~ao de uma m\'etrica de riqueza fon\'etica (?), proposta para a
extra\c{c}\~ao de senten\c{c}as foneticamente balanceadas;

o teste e a avalia\c{c}\~ao da m\'etrica, a partir de um corpus de textos de
aprendizes de ingl\^es, o COMAprend.

No que diz respeito às senten\c{c}as foneticamente balanceadas, h\'a, para o
ingl\^es, diversas listas dispon\'iveis, como as Harvard Sentences (IEEE,
1969), os TIMIT Sentence Prompts (Garofolo, et al., 1993), as
MOCHA-TIMIT Sentences (Wrench, 1999), al\'em de diversas listas fornecidas
pela Carnegie Mellon University para o motor de reconhecimento Sphinx
(Lee, Hon, \& Reddy, 1990). No entanto, dado que essas listas foram
elaboradas para falantes nativos de ingl\^es, h\'a diversas palavras de
baixa frequ\^encia, bem como senten\c{c}as cuja estrutura sint\'atica \'e pouco
usual. Em um contexto de aprendizes, isso \'e problem\'atico, pois pode
ocasionar disflu\^encias na fala do aprendiz, causando problemas na
utiliza\c{c}\~ao dos dados, al\'em de padr\~oes de pron\'uncia altamente
irregulares, quando o aprendiz desconhecer a palavra que est\'a lendo.

Para contornar o problema, objetivamos criar um conjunto de senten\c{c}as
foneticamente balanceadas, a partir de um corpus de texto de brasileiros
aprendizes de ingl\^es, como o COMAprend (Tagnin \& Fromm, 2009). O
COMAprend possui textos em formato ortogr\'afico. A fim de se obter a
transcri\c{c}\~ao fon\'etica dos textos, ser\'a elaborado um script semelhante ao
descrito, anteriormente, na Se\c{c}\~ao 3.1.3.2.

8 Elabora\c{c}\~ao do Dicion\'ario de Pron\'uncia

O dicion\'ario de pron\'uncia ser\'a formado com base no CMU Pronouncing
Dictionary, o qual ser\'a acrescido de transcri\c{c}\~oes das poss\'iveis
pron\'uncias desviantes dos aprendizes, por meio de regras
transformacionais. Dicion\'arios contendo tais caracter\'isticas s\~ao tamb\'em
chamados na literatura como dicion\'arios multipron\'uncia (Strik \&
Cucchiarini, 1999).

9 CMU Pronouncing Dictionary

A base do dicion\'ario provir\'a do CMU Pronouncing Dictionary (tamb\'em
conhecido por CMUdict), disponibilizado no motor de reconhecimento de
fala Sphinx, pela Universidade Carnegie Mellon. O CMUdict constitui um
dicion\'ario de pron\'uncia machine readable de refer\^encia na \'area de
reconhecimento de fala. Atualmente, nele est\~ao registradas 131.411
entradas, transcritas foneticamente em formato ARPAbet (Zue \& Seneff,
1988). O Quadro 6 ilustra a entrada de algumas palavras no dicion\'ario.

Quadro 6: Exemplo de entradas no dicion\'ario de pron\'uncia do CMU
Pronouncing Dictionary.

                                [pic]

Como se observa, o dicion\'ario possui tr\^es campos: (i) um interno, para
identifica\c{c}\~ao da palavra, (ii) um com a palavra em sua forma
ortogr\'afica, convencionalizada em letras mai\'usculas, (iii) e um \'ultimo
campo com a transcri\c{c}\~ao fon\'etica da palavra, em formato ARPAbet.

10 Adi\c{c}\~ao das Formas Variantes de Pron\'uncia do Aprendiz

Ser\~ao utilizadas regras transformacionais para acrescentar ao dicion\'ario
as poss\'iveis hip\'oteses de pron\'uncia do aprendiz. A utiliza\c{c}\~ao de regras
transformacionais \'e de f\'acil implementa\c{c}\~ao computacional, correspondendo
a simples estruturas de sele\c{c}\~ao, ou constru\c{c}\~oes condicionais. Contexto
para aplica\c{c}\~ao de regras, como os elencados no Quadro 7, ser\~ao
levantadas, de acordo com a literatura lingu\'istica de ASL, e utilizados
para as variantes de pron\'uncias do aprendizes, a partir das palavras do
CMUdict. Casos marginais, de contexto muito restrito ou que n\~ao se
adaptem às regras criadas, ser\~ao adicionados, manualmente, em formato de
dicion\'ario de exce\c{c}\~oes.

Quadro 7: Contextos para aplica\c{c}\~ao de regras de simplifica\c{c}\~ao sil\'abica.

                                [pic]

11 Elabora\c{c}\~ao do Modelo de L\'ingua

Um modelo de l\'ingua ser\'a fornecido ao Listener, de modo a possibilitar
seu uso em um contexto de ditado. H\'a diversos modelos de l\'ingua para o
ingl\^es (como o Gigaword{[}21{]}, CSR LM-1{[}22{]}, HUB4{[}23{]}). Por\'em
a grande maioria desses modelos foi gerada a partir de corpora de
artigos de jornal e \'e sabido que textos jornal\'isticos de jornais
tradicionais, para p\'ublicos A, B e C tendem a possuir estrutura
sint\'atica e vocabul\'ario complexos, dado que as primeiras ora\c{c}\~oes de uma
not\'icia tendem a compactar muita informa\c{c}\~ao e podem trazer jarg\~ao n\~ao
dominado por aprendizes (Canning, 2002). Como a inten\c{c}\~ao \'e lidar com a
fala de aprendizes, propomos a cria\c{c}\~ao de um modelo de l\'ingua que seja
mais simplificado e condizente com a sintaxe dos aprendizes. Ser\'a
elaborado um modelo de l\'ingua estat\'istico, que considera trigramas na
an\'alise e se baseia em HMMs.

12 Simple English Wikipedia

Como corpus para cria\c{c}\~ao do modelo de l\'ingua, utilizaremos a Simple
English Wikipedia, cuja proposta \'e desenvolver uma Wikipedia em ingl\^es
de n\'ivel b\'asico, com vocabul\'ario e constru\c{c}\~oes sint\'aticas mais simples,
de modo a prover acesso a crian\c{c}as, estudantes, adultos com baixo n\'ivel
de letramento e aprendizes de ingl\^es como L2. A vers\~ao dispon\'ivel da
Simple English Wikipedia, referente ao dia 16 de janeiro de 2014, possui
108.665{[}24{]}. Todos os arquivos est\~ao codificados em XML. A
ferramenta SRILM (The SRI Language Modeling Toolkit){[}25{]} ser\'a
utilizada para auxiliar na cria\c{c}\~ao do modelo de l\'ingua.

13 Elabora\c{c}\~ao da Interface Web

A interface web desenvolvida para o Listener emprega conceitos de
gamifica\c{c}\~ao, com o prop\'osito de estimular a participa\c{c}\~ao dos usu\'arios e
de tornar o processo de aprendizagem mais apraz\'ivel.

14 Gamifica\c{c}\~ao

Segundo o historiador Huizinga (1938), um jogo constitui:

``uma atividade ou ocupa\c{c}\~ao volunt\'aria, exercida dentro de certos e
determinados limites de tempo e de espa\c{c}o, segundo regras livremente
consentidas, mas absolutamente obrigat\'orias, dotado de um fim em si
mesmo, acompanhado de um sentimento de tens\~ao e de alegria e de uma
consci\^encia de ser diferente da 'vida cotidiana'' (Huiziga 1938).

Jogos t\^em, desde sempre, fascinado as pessoas. Milh\~oes de pessoas vibram
e se emocionam ao assistir a uma partida de futebol de seu time
preferido. Mestres do xadrez, como Garry Kasparov, chegam a dedicar
cerca de seis horas di\'arias para dominar o jogo (ICC, 1998). Massive
Multiplayer Online Role Playing Games mobilizam milh\~oes de pessoas
online a um mesmo tempo. H\'a, tamb\'em, casos tr\'agicos, como o do taiwan\^es
que teve problemas vasculares e faleceu, logo ap\'os jogar Diablo 3 em uma
lan-house por 40 horas ininterruptas (Daily Mail, 2012).

De acordo com Lazzaro (2004), as pessoas jogam, basicamente, por um
dentre os quatro motivos a seguir: (i) pelo divertimento que o jogo
proporciona, (ii) pela competitividade que ele incita, (iii) pelas
sensa\c{c}\~oes diversas que ele pode gerar - surpresa, alegria, temor, etc.;
e (iv) pelo pretexto para socializar com os amigos. Jogos bem desenhados
tendem sempre a explorar esses pontos, a fim de fidelizar jogadores.

Deburr (2013) define a gamifica\c{c}\~ao a aplica\c{c}\~ao de estrat\'egias e t\'ecnicas
usadas no design de jogos em outros contextos, que n\~ao jogos. Zicherman
e Cunningham (2013) trazem uma defini\c{c}\~ao an\'aloga, para os autores a
gamifica\c{c}\~ao \'e ``o processo de utilizar a mec\^anica dos jogos no intuito
de motivar usu\'arios a resolver problemas'', sendo que a mec\^anica
comporta sete aspectos: (i) sistema de pontos, (ii) n\'iveis, (iii)
rankings, (iv) atribui\c{c}\~ao de t\'itulos, (v) desafios/miss\~oes, (vi) busca
de jogadores e (vii) um loop de est\'imulo social constante. Explicar cada
um desses aspectos

A gamifica\c{c}\~ao tem sido aplicada com sucesso a uma gama de contextos. H\'a,
por exemplo, comunidades de perguntas e respostas, como Stackoverflow e
Yahoo! Respostas, que a empregam a fim de motivar seus usu\'arios a
participarem da comunidade e a cooperarem entre si. Os usu\'arios que
fornecem as melhores respostas recebem pontos no site, sobem n\'iveis,
melhoram nos rankings, tornando-se cada vez mais influentes na
comunidade. Aplicativos para celulares que incitam a pr\'atica de corrida,
como o Runstatic e o Nike+, tamb\'em t\^em utilizado a gamifica\c{c}\~ao com
\^exito. Tais aplicativos extraem dados do percurso do usu\'ario via GPS e
lhes fornece estat\'isticas de sua corrida, de maneira que os usu\'arios
podem monitorar suas corridas para vencer metas, bater recordes pessoais
e, tamb\'em, competir com os amigos. \'E poss\'ivel tamb\'em desafiar outros
usu\'arios, para ver quem corre mais r\'apido, quem faz um mesmo percurso
maior em menor tempo, etc.

No \^ambito da educa\c{c}\~ao, a gamifica\c{c}\~ao tem sido explorada no chamado
edutainment - palavra am\'algama formada a partir de education e
entertainment (XXX:XX). Segundo Zicherman e Cunningham (2013), a
ind\'ustria tem falhado em utilizar a gamifica\c{c}\~ao para aplica\c{c}\~oes
educacionais. Muitas vezes, o aspecto instrucional \'e ressaltado de
maneira exagerada, de modo que os jogos se tornam chatos e as pessoas se
sentem desmotivadas para jog\'a- los. Para os autores, o jogo educacional
que melhor explorou os aspectos da gamifica\c{c}\~ao foi ``Where in the World
Is Carmen Sandiego?'', cujo prop\'osito \'e ensinar Geografia, mais
especificamente, o nome e a localiza\c{c}\~ao dos pa\'ises e de suas capitais.
Em ``Where in the World Is Carmen Sandiego?'', o jogador encarna um
detetive que deve juntar pistas para encontrar a criminosa Carmen
Sandiego. Ao longo da jornada, o jogador visita v\'arios pa\'ises e encontra
pessoas que lhe d\~ao pistas sobre onde Carmen Sandiego pode estar
escondida. As pistas cont\^em informa\c{c}\~oes gerais sobre os pa\'ises, como
``Uma pessoa suspeita veio aqui e disse que iria viajar à terra dos
Vikings'' ou ``Eu a vi embarcando em um avi\~ao cuja bandeira era vermelha
e azul''. A Figura 15 apresenta uma tela do jogo.

  Figura 15: Captura de tela do jogo "Where in the World Is Carmen
                             Sandiego?".

``Where in the World Is Carmen Sandiego?'' foi lan\c{c}ado em 1985 e, desde
ent\~ao, diversas empresas de edutainment t\^em tentado repetir o sucesso do
jogo, mas sem muito \^exito. Merece destaque o lan\c{c}amento, em 2012, da
plataforma para ensino de idiomas Duolingo (XXX:XX), que emprega
gamifica\c{c}\~ao e, s\'o no Brasil, j\'a consta com cerca 6,8 milh\~oes de
alunos{[}26{]}. Como se trata de uma aplica\c{c}\~ao de ensino de l\'inguas que
possui tamb\'em um m\'odulo de ensino de pron\'uncia, o Duolingo ser\'a tratado
em mais detalhes na Se\c{c}\~ao XXX.

The leaderboard of today has seensome radical redesign since the heyday
of pinball machines and quarter arcades. In the era of Facebook and the
social graph, leaderboards are mostly tools for creating social
incentive

In many instances, such as losing weight or even writing a book, it's
difficult for a player to understand where he is at the outset or during
early interactions. Moreover, the length and complexity of the overall
journey is such that sometimes players can be paralyzed by the seeming
lack of progress. Especially in health, education, and other ``epic
journey'' contexts, feedback forms the most important overarching game
mechanic, intricately tied to score and progress.

LER Jon Radoff's Game On

LER Jesse Schell's The Art of Game Design: A Book of Lenses

Segundo Zicherman e Cunningham (2013), como modelo de neg\'ocios, a
gamifica\c{c}\~ao transforma a rela\c{c}\~ao empresa-usu\'ario em uma rela\c{c}\~ao
simbi\'otica, cujo benef\'icio para o usu\'ario \'e o prazer obtido com o jogo e
para a empresa \'e a fideliza\c{c}\~ao do usu\'ario com o produto oferecido.

O corpus de erros induzidos

Para este prot\'otipo, reduziu-se a cria\c{c}\~ao do corpus de leitura de
senten\c{c}as foneticamente balanceadas a uma tarefa mais simples. Um
informante do sexo masculino foi gravado, lendo palavras em isolamento
e, propositalmente, enunciando-as com erros de transfer\^encia de L1 para
L2, a fim de simular variantes de pron\'uncia que ocorrer\~ao no corpus. As
palavras foram selecionadas a partir da lista das 5.000 palavras mais
frequentes da l\'ingua inglesa, segundo o Corpus of Contemporary American
English (COCA){[}29{]}. Cerca de 2h20min de fala foram compiladas. A
grava\c{c}\~ao ocorreu em uma sala fechada, com baixo ru\'ido externo, n\~ao
isolada acusticamente.

O COCA \'e o maior corpus de ingl\^es dispon\'ivel atualmente, tendo sido
compilado a partir de cerca de 160.000 textos escritos e transcritos, de
v\'arios g\^eneros, os quais totalizam 450 milh\~oes de tokens. O projeto \'e
coordenado por Mark Davies, professor de Lingu\'istica de Corpus da
Brigham Young University (BYU). O corpus \'e gratuito, mas as listas de
frequ\^encia s\~ao pagas, a exce\c{c}\~ao da menor delas, contendo as 5.000
palavras mais frequentes do ingl\^es, a qual foi utilizada no prot\'otipo.

Um script em Python foi elaborado para aplicar à lista as regras de
simplifica\c{c}\~ao sil\'abica elencadas na Figura 18. Quando havia contexto
para aplica\c{c}\~ao de qualquer uma das regras, a palavra era selecionada e
adicionada a um banco de palavras. Figura 18 sintetiza o funcionamento
do script. Das 5.000 palavras mais frequentes no COCA, 1.855
apresentaram contexto em que \'e poss\'ivel haver simplifica\c{c}\~ao sil\'abica,
tais palavras, portanto, foram selecionadas para leitura.

                                [pic]

Figura 18. Fluxograma de funcionamento do script seletor de palavras.

Um microfone condensador de diafragma pequeno (1/2``), com padr\~ao polar
cardi\'oide, do tipo Superlux S241/U3 foi utilizado nas grava\c{c}\~oes. O
microfone foi ligado a uma mesa de som anal\'ogica Yamaha MG102, atrav\'es
de cabos balanceados XLR, e alimentado por corrente fantasma (48V). A
capta\c{c}\~ao do \'audio se deu por um laptop LG A51, via cabos RCA. O software
Audacity (Mazzoni \& Dannenberg, 2000) foi usado na capta\c{c}\~ao, tendo-se
ativado o filtro supressor de ru\'ido. O ambiente de grava\c{c}\~ao consistiu de
uma sala fechada, sem isolamento ac\'ustico, em situa\c{c}\~ao de baixo ru\'ido
externo.

Os dados foram segmentados atrav\'es do Adintool, de forma similar ao
m\'etodo descrito na Se\c{c}\~ao 3.4.1.1. O Appendix II re\'une os par\^ametros
utilizados na segmenta\c{c}\~ao. Os arquivos foram alinhados com sua
transcri\c{c}\~ao ortogr\'afica, manualmente. A seguir, procedeu-se à
transcri\c{c}\~ao fon\'etica das palavras, de forma tamb\'em manual. Detalhes
sobre a transcri\c{c}\~ao constam, a seguir, na Se\c{c}\~ao 3.4.2.

O software Praat (Boersma \& Weenink, 2014) foi utilizado na
transcri\c{c}\~ao, para visualizar o espectrograma e, tamb\'em, tocar os
arquivos de \'audio. O LibreOffice Calc (The Document Foundation, 2014)
foi empregado para facilitar a organiza\c{c}\~ao das transcri\c{c}\~oes. Aten\c{c}\~ao
especial foi dedicada à an\'alise da ocorr\^encia ou n\~ao do fen\^omeno de
simplifica\c{c}\~ao sil\'abica, erro alvo do prot\'otipo.

                                [pic]

  Figure 1: Workspace utilizado na transcri\c{c}\~ao fon\'etica do corpus.

4 Elabora\c{c}\~ao do modelo dicion\'ario de pron\'uncia

O modelo de pron\'uncia foi elaborado com base no CMU Pronouncing
dictionary, vers\~ao 0.7a, de 01 de abril de 2008{[}30{]}, a qual conta
com 133.315 entradas. De modo a manter a compatibilidade com o Julius,
16 palavra com s\'imbolos especiais (``\&'', ``/'', ``;'', etc.) foram
retiradas, restando-se 133.304 entradas. O dicion\'ario foi, ent\~ao,
reordenado para manter a ordem esperada pelo motor de reconhecimento.

A seguir, foram selecionadas do dicion\'ario apenas as 1.855 palavras que
apresentaram contexto poss\'ivel de haver simplifica\c{c}\~ao sil\'abica (cf.
Quadro 8). A inten\c{c}\~ao de restringir o modelo de pron\'uncia às palavras
selecionadas para grava\c{c}\~ao deu-se de maneira a diminuir a confus\~ao do
reconhecedor. Como se trata de um prot\'otipo, treinado a partir de poucas
horas de \'audio, o modelo ac\'ustico n\~ao \'e suficientemente robusto para
percorrer um espa\c{c}o de busca de 133.304 palavras e obter boa acur\'acia no
reconhecimento.

A ferramenta HDMan, do HTK Toolkit, foi empregada na elabora\c{c}\~ao do
dicion\'ario de pron\'uncia contendo as 1.855 palavras. Tal ferramenta visa
a criar novos dicion\'arios a partir de dicion\'arios fontes. Seu
funcionamento d\'a-se da seguintes forma: listas de palavras s\~ao
fornecidas como entrada, em conjunto com dicion\'arios fontes e um script
de edi\c{c}\~ao, os dados dos dicion\'arios-fontes s\~ao processados de acordo com
as op\c{c}\~oes especificadas no script e tem-se como sa\'ida um novo
dicion\'ario, o qual cont\'em as palavras fornecidas na lista, juntamente
com as respectivas pron\'uncias.

Tendo-se obtido o dicion\'ario com as 1.855 palavras utilizadas na
grava\c{c}\~ao, procedeu-se à inser\c{c}\~ao das variantes de pron\'uncia. Para esse
fim, um script em Python, com 21 regras de transcri\c{c}\~ao, foi elaborado e
aplicado ao dicion\'ario. O script percorreu cada uma das palavras,
analisando a sequ\^encia de grafemas e de fones, de modo a verificar se
havia contexto propenso à simplifica\c{c}\~ao sil\'abica. As regras foram
compostas por estruturas condicionais do tipo if\ldots{}then e seus
contextos de aplica\c{c}\~ao est\~ao descritos no Quadro 7. O conjunto de fones
utilizado na elabora\c{c}\~ao das variantes \'e constitu\'ido pela uni\~ao do
invent\'ario fon\'etico do PB segundo Crist\'ofaro-Silva (2005), e o do AmE
segundo Ogden (2012).

  Certas regras criam contexto fon\'etico para  que  outras  se  apliquem.

Por exemplo, ap\'os se aplicar regra 15, {[}Vm\#{]} \textgreater{}
{[}VNASALbi\#{]} / , \'e poss\'ivel se aplicar tamb\'em a regra 7, {[}b\#{]}
\textgreater{} {[}bi\#{]}, de forma a gerar uma nova variante a partir
de outra j\'a gerada. Aplicando-se a regra 15 à palavra ``bomb''
{[}bom{]}, seria poss\'ivel obter {[}bomb{]}, com isso, haveria contexto
para a aplica\c{c}\~ao da regra 7, gerando-se tamb\'em a variante {[}bombi{]}.
De tal maneira, as regras t\^em de ser aplicadas iterativamente, at\'e
esgotar todas as possibilidades de criar novas variantes.

  No caso do dicion\'ario de pron\'uncia do prot\'otipo, as  21  regras  foram

aplicadas cinco vezes, at\'e o n\'umero de palavras do dicion\'ario se
estabilizar, isto \'e, at\'e a aplica\c{c}\~ao das regras n\~ao adicionar nenhuma
palavra nova ao dicion\'ario. O Gr\'afico 1 descreve o crescimento do n\'umero
de palavras.

Gr\'afico 1: Crescimento do n\'umero de palavras do dicion\'ario de pron\'uncia.

                                [pic]

Como se observa, houve um aumento consider\'avel no n\'umero de palavras.
Para 1.855 palavras-base, foram geradas 5.742 variantes, de maneira que
o dicion\'ario final contabilizava 7.597 possibilidades de pron\'uncia. O
valor m\'edio de pron\'uncias por palavras foi de 4,1. Certas palavras
chegaram a apresentar at\'e 24 possibilidades de pron\'uncia, como
``employment'', ``entertainment'', ``independent'' e ``unemployment''. A
Tabela 4 resume as estat\'isticas do n\'umero de variantes por palavra.

            Tabela 4: Variantes de pron\'uncia por palavra.

                                [pic]

O grande n\'umero de possibilidades geradas para algumas palavras n\~ao era
esperado e pode trazer preju\'izos para o reconhecedor. Uma palavra
aparentemente simples, como ``combined'', apresentou 12 variantes de
pron\'uncia apenas no que diz respeito à simplifica\c{c}\~ao sil\'abica.
Considerando- se a influ\^encia da escrita na fala, seria poss\'ivel que o
aprendiz pronunciasse {[}ka{]} como {[}ko{]} e inserisse um {[}e{]} em
raz\~ao de haver na forma escrita, de tal forma, ``combined'' apresentaria
48 variantes de pron\'uncia (= 4 x 12). Tendo em vista que objetivo \'e
tratar nove tipos de erros de pron\'uncia (cf.~Se\c{c}\~ao 3.1.1), \'e poss\'ivel
que, ao se obter todas as regras, o n\'umero de variantes de pron\'uncia
geradas seja t\~ao grande que o reconhecimento seja prejudicado. Caso isso
ocorra, uma sa\'ida vi\'avel seria a cria\c{c}\~ao de dicion\'arios separados para
cada tipo de erro ou, mesmo, a utiliza\c{c}\~ao de um especialista para
cercear as variantes geradas.

\subsection{Evaluation}

Since we are proposing a method to build a pronunciation training system, we could
evaluate our method in two ways: extrinsic or intrinsically.

The extrinsic evaluation is the one which considers the purpose of the system, that is,
it assess if users are appropriately learning with the system, by improving their 
pronunciation perception and production. Therefore, the extrinsic evaluation of the
system would require the development of a longitudinal study, in which a significant 
portion of individuals would be analyzed for a large amount of time, with regular interviews
and tests to check their pronunciation skills. Given time limitations and also the scope
of this Master's thesis, an extrinsic evaluation of the system is not feasible. Instead, we 
are going to evaluate the method solely in an intrinsic way, which consists 
of assessing the the pronunciation training system itself, regardless of its practical purpose.

To put another way, our evaluation 

\subsection{Evaluation Metrics}

The \ac{WER} measures the performance of an \ac{ASR} system in terms of how
much its output (i.e. the words it recognized) diverges from a given reference. The metric
is defined as follows:
\begin{equation}
 \textit{WER}=\frac{S_w+D_w+I_w}{N_w}
\end{equation}
where $S_w$ corresponds to the number of word substitutions, $D_w$ is the number of deletions, 
$I_w$ is the number of insertions, and $N_w$ is the number of words in the sentence used
as reference, that is to say, the expected output.

The \ac{PER} analyses the system's performance in recognizing phones. It
is calculated exactly like the \ac{WER}, but considering phones as units. It is defined as:
\begin{equation}
 \textit{PER}=\frac{S_p+D_p+I_p}{N_p}
\end{equation}
where $S_p$ corresponds to the number of phone substitutions, $D_p$ is the phone deletions, 
$I_p$ is the phone insertions, and $N_p$ is the total number of reference phones in 
the transcription.

Differently fromThe Real Time Factor (\ac{RTF}) is a metric used to measure the computational performance of 
an \ac{ASR} system. It is defined as:
\begin{equation}
 \textit{RTF}=\frac{P_i}{T_i}
\end{equation}
where $P_i$ is the time it takes to process an input $i$, and $T_i$ is the duration of $i$.

The \ac{RTF} is a metric used to measure the computational performance of 
an \ac{ASR} system. It is defined as:
\begin{equation}
 \textit{RTF}=\frac{P_i}{T_i}
\end{equation}
where $P_i$ is the time it takes to process an input $i$, and $T_i$ is the duration of $i$.


Um reconhecedor de pron\'uncia pode ser avaliado de dois modos: intr\'inseca
ou extrinsecamente. A avalia\c{c}\~ao intr\'inseca (tamb\'em chamada in vitro) \'e
aquela que se at\'em à avalia\c{c}\~ao do reconhecedor em si, isolado de seu fim
pr\'atico. Em outras palavras, na avalia\c{c}\~ao intr\'inseca, o foco de
avalia\c{c}\~ao \'e a tarefa de reconhecimento, avalia-se a efici\^encia do
reconhecedor em obter, dado um sinal ac\'ustico, sua contraparte textual.

Para isso, usam-se, comumente, m\'etricas como Word Error Rate (WER),
Character Error Rate e Matrizes de Confus\~ao (Chen, Beeferman, \&
Rosenfeld, 1998; Goronzy, 2002). J\'a a avalia\c{c}\~ao extr\'inseca (tamb\'em chama
in vivo) \'e aquela se volta à avalia\c{c}\~ao do prop\'osito para o qual o
reconhecedor foi constru\'ido, no caso de um reconhecedor de pron\'uncia, o
objetivo final \'e o aprendizado de pron\'uncia pelos seus usu\'arios.
M\'etricas utilizadas para nesse tipo de avalia\c{c}\~ao s\~ao a Goodness of
Pronunciation (GOP) e a Weighted Goodness of Pronunciation (wGOP), al\'em
da verifica\c{c}\~ao do desempenho dos aprendizes em testes de profici\^encia de
l\'ingua inglesa (Witt, 1999).

Neste projeto, ser\'a realizada apenas a avalia\c{c}\~ao intr\'inseca do
reconhecedor de pron\'uncia, atrav\'es das m\'etricas Word Error Rate (WER),
Character Error Rate e Matrizes de Confus\~ao, aplicadas sobre os dados de
ambos os corpora coletados, por meio de ten-fold cross validation. A
escolha por este tipo de avalia\c{c}\~ao se deveu à natureza do projeto: como
se trata de um trabalho de mestrado, n\~ao haveria tempo h\'abil para
realizar um estudo longitudinal com aprendizes de ingl\^es, de modo a
avaliar a efici\^encia do Listener no ensino de pron\'uncia.

\section{Tools and Libraries}
Lorem ipsum quod dolor sit amet.

\subsection{HTK}
Lorem ipsum quod dolor sit amet.

\subsection{Julius}
Lorem ipsum quod dolor sit amet.

\section{Speech Corpora}
Lorem ipsum quod dolor sit amet.

\subsection{TIMIT}
Lorem ipsum quod dolor sit amet.

\subsection{WSJ0}
The CSR-I WSJ0 corpus was compiled by \citeauthor{Garofolo1993} \cite{Garofolo1993}, within the 
DARPA Spoken Language Program in order to support research on large-vocabulary \ac{CSR} systems.

It focuses on American English and contains read speech of texts drawn from a corpus with Wall 
Street Journal articles. The texts to be read were selected to fall within either a 5,000-word or 
a 20,000-word subset of the WSJ text corpus. All verbal punctuation is read out aloud and the 
prompting texts have been pre-filtered to insure unambiguous pronunciations of words. The corpus 
comprises spontaneous dictation by journalists with varying degrees of experience in dictation, 
the precise number of speakers is informed in the documentation. As for the recording environment, 
a Sennheiser close-talking head-mounted microphone was used together with a secondary microphone of 
varying types.  

\autoref{tab:wsj0-summary} describes a summary of the WSJ0 corpus.

\begin{table}[H]
\caption[Summary of the entire WSJ0 Corpus.]{Summary of the entire WSJ0 Corpus.}
\smallskip
\centering
\begin{tabular}{ccc} \toprule
  Recorded files & X \\
  Total speech time & X \\
  Average time per file & X \\
  Original format & X \\
  Number of different speakers & X \\
  \bottomrule
\end{tabular}
\label{tab:wsj0-summary}
\end{table}

For building the acoustic model, we decided to use only a portion of the WSJ0. This decision was made 
since a considerable number of files in the WSJ recordings had:

\begin{enumerate}
 \item disfluencies phenomena, such as mispronunciations, verbal deletions, false starts and spoken word fragments; 
 \item emphatic stress in words which would normally not be stressed due to lexical or syntactic factors;
 \item non-speech events (chair squeak, cross talk, door slams, paper rustle, phone ring, etc.)
 \item truncated audio files.
\end{enumerate}

All these recordings would degrade the estimation of the acoustic model, giving rise to poor phone or 
triphone \ac{HMM}s. On account of this problem, we excluded all these files from the training process.

After that, to check the consistency of the transcription, we used forced alignment with an acoustic
monophone model trained over all other English corpora. Forced alignment was performed through 
a general-purpose Viterbi recognizer, which employed beam search to find the most likely \ac{HMM} states 
for an utterance. The beam-width was set to 250. That is, for each audio file, if the transcription provided 
did not correspond to any alignment found by expanding each node over the best 250 hypotheses, 
the file was pruned and not considered for training.




The portion of the WSJ0 that we used is detailed in \autoref{tab:wsj0-used-summary}

\begin{table}[H]
\caption[Summary of WSJ0 Part We Used.]{Summary of WSJ0 Part We Used.}
\smallskip
\centering
\begin{tabular}{ccc} \toprule
  Recorded files & X \\
  Total speech time & X \\
  Average time per file & X \\
  Original format & X \\
  Number of different speakers & X \\
  \bottomrule
\end{tabular}
\label{tab:wsj0-used-summary}
\end{table}

\clearpage
\subsection{SpeechDat}
Lorem ipsum quod dolor sit amet.

\subsection{Listener's Corpus}
Lorem ipsum quod dolor sit amet.

\begin{table}[H]
\caption[Summary of Listener's Corpus.]{Summary of the Listener's Corpus.}
\smallskip
\centering
\begin{tabular}{ccc} \toprule
  Recorded files & 6,892 \\
  Total speech time & ~6.8 hours \\
  Average time per file & 1.02 seconds \\
  Original format & WAV 16kHz \\
  Number of different speakers & 53 \\
  \bottomrule
\end{tabular}
\end{table}

\clearpage
\subsection{Oxford Dictionary AmE Corpus}
The Oxford Dictionary \ac{AmE} corpus was compiled by web crawling specially to this project. Oxford University Press
has been making dictionaries for the English language for more than 150 years. Their dictionaries are very traditional and widely 
known whether in lexicographers' or laymen's circles. Recently, they made the dictionaries publicly available
on the web\footnote{\url{http://www.oxforddictionaries.com/}}. For the \ac{AmE} version, one can browse $350,000$ words, definitions, 
and entries, together with over $600,000 synonyms$\footnote{\url{http://www.oxforddictionaries.com/words/content-help}}.
A word example can be found in \autoref{fig:oxford-example}.

\begin{figure}[H]
        \myfloatalign
        {\includegraphics[width=.66\linewidth]{gfx/example-oxford-definition.png}}
        \caption{Entry example in the Oxford Dictionary online.}
        \label{fig:oxford-example}
\end{figure}

As one may notice in \autoref{fig:oxford-example}, the entry has many information: (i) the word itself, in ortographic form;
(ii) word syllabification; (iii) its pronunciation; (iv) the word audio file;
(v) \ac{POS} data; (vi) definition; (vii) some example sentences; (viii) and a list of synonyms. The pronunciation
follows Oxford's own transcription convention, which can be mapped on the \ac{IPA} as in \autoref{tab:oxford-dictionary-ipa}.

\renewcommand{\arraystretch}{0.8}% Tighter
\begin{table}[Hp]
\caption[Oxford Dictionary phone convention.]{Oxford Dictionary phone convention.}
\smallskip
\centering
\begin{tabular}{ccccc} \toprule
\tableheadline{\#} & \tableheadline{Oxford Phone} & \tableheadline{IPA Phone} & \tableheadline{Example} & \tableheadline{Transcription} \\ \midrule
1 & (h)w & \textipa{aaaaa} & when & trans \\ 
2 & \"a & \textipa{O} & hot & trans \\ 
3 & \^o & \textipa{O} & saw & trans \\ 
4 & \={oo} & \textipa{u} & too & trans \\ 
5 & \=a & \textipa{eI} & day & trans \\ 
6 & \=e & \textipa{i} & see & trans \\ 
7 & \=i & \textipa{aI} & my & trans \\ 
8 & \=o & \textipa{oU} & no & trans \\ 
9 & \textipa{@} & \textipa{@} & ago & trans \\ 
10 & \textipa{\oe} (foreign) & \textipa{\oe} & Goethe (German) & trans \\ 
11 & \textipa{\u{oo}} & \textipa{U} & put & trans \\ 
12 & \textipa{Y} (foreign) & \textipa{Y} & Utrecht (French) & trans \\ 
13 & a & \textipa{\ae} & cat & trans \\ 
14 & b & \textipa{b} & bad & trans \\ 
15 & CH & \textipa{tS} & chip  & trans \\ 
16 & d & \textipa{d} & day & trans \\ 
17 & e & \textipa{E} & bed & trans \\ 
18 & e(\textipa{@})r & \textipa{Er} & hair & trans \\ 
19 & f & \textipa{f} & fight & trans \\ 
20 & g & \textipa{g} & get & trans \\ 
21 & h & \textipa{h} & hi & trans \\ 
22 & i & \textipa{I} & sit & trans \\ 
23 & i(\textipa{@})r & \textipa{ir} & near & trans \\ 
24 & j & \textipa{dZ} & jar & trans \\ 
25 & k & \textipa{k} & kick & trans \\ 
26 & KH & \textipa{x} & loch & trans \\ 
27 & l & \textipa{l} & lie & trans \\ 
28 & m & \textipa{m} & man & trans \\ 
29 & N (foreign) & \textipa{\~v} & bon (French) & trans \\ 
30 & n & \textipa{n} & no & trans \\ 
31 & NG & \textipa{n} & ring & trans \\ 
32 & oi & \textipa{OI} & boy & trans \\ 
33 & ou & \textipa{aU} & how & trans \\ 
34 & p & \textipa{p} & pie & trans \\ 
35 & r & \textipa{r} & run & trans \\ 
36 & s & \textipa{s} & save & trans \\ 
37 & SH & \textipa{S} & she & trans \\ 
38 & t & \textipa{t} & time & trans \\ 
39 & TH & \textipa{D} & this & trans \\ 
40 & TH & \textipa{T} & thin & trans \\ 
41 & v & \textipa{v} & vow & trans \\ 
42 & y & \textipa{y} & yes & trans \\ 
43 & z & \textipa{z} & zoo & trans \\ 
44 & ZH & \textipa{Z} & decision & trans \\
\bottomrule
\end{tabular}
\label{tab:oxford-dictionary-ipa}
\end{table}
\renewcommand{\arraystretch}{1.0}% Normal

For compiling the corpus, we built a spider through Scrapy \cite{Scrapy2014}, a web crawling framework for Python specially 
designed to crawl websites and extract structured data from their pages. In total, $49,263$ entries were crawled. This number 
was defined in order to be consonant with the dictionary's legal aspects, which defines that only a fraction of its
content might be downloaded either personal or institutional use. For each word, we crawled three fields: the word, its transcription 
and the audio file. According to Oxford University Press' legal notice, we may not display or distribute any of the crawled 
content in any media, nor we may use commercially. Therefore, we used the data only for estimating the parameters of the acoustic model.

Some xenophones, i.e. phones from other languages, might be seen in \autoref{tab:oxford-dictionary-ipa}. This happens because
Oxford Dictionary also register some loanwords (specially those coming from Frech) with their original pronunciation,
like ``Utrecht'' and ``bon''. Since our goal is to deal only with Brazilian-accented English, words such as these
were excluded, so no word among the $49,263$ entries contain xenophones.

Oxford Dictionary's were recorded by many speakers (both male and female), with high-quality microphones, in sound-isolation rooms.
The audios are saved in MP3 format with \ac{VBR}, for this reason we had convert each file into WAV and downsample them to $16$ kHz.
The quality of the audio is excelent with a very high \ac{SNR}, as can be seen in the spectrogram for the word ``happiness''.

As one may observe in the regions of silence at the beginning and at the end of the utterance, the background noise approaches to zero.
A summary of the corpus can be seen in \autoref{fig:spectrogram-happiness}.

\begin{table}[H]
\caption[Summary of the Oxford Dictionary AmE Corpus.]{Summary of the Oxford Dictionary AmE Corpus.}
\smallskip
\centering
\begin{tabular}{ccc} \toprule
  Recorded files & 49,263 \\
  Total speech time & ~4 hours \\
  Average time per file & 1.02 seconds \\
  Original format & MP3 (VBR) \\
  Number of different speakers & Unknown \\
  \bottomrule
\end{tabular}
\label{tab:oxford-summary}
\end{table}

\clearpage
\subsection{Cambridge Dictionary}
Lorem ipsum quod dolor sit amet.

\subsection{OGI-22}
Lorem ipsum quod dolor sit amet.

\subsection{Westpoint}
Lorem ipsum quod dolor sit amet.

\subsection{LapsBM}
Lorem ipsum quod dolor sit amet.

\subsection{Youtube}
Lorem ipsum quod dolor sit amet.

\begin{table}[H]
\caption[Summary of Listener's Corpus.]{Summary of the Listener's Corpus.}
\smallskip
\centering
\begin{tabular}{ccc} \toprule
  Recorded files & 6,892 \\
  Total speech time & ~6.8 hours \\
  Average time per file & 1.02 seconds \\
  Original format & WAV 16kHz \\
  Number of different speakers & 53 \\
  \bottomrule
\end{tabular}
\end{table}


\clearpage
\section{Building the Acoustic Model}

\subsection{The Phoneset}\label{sec:listener-phoneset}

Since our goal is to build an \ac{ASR} system capable of recognizing non-native speech, we propose
to use an interlingual phoneset as the basis of the pronunciation model. By doing this, we can define
\ac{HMM} models for estimating phones which are part of both the speaker's native language (\ac{BP}) and 
the target language (\ac{AmE}).

A straightforward approach would be to look up the literature in phonetics and phonology in order to find
a \ac{BP}--\ac{AmE} interlingual phoneset. However, to the best of our knowledge, no previous works were carried 
out in this regard. There are papers addressing specific mispronunciations, such as those discussed in \autoref{sec:common-mispronunciation}, 
which list some occurring interlingual phones, but there is not a wide-ranging study available about this matter.

Therein we had to develop our own interlingual phoneset. For simplicity, we decided to adapt the union set formed 
by the phones contained in two machine-readable dictionaries, one for \ac{AmE}: \ac{CMUdict} \cite{CMU2008}, and another for \ac{BP}: 
Aeiouad\^o \cite{Mendonca2014}. By doing this, we can cover most of the phone productions that brazilian \ac{ESL} learners are likely
to make. Advanced students will tend to use more properly the English phones, whereas beginners will have a stronger accent,
thus producing more phones from the \ac{BP} phoneset. A brief description of each of these dictionaries is given below, before 
we go into further details of our interlingual set.

\ac{CMUdict} \citep{CMU2008} is a machine-readable pronunciation dictionary for \ac{AmE} which has about $125,000$ 
words and their transcriptions. It was designed primarily for speech applications, such as speech recognition and synthesis, 
and it has been widely tested in both Academia and industry.

Words in \ac{CMUdict} are transcribed using ARPAbet, a phonetic transcription code developed by the Advanced Research Projects Agency in 
$1971$. It represents each \ac{AmE} phone with a distinct sequence of \ac{ASCII} characters. In the dictionary, there are in total $39$ 
phones plus stress marks. The phone convention is described on \autoref{tab:cmu-conv}.

\renewcommand{\arraystretch}{0.95}% Tighter
\begin{table}[p]
\caption[CMUdict phone convention.]{CMUdict phone convention.}
\smallskip
\centering
\begin{tabular}{ccccc} \toprule
\tableheadline{\#} & \tableheadline{CMU Phone} & \tableheadline{IPA Phone} & \tableheadline{Example} & \tableheadline{Transcription} \\ \midrule
1 & AA & [\textipa{A}] & odd & AA D \\
2 & AE & [\textipa{\ae}] & at & AE T \\
3 & AH & [\textipa{@}] & hut & HH AH T \\
4 & AO & [\textipa{O}] & ought & AO T \\
5 & AW & [\textipa{aU}] & cow & K AW \\
6 & AY & [\textipa{aI}] & hide & HH AY D \\
7 & B & [\textipa{b}] & be & B IY \\
8 & CH & [\textipa{tS}] & cheese & CH IY Z \\
9 & D & [\textipa{d}] & dee & D IY \\
10 & DH & [\textipa{D}] & thee & DH IY \\
11 & EH & [\textipa{E}] & Ed & EH D \\
12 & ER & [\textipa{@r}] & hurt & HH ER T \\
13 & EY & [\textipa{A\*r}] & ate & EY T \\
14 & F & [\textipa{f}] & fee & F IY \\
15 & G & [\textipa{g}] & green & G R IY N \\
16 & HH & [\textipa{h}] & he & HH IY \\
17 & IH & [\textipa{I}] & it & IH T \\
18 & IY & [\textipa{i}] & eat & IY T \\
19 & JH & [\textipa{dZ}] & gee & JH IY \\
20 & K & [\textipa{k}] & key & K IY \\
21 & L & [\textipa{l}] & lee & L IY \\
22 & M & [\textipa{m}] & me & M IY \\
23 & N & [\textipa{n}] & knee & N IY \\
24 & NG & [\textipa{N}] & ping & P IH NG \\
25 & OW & [\textipa{oU}] & oat & OW T \\
26 & OY & [\textipa{OI}] & toy & T OY \\
27 & P & [\textipa{p}] & pee & P IY \\
28 & R & [\textipa{\*r}] & read & R IY D \\
29 & S & [\textipa{s}] & sea & S IY \\
30 & SH & [\textipa{S}] & she & SH IY \\
31 & T & [\textipa{t}] & tea & T IY \\
32 & TH & [\textipa{T}] & theta & TH EY T AH \\
33 & UH & [\textipa{U}] & hood & HH UH D \\
34 & UW & [\textipa{u}] & two & T UW \\
35 & V & [\textipa{v}] & vee & V IY \\
36 & W & [\textipa{w}] & we & W IY \\
37 & Y & [\textipa{y}] & yield & Y IY L D \\
38 & Z & [\textipa{z}] & zee & Z IY \\
39 & ZH & [\textipa{Z}] & seizure & S IY ZH ER \\
\bottomrule
\end{tabular}
\label{tab:cmu-conv}
\end{table}
\renewcommand{\arraystretch}{1.0}% Normal

As for Aeiouad\^o, as described in Section~\autoref{sec:aeiouado}, its transcriptions are
based on the dialect of S\~ao Paulo city and contains 39 different phones. The dictionary makes
use of a hybrid approach for converting graphemes into phonemes, 
which employs both manual transcription rules and machine learning algorithms. Its phone convention
is presented in Table~\autoref{tab:aeiouado-conv}.

\renewcommand{\arraystretch}{0.9}% Tighter
\begin{table}[p]
\caption[Aeiouad\^o phone convention.]{Aeiouad\^o phone convention.}
\smallskip
\centering
\begin{tabular}{ccccc} \toprule
\tableheadline{\#} & \tableheadline{Aeiouad\^o Phone} & \tableheadline{IPA Phone} & \tableheadline{Example} & \tableheadline{Transcription} \\ \midrule
1 & a & [\textipa{a}] & amor & a m o x \\
2 & a$\sim$ & [\textipa{\~a}] & canto & k a$\sim$ t U \\
3 & b & [\textipa{b}] & besta & b e s t @ \\
4 & d & [\textipa{d}] & da & d a \\
5 & dZ & [\textipa{dZ}] & dia & dZ i @ \\
6 & E & [\textipa{E}] & \'e & E \\
7 & e & [\textipa{e}] & dedo & d e d U \\
8 & e$\sim$ & [\textipa{\~e}] & venda & v e$\sim$ d @ \\
9 & f & [\textipa{f}] & frio & f 4 i U \\
10 & g & [\textipa{g}] & gula & g u l @  \\
11 & G & [\textipa{G}] & carga & carga \\
12 & i & [\textipa{i}] & a\'i & a i \\
13 & I & [\textipa{I}] & come & k o$\sim$ m I \\
14 & i$\sim$ & [\textipa{\~i}] & sim & s i$\sim$ \\
15 & J & [\textipa{\textltailn}] & ganho & g a$\sim$ J U \\
16 & j & [\textipa{y}] & pai & p a j \\
17 & j$\sim$ & [\textipa{\~y}] & parem & p a 4 e$\sim$ j$\sim$ \\
18 & k & [\textipa{k}] & compra & k o$\sim$ p 4 @ \\
19 & l & [\textipa{l}] & l\'a & l a \\
20 & L & [\textipa{L}] & palha & p a L @ \\
21 & m & [\textipa{m}] & m\~ae & m a$\sim$ j$\sim$ \\
22 & n & [\textipa{n}] & n\~ao & n a$\sim$ w$\sim$ \\
23 & O & [\textipa{O}] & p\'o & p O \\
24 & o & [\textipa{o}] & gorro & g o x U \\
25 & o$\sim$ & [\textipa{\~o}] & com & k o$\sim$ \\
26 & p & [\textipa{p}] & pessoa & p e s o @ \\
27 & s & [\textipa{s}] & susto & s u s t U \\
28 & s & [\textipa{S}] & chato & S a t U \\
29 & t & [\textipa{t}] & tato & t a t U \\
30 & tS & [\textipa{tS}] & noite & n o j tS I \\
31 & u & [\textipa{u}] & durmo & d u G m U \\
32 & U & [\textipa{U}] & c\'umulo & k u m u l U \\
33 & u$\sim$ & [\textipa{\~u}] & um & u$\sim$ \\
34 & v & [\textipa{v}] & vida & v i d @ \\
35 & w & [\textipa{w}] & aula & a w l @ \\
36 & w$\sim$ & [\textipa{\~w}] & canh\~ao & k a$\sim$ J a$\sim$ w$\sim$ \\
37 & x & [\textipa{x}] & rato & x a t U \\
38 & z & [\textipa{z}] & zebra & z e b 4 @ \\
39 & 4 & [\textipa{R}] & arara & a 4 a 4 @ \\
40 & @ & [\textipa{@}] & bola & b O l @ \\
\bottomrule
\end{tabular}
\label{tab:aeiouado-conv}
\end{table}
\renewcommand{\arraystretch}{1.0}% Normal

In what concerns to our interlingual phoneset, we kept all \ac{CMUdict} phones, since that is the target language's
phoneset the learners are trying to achieve. Then we compared each phone in \ac{CMUdict} with those contained in 
Aieouad\^o, in order to check for missing phones and to analyze whether the overlapping ones really correspond to the same
sound.

At a first glance, it seems that \ac{BP} and \ac{AmE} share a pool of 24 common phones, comprising fifteen consonantes
[\textipa{b}, \textipa{d}, \textipa{dZ}, \textipa{f}, \textipa{g}, 
\textipa{k}, \textipa{l}, \textipa{m}, \textipa{n}, \textipa{p}, \textipa{s}, \textipa{S}, \textipa{t}, \textipa{tS}, \textipa{v}, \textipa{z}];
seven vowels [\textipa{E}, \textipa{i}, \textipa{I}, \textipa{O}, \textipa{u}, \textipa{U}, \textipa{@}];
together with two glides [\textipa{y}, \textipa{w}]. However it is worth noticing that this first impression does not
hold true. 

Despite the fact that both dictionaries present \ac{IPA} correspondences, such correspondes should not be taken
for granted, without previous analysis. In theory, \ac{IPA} is capable of describing any sound produced by the human vocal 
tract with exactness. Still \ac{IPA} transcriptions are biased by the level of detail one wishes to express and by the 
assumptions of the transcriber. This is the case for some of these overlapping phones. Several of these consonants and vowels, 
although marked with the same \ac{IPA} symbol by \ac{CMUdict} and Aeiouad\^o,
in fact, can not be regarded as being the same sound, since they show a very different distribution
in English and \ac{BP}. 

For instance, the production of /\textipa{p}, \textipa{t}, \textipa{k}/, 
in \ac{BP} and \ac{AmE} can be quite different. In English, such consonants are generally produced as 
[\textipa{p}, \textipa{t}, \textipa{k}]. However it is known that when they occur in certain contexts, 
for example, in word initial position or onset of a stressed syllable, they become aspirated; whence
[\textipa{p\super h}, \textipa{t\super h}, \textipa{k\super h}] \citep{Lisker1985}. 
On the other hand, this process is not found in \ac{BP}, where, disregard of the context, [\textipa{p}, \textipa{t}, \textipa{k}] show no 
relevant levels of aspiration \citep{Klein1999}. 

For that reason, in order to properly estimate the \ac{HMM} states for the interlingual phones, it is mandatory
to create aspirated phone models for these consonants that are different from the non-aspirated ones. In our interlingual
phoneset, we decided to keep the distinction between aspirated and non-aspirated phones. Therefore, according to our
convention, the /\textipa{p}/ that occurs in a stressed syllable of an English word like ``pie'' is transcribed as [\textipa{p\super h}], 
while the /\textipa{p}/ of an unstressed syllable or a BP word is transcribed as [\textipa{p}]. 

Additionally, some vowels that are described by both dictionaries with the same \ac{IPA} symbol are not exactly equal.
Although English and \ac{BP} both possess the vowels [\textipa{I, @, U}], the distribution
of these phones between both languages is fairly different. In \ac{AmE} such vowels hold phonological status, i.e. they 
are phonemes, so that one could find minimal pairs differing only by [\textipa{I, @, U}], such as ``sheep'' [\textipa{"Sip}] vs. 
``ship'' [\textipa{"SIp}]; ``cut'' [\textipa{"k\super h@t}] vs. ``cat'' [\textipa{"k\super h\ae t}]; and ``pull'' 
[\textipa{"p\super hUl}] vs. pool [\textipa{"p\super hul}]. As for \ac{BP}, these vowels exist solely as part
of a phonological process. When the tense vowels [\textipa{i, a, u}] occur in unstressed word-final position, they undergo a lenition process
and are produced as the lax vowels [\textipa{I, @, U}], respectivelly. Hence the \ac{BP} [\textipa{I, @, U}] have 
different formant values \cite{Fails1992} and all the typical characteristics of lax vowels, that is, they are short, 
they have less energy and, consequently, less clear-cut formants \cite{Nobre1987}.

With regard to the missing phones, there were 15 phones which are only present in the \ac{BP} inventory, these include eight vowels
[\textipa{a}, \textipa{\~a}, \textipa{e}, \textipa{\~e}, \textipa{\~i}, \textipa{o}, \textipa{\~o}, \textipa{\~u}]; five
consonants [\textipa{G}, \textipa{\textltailn}, \textipa{L}, \textipa{x}, \textipa{R}]; and two nasal glides
[\textipa{\~y}, \textipa{\~w}]. 

All eight vowels were added to the interlingual phoneset, for they encompass negative transfer problems. For instance, 
[\textipa{\~a}, \textipa{\~e}, \textipa{\~i}, \textipa{\~o}, \textipa{\~u}] are related to
vocalization of final nasals (\emph{vide} \ref{sec:voc-nasals}) and [\textipa{a}, \textipa{e}, \textipa{o}] to vowel assimilation (\emph{vide} \ref{sec:voc-assimilation}).

The \ac{BP} rhotic consonants [\textipa{R}, \textipa{G}, \textipa{x}], were merged onto the same sound, [\textipa{h}] owing to the fact
they they represent \ac{BP} dialectal variants not relevant to L1-L2 interphonology.

Furthermore, the \ac{BP} palatal consonants [\textipa{\textltailn}, \textipa{L}] were not considered in the final interlingual phoneset,
since we could not find, in the literature, any negative transfer process in which they occur. 

The nasal glides were excluded from the interlingual phoneset, instead we preffered to combine them with their accompanying vowels, in
order to create nasal diphthongs, such as [\textipa{\~a\~I}, \textipa{\~a\~U}, \textipa{\~e\~I}, \textipa{\~o\~I}]. This decision was 
made in accordance with \cite{Demasi2010}, which found that \ac{BP} nasal diphthongs have a very particular behavior, with 
articulatory, acoustic and aerodynamic patterns different from the non-nasalized ones. 
We believe that a single \ac{HMM} model for each diphthong will be able to better gauge this behavior.

The final phoneset, which we used as the basis for our \ac{ASR} system, can be found in \autoref{tab:interlingual-conv}.

\renewcommand{\arraystretch}{0.6}% Tighter

\begin{table}[p]
\caption[Interlingual dictionary phone convention.]{Interlingual dictionary phone convention. BP word examples are shown in \emph{italics}.}
\smallskip
\centering
\begin{tabular}{ccccc} \toprule
\tableheadline{\#} & \tableheadline{Interl. Phone} & \tableheadline{IPA Phone} & \tableheadline{Example} & \tableheadline{Transcription} \\ \midrule
 1 & \small a & \small  [\textipa{@}] & \small \emph{da} & \small d a \\ 
\small 2 & \small aa & \small  [\textipa{A}] & \small odd & \small aa d \\ 
\small 3 & \small aaa & \small  [\textipa{a}] & \small \emph{d\'a} & \small d aaa \\ 
\small 4 & \small ae & \small  [\textipa{\ae}] & \small cat & \small k ae t \\ 
\small 5 & \small ah & \small  [\textipa{@}] & \small but & \small b ah t \\ 
\small 6 & \small ahw & \small  [\textipa{@U}] & \small xxx & \small xx xx \\ 
\small 7 & \small am & \small  [\textipa{\~a}] & \small xxx & \small xx xx \\ 
\small 8 & \small ao & \small  [\textipa{O}] & \small for & \small f ao r \\ 
\small 9 & \small aow & \small [\textipa{OU}] & \small xxx & \small xx xx \\ 
\small 10 & \small aw & \small [\textipa{aU}] & \small cow & \small k aw \\ 
\small 11 & \small awm & \small [\textipa{\~a\~U}] & \small \emph{n\~ao} & \small n awm \\ 
\small 12 & \small ay & \small [\textipa{aI}] & \small I & \small ay \\ 
\small 13 & \small aym & \small [\textipa{\~a\~I}] & \small m\~ae & \small m aym \\ 
\small 14 & \small b & \small [\textipa{b}] & \small boot & \small b uw tt \\ 
\small 15 & \small ch & \small [\textipa{tS}] & \small cheek & \small ch iy kk \\ 
\small 16 & \small d & \small [\textipa{d}] & \small do & \small d uw \\ 
\small 17 & \small dh & \small [\textipa{D}] & \small that & \small dh ae tt \\ 
\small 18 & \small e & \small [\textipa{e}] & \small \emph{eu} & \small e w \\ 
\small 19 & \small eh  & \small [\textipa{E}] & \small merry & \small m eh r iy \\ 
\small 20 & \small em & \small [\textipa{\~e}] & \small \emph{entendi} & \small em tt em jh i \\ 
\small 21 & \small ey & \small [\textipa{eI}] & \small April & \small ey pp r iy ll \\ 
\small 22 & \small eym & \small [\textipa{\~e\~I}] & \small \emph{hein} & \small eym \\ 
\small 23 & \small f & \small [\textipa{f}] & \small fat & \small f ae tt \\ 
\small 24 & \small g & \small [\textipa{g}] & \small guy & \small g ay \\ 
\small 25 & \small hh & \small [\textipa{h}] & \small heat & \small hh iy tt  \\ 
\small 26 & \small ih & \small [\textipa{I}] & \small bit & \small b ih tt \\ 
\small 27 & \small i & \small [\textipa{I}] & \small \emph{comi} & \small k o m i \\ 
\small 28 & \small im & \small [\textipa{\~i}] & \small \emph{sim} & \small s im \\ 
\small 29 & \small iy & \small [\textipa{i}] & \small eat & \small iy tt \\ 
\small 30 & \small jh & \small [\textipa{dZ}] & \small judge & \small jh ah jh \\ 
\small 31 & \small k & \small [\textipa{k\super h}] & \small cool & \small k uw ll \\ 
\small 32 & \small kk & \small [\textipa{k}] & \small cai & \small kk ay \\ 
\small 33 & \small l & \small [\textipa{l}] & \small lounge & \small l aa uh n jh \\ 
\small 34 & \small ll & \small [\textipa{l\super G}] & \small fall & \small f ao ll \\ 
\small 35 & \small m & \small [\textipa{m}] & \small mother & \small m ah dh ah r \\ 
\small 36 & \small n & \small [\textipa{n}] & \small neat & \small n iy tt \\ 
\small 37 & \small ng & \small [\textipa{N}] & \small king & \small k ih ng \\ 
\small 38 & \small o & \small [\textipa{o}] & \small s\^o & \small s o \\ 
\small 39 & \small om & \small [\textipa{\~o}] & \small \emph{conto} & \small kk om tt u\\ 
\small 40 & \small ow & \small [\textipa{oU}] & \small no & \small n ow \\ 
\small 41 & \small ohy & \small [\textipa{OI}] & \small boys & \small b oy z\\ 
\small 42 & \small oym & \small [\textipa{\~o\~I}] & \small \emph{doa\c{c}\~oes} & \small d o a s oym s\\ 
\small 43 & \small p & \small [\textipa{p\super h}] & \small pity & \small p ih rd iy \\ 
\small 44 & \small pp & \small [\textipa{p}] & \small \emph{pai} & \small pp ay \\ 
\small 45 & \small r & \small [\textipa{r}] & \small run & \small r ah n \\ 
\small 46 & \small rd & \small [\textipa{R}] & \small city & \small s ih rd iy \\ 
\small 47 & \small s & \small [\textipa{s}] & \small six & \small s ih k s \\ 
\small 48 & \small sh & \small [\textipa{S}] & \small shoes & \small sh uw z \\ 
\small 49 & \small t & \small [\textipa{t\super h}] & \small time & \small t ay m \\ 
\small 50 & \small th & \small [\textipa{T}] & \small three & \small th r iy \\ 
\small 51 & \small tt & \small [\textipa{t}] & \small \emph{tudo} & \small tt uw d u \\ 
\small 52 & \small u & \small [\textipa{U}]  & \small \emph{como} & \small kk om m u \\ 
\small 53 & \small uh & \small [\textipa{U}] & \small could & \small k uw d \\ 
\small 54 & \small um & \small [\textipa{\~u}] & \small \emph{rum} & \small hh um \\ 
\small 55 & \small uw & \small [\textipa{u}] & \small wood & \small w uw d \\ 
\small 56 & \small v & \small [\textipa{v}] & \small van & \small v ae n \\ 
\small 57 & \small w & \small [\textipa{w}] & \small what & \small w aa tt \\ 
\small 58 & \small y & \small [\textipa{y}] & \small union & \small y uw n y ah n \\ 
\small 59 & \small z & \small [\textipa{z}] & \small zoo & \small z uw \\ 
\small 60 & \small zh & \small [\textipa{Z}] & \small leisure & \small l eh zh ah r \\ 
\bottomrule
\end{tabular}
\label{tab:interlingual-conv}
\end{table}
\renewcommand{\arraystretch}{1.0}% Normal

\clearpage
\subsection{Speech Data}

Many of the 



\subsection{HMM topology}

\subsection{Tree-Based State Tying}

Data-driven approaches also show limitations. Since such approaches are generally based on 
the positive examples which occur in a corpus, rare or non-occuring phenomena are often poorly estimated or even neglected.
This is the case for triphone \ac{HMM} models. 

Natural languages have, on average, $30$ different phones. The language believed to have the smallest phonetic inventory is 
Rotokas (East Papuan, New Guinea), with 11 phones, and the one with largest is !X\'o\~o  (Khoisan, Botswana/Namibia), 
with 160 \citep{Hayes2011}. English is usually assumed to have 37 to 41 phones, depending on the dialect. 

In the \ac{CMUdict} \citep{CMU2008}, 39 phones are used to describe the words of \ac{AmE}. When it comes to triphones, in theory, 
this number might grow by three orders of magnitude, that is $39^3$ or $59,319$. It is true that due to phonotactic constraints
many of the virtually possible triphones never take place in practice. 
For instance, the triphone sequence [\textipa{N}-\textipa{s}+\textipa{p}] does not exist in English, although it 
is made of valid and existing monophones [\textipa{N}], [\textipa{s}] and [\textipa{p}]. 

\citeauthor{Kuperman2008} \citep{Kuperman2008} examined the monophone, diphone and triphone frequencies in speech corpora
for many languages. In what concerns to English, they analysed the Buckeye Corpus and reported that it contains $29,804$ different 
occurring triphones (or types), distributed among $431,000$ tokens. Although the actual number of triphones for English is
almost half the number of possible permutations, it is still a huge number of triphones to model over a corpus. 
Therefore, in order to build an \ac{ASR} system, one always has to deal with data scarcity and try to overcome its limitations.

Within \ac{HMM} \ac{ASR}, tree-based state tying is a technique to improve the modelling of rare triphones and allow the 
estimation of non-occurring ones. It was initially proposed by \citeauthor{Young1994} \citep{Young1994} and since then
has become a standard procedure in \ac{HMM} \ac{ASR} systems. The aim of tree-based state tying is to maintain the balance 
between the model complexity and the available training data by tying the \ac{HMM} states of acoustically similar triphones.

In contrast to the majority of methods in \ac{ASR}, tree-based state tying is carried out in a top-down, knowledge-based 
way. A specialist (generally a speech sciencist, phologist or phoneticist) uses his/her knowledge to build a phonetic decision 
tree which will organize the phones into sets with similar acoustic parameters. 

In most cases, the criteria used to organize phones into a decision tree are the so-called natural classes. 
Natural classes were proposed within the generative phonology framework. In this framework, phones and and phonemes are no 
longer considered the basic units of analysis, instead it is assumed that they can be broken down into smaller components,
which describe aspects of articulation and perception, such as [+nasal], [-continuant], [+strident], etc. For instance,
the phone [\textipa{s}] would not be represented in generative phonology as a single phonetic symbol [\textipa{s}], but 
as a bundle of distinctive features \citep{Jensen2004}:
\[
[{\textipa{a}}] \rightarrow 
\begin{bmatrix}
+consonantal \\ -syllabic \\ -sonorant \\ +continuant \\ +anterior \\ +coronal \\ +strident \\ -voiced
\end{bmatrix}
\]

Distinctive features were probably the most important contribution of generative phonology. They became popular as a
model specially because they were able of simplify phonological processes by grouping segments. Consider the case of
final-obstruent devoicing. In many pronunciations of Standard German, voiced obstruent consonants become devoiced when
they occur in word final position. This is the case for a large number of german nouns which make their plural by the 
addition of the suffix \{-\textipa{e}\}. \autoref{tab:german-devoicing} contains a few examples extracted from 
\citeauthor{Grijzenhout2000} \citep{Grijzenhout2000}.

\begin{table}[H]
\caption[Examples of final-obstruent devoicing in German.]{Examples of final-obstruent devoicing in German.}
\smallskip
\centering
\begin{tabular}{ccc} \toprule
  \tableheadline{Plural form} & \tableheadline{Singular form} & \tableheadline{Gloss} \\ \midrule
  Hun[\textipa{d}]e & Hun[\textipa{t}] & \small{dog} \\
  Die[\textipa{b}]e & Die[\textipa{p}] & \small{thief} \\
  Ber[\textipa{g}]e & Ber[\textipa{k}] & \small{mountain} \\
  M\"au[\textipa{z}] & Mau[\textipa{s}] & \small{mouse} \\
  \bottomrule
\end{tabular}
\label{tab:german-devoicing}
\end{table}

As one can observe at the provided examples, the plural forms contain only voiced obstruents whilst the singular forms contain their 
devoiced counterparts. To express this phonological process within structural phonology, one would have to introduce at
least four rules:
\[
[\textipa{b}] \rightarrow [\textipa{p}]  / \_ \# 
\]
\[
[\textipa{d}] \rightarrow [\textipa{t}]  / \_ \#
\]
\[
[\textipa{g}] \rightarrow [\textipa{k}]  / \_ \#
\]
\[
[\textipa{d}] \rightarrow [\textipa{s}]  / \_ \#
\]

Whereas using distinctive features, all cases of german final-obstruent devoicing can be explained by a single rule:
\[
\begin{bmatrix}
+consonantal \\ -syllabic \\ -sonorant \\ +voiced
\end{bmatrix} \rightarrow 
\begin{bmatrix}
-voiced
\end{bmatrix}
 / \_ \#
\]

The main benefits of distinctive features lie in their capability of making generalizations, that is to say
of grouping phones together in a meningful way.  Phonologists have long known that sounds that share the same manner, 
place of articulation, or voicing level behave similarly. Distinctive features provide a way to express this
in an elegant way, by stablishing that feature matrices which are not fully specified do form a natural class. 
Therefore a distinctive features matrix
\[
\begin{bmatrix}
+consonantal \\ -syllabic
\end{bmatrix}
\]
represents all consonants, whereas
\[
\begin{bmatrix}
+consonantal \\ -syllabic \\ +continuant \\ +voiced
\end{bmatrix}
\]
represents all voiced fricatives, etc. For developing the phonetic decision tree, the specialist employs his/her 
knowledge of phonetics and phonology in order to define questions about the phonetic environment of a given 
phone. This phonetic environment is defined by using natural classes, such that an example question would
be ``is the phone preceded by an obstruent?'' or ``is there a fricative after the phone?'' Technically, the tree is a 
binary one, i.e a connected acyclic graph such that the degree of each vertex is no more than three. Each
internal node of the tree represents a question about the phonetic context a triphone, each branch represents the answer in a yes-no
form and leaf nodes define \ac{HMM} states. Once the tree is built, its structure is used to decide how \ac{HMM} states will
be tied among triphone models. \autoref{fig:decision-tree} presents an example of a phonetic decision tree.

\begin{figure}[H]
        \myfloatalign
        {\includegraphics[width=.66\linewidth]{gfx/decision-tree.png}}
        \caption{Phonetic decision tree for HMM state tying \citep{Young1994}.}
        \label{fig:decision-tree}
\end{figure}

The tree input is each triphone being analysed.

The clustering procedures begins by placing all observations in a single root node. All questions are 
analised and the one which maximise the likelihood of a single diagonal
covariance Gaussian is chosen, then the node is split and child nodes generated. The splitting goes on
until it falls below a threshold. Commonly, a minimum threshold is also set, the usual approach is to consider
the frequency of occurence of the phones in a corpus. That it, if a given phone $p$ has $m$ samples and the minimum
threshold is $m$, such that $m>n$, the phone $p$ is mapped unto a more general and robust phone. Consider
that the triphone \textipa{[I-N+g]} appeared 30 times on corpus, and the minimum threshold for splitting the node
was $45$, then \textipa{[I-N+g]} would be modeled into a more general phone, say, for instance, [\textipa{i-N+g}] or 
[\textipa{i-N}].

In the example shown in \autoref{fig:decision-tree}, the first node 
(the root of the tree) checks if the left part of the triphone contains a nasal consonant. If positive, the tree 
examines the right side of the triphone, by questioning
whether it is a liquid consonant. If negative, a leaf is reached and the HMM state to be tied is outputted, e.g. ``tie
the 1\textsuperscript{st} emitting HMM state of the analysed triphone $A$ to [Nasal-$A$+*]''. 
\autoref{fig:state-tying-tree} summarizes the state tying procedure.

\begin{figure}[H]
        \myfloatalign
        {\includegraphics[width=.66\linewidth]{gfx/state-tying-tree.png}}
        \caption{Tied-state HMM system build procedure \citep{Young1994}.}
        \label{fig:state-tying-tree}
\end{figure}

For building the phonetic decision tree for Listener, we based ourselves on the distinctive feature set proposed by 
\citeauthor{Jensen2004} \citep{Jensen2004}, which follows the main guidelines of generative phonology by dividing
the features into four central classes: i) major class features, ii) manner features; iii) place features; and 
iv) laryngeal features.

In practice, classes work as following. Major class features distinguish among the most general classes of sound, 
i.e. vowels, consonants and glides (or semi-vowels). Manner features determine how sounds are articulated, 
that is do they consist of a stop, a nasal, a fricative, a liquid, a trill or a flap? Place features describe 
where in the mouth sounds are produced, whether in the labial region, the alveolar region, the velar region, etc.
Finally, laryngeal features represent the glottal states of sounds, their basic purpose is to differ voiced 
sounds from unvoiced ones.

\citeauthor{Jensen2004} \citep{Jensen2004} proposes a set of $17$ features to describe most of the world languages.

\begin{enumerate}
 \item \emph{Major class features}: syllabic, consonantal and sonorant;
 \item \emph{Manner features}: continuant, nasal, lateral, strident and delayed release;
 \item \emph{Place features}: anterior, coronal, distributed, high, low, back, round and \ac{ATR}.
 \item \emph{Laryngeal features}: voice and \ac{HSP};
\end{enumerate}

The somewhat reduced set for place features is capable of representing a large amount of places of articulation 
since features which are regularly restricted to vowels (such as high, low, back and round) are also shared 
with consonants. Besides, once features have a binary nature, therefore, in theory, a number $n$ of features is able
to dintinguish up to $2^n$ phones. In \autoref{fig:features-place}, a comparison is shown between place features and their corresponding
regions of articulation. For a full explanation of each feature, please see \citeauthor{Jensen2004} \citep{Jensen2004}.

\begin{figure}[H]
        \myfloatalign
        {\includegraphics[width=.66\linewidth]{gfx/features-place.png}}
        \caption{Distinctive features for places of places of articulation \citep{Jensen2004}.}
        \label{fig:features-place}
\end{figure}

According to this set of distinctive features, we can arrange the entire phonetic inventory of \ac{AmE} as in \autoref{tab:dist-features-eng}
and that of \ac{BP} as in \autoref{tab:dist-features-bp}.

\tabcolsep=0.15cm
\begin{table}[htbp]
\caption{Distinctive features chart for AmE phones \citep{Jensen2004}.}
\begin{center}
\begin{tabular}{|cc|cccccccccccccccccc|}\hline
 &  & \multicolumn{18}{c|}{\textsc{Distinctive features}} \\ \cline{3-20}
\rotatebox[origin=c]{90}{\textsc{\textsc{CMU symbol}}} & \rotatebox[origin=c]{90}{\textsc{\textsc{IPA symbol}}} & \rotatebox[origin=c]{90}{\textsc{\textsc{syllabic}}} & \rotatebox[origin=c]{90}{\textsc{consonantal}} & \rotatebox[origin=c]{90}{\textsc{sonorant}} & \rotatebox[origin=c]{90}{\textsc{voice}} & \rotatebox[origin=c]{90}{\textsc{HSP}} & \rotatebox[origin=c]{90}{\textsc{continuant}} & \rotatebox[origin=c]{90}{\textsc{nasal}} & \rotatebox[origin=c]{90}{\textsc{lateral}} & \rotatebox[origin=c]{90}{\textsc{strident}} & \rotatebox[origin=c]{90}{\textsc{del. release}} & \rotatebox[origin=c]{90}{\textsc{anterior}} & \rotatebox[origin=c]{90}{\textsc{coronal}} & \rotatebox[origin=c]{90}{\textsc{  distributed  }} & \rotatebox[origin=c]{90}{\textsc{high}} & \rotatebox[origin=c]{90}{\textsc{low}} & \rotatebox[origin=c]{90}{\textsc{back}} & \rotatebox[origin=c]{90}{\textsc{ATR}} & \rotatebox[origin=c]{90}{\textsc{round}} \\ \hline
AA & [\textipa{A}] & + & - & + & + & - & + & - & - & - & - & - & - & - & - & + & + & + & - \\[-3.5pt]
AE & [\textipa{ae}] & + & - & + & + & - & + & - & - & - & - & - & - & - & - & + & - & - & - \\[-3.5pt]
AH & [\textipa{@}] & + & - & + & + & - & + & - & - & - & - & - & - & - & - & - & + & - & - \\[-3.5pt]
AO & [\textipa{O}] & + & - & + & + & - & + & - & - & - & - & - & - & - & - & + & + & - & + \\[-3.5pt]
AW & [\textipa{aU}] & + & - & + & + & - & + & - & - & - & - & - & - & - & - & - & - & - & - \\[-2pt] \hline
AY & [\textipa{aI}] & + & - & + & + & - & + & - & - & - & - & - & - & - & - & - & - & - & - \\[-3.5pt]
B & [\textipa{b}] & - & + & - & + & - & - & - & - & - & - & + & - & - & - & - & - & - & - \\[-3.5pt] 
CH & [\textipa{tS}] & - & + & - & - & - & - & - & - & + & + & - & - & - & - & - & - & - & - \\[-3.5pt] 
D & [\textipa{d}] & - & + & - & + & - & - & - & - & - & - & + & + & - & - & - & - & - & - \\[-3.5pt] 
DH & [\textipa{D}] & - & + & - & + & - & + & - & - & - & - & + & + & - & - & - & - & - & - \\[-2pt] \hline
EH & [\textipa{E}] & + & - & - & + & - & + & - & - & - & - & - & - & - & - & - & - & - & - \\[-3.5pt] 
EY & [\textipa{A\*r}] & + & - & + & + & - & + & - & - & - & - & - & - & - & - & - & - & + & - \\[-3.5pt] 
F & [\textipa{f}] & - & + & - & - & - & + & - & - & - & - & + & - & - & - & - & - & - & - \\[-3.5pt] 
G & [\textipa{g}] & - & + & - & + & - & - & - & - & - & - & - & - & - & - & - & - & - & - \\[-3.5pt] 
HH & [\textipa{h}] & - & + & - & - & - & + & - & - & - & - & - & - & + & - & - & - & - & - \\[-2pt] \hline
IH & [\textipa{I}] & + & - & + & + & - & + & - & - & - & - & - & - & - & + & - & - & - & - \\[-3.5pt] 
IY & [\textipa{i}] & + & - & + & + & - & + & - & - & - & - & - & - & - & + & - & - & + & - \\[-3.5pt] 
JH & [\textipa{dZ}] & - & + & - & + & - & - & - & - & + & + & - & - & - & - & - & - & - & - \\[-3.5pt] 
K & [\textipa{k}] & - & + & - & - & + & - & - & - & - & - & - & - & - & - & - & - & - & - \\[-3.5pt] 
L & [\textipa{l}] & - & + & + & + & - & + & - & + & - & - & + & + & - & - & - & - & - & - \\[-2pt] \hline
M & [\textipa{m}] & - & + & + & + & - & - & + & - & - & - & + & - & - & - & - & - & - & - \\[-3.5pt] 
N & [\textipa{n}] & - & + & + & + & - & - & + & - & - & - & + & + & - & - & - & - & - & - \\[-3.5pt] 
NG & [\textipa{N}] & - & + & + & + & - & - & + & - & - & - & - & - & - & - & - & - & - & - \\[-3.5pt] 
OW & [\textipa{oU}] & + & - & + & + & - & + & - & - & - & - & - & - & - & - & - & + & + & + \\[-3.5pt] 
OY & [\textipa{OI}] & + & - & + & + & - & + & - & - & - & - & - & - & - & - & - & + & - & + \\[-2pt] \hline
P & [\textipa{p}] & - & + & - & - & + & - & - & - & - & - & + & - & - & - & - & - & - & - \\[-3.5pt] 
R & [\textipa{\*r}] & - & + & + & + & - & + & - & - & - & - & + & - & - & - & - & - & - & - \\[-3.5pt] 
S & [\textipa{s}] & - & + & - & - & - & + & - & - & + & - & + & + & - & - & - & - & - & - \\[-3.5pt] 
SH & [\textipa{S}] & - & + & - & - & - & + & - & - & + & - & - & - & + & + & - & - & - & - \\[-3.5pt] 
T & [\textipa{t}] & - & + & - & - & + & - & - & - & - & - & + & + & - & - & - & - & - & - \\[-2pt] \hline
TH & [\textipa{T}] & - & + & - & - & - & + & - & - & - & - & + & + & - & - & - & - & - & - \\[-3.5pt] 
UH & [\textipa{U}] & + & - & + & + & - & + & - & - & - & - & - & - & - & + & - & + & - & + \\[-3.5pt] 
UW & [\textipa{u}] & + & - & + & + & - & + & - & - & - & - & - & - & - & + & - & + & + & + \\[-3.5pt] 
V & [\textipa{v}] & - & + & - & + & - & + & - & - & - & - & + & - & - & - & - & - & - & - \\[-3.5pt] 
W & [\textipa{w}] & - & - & + & + & - & + & - & - & - & - & - & - & - & - & - & - & - & - \\[-2pt] \hline
Y & [\textipa{y}] & - & - & + & + & - & + & - & - & - & - & - & - & - & - & - & - & - & - \\[-3.5pt] 
Z & [\textipa{z}] & - & + & - & + & - & + & - & - & + & - & + & + & - & - & - & - & - & - \\[-3.5pt] 
ZH & [\textipa{Z}] & - & + & - & + & - & - & - & - & + & - & - & - & - & - & - & - & - & - \\ \hline
\end{tabular}
\end{center}
\label{tab:dist-features-eng}
\end{table}


\tabcolsep=0.15cm
\begin{table}[htbp]
\caption{Distinctive features chart for BP phones \citep{Jensen2004}.}
\begin{center}
\begin{tabular}{|cc|cccccccccccccccccc|}\hline
 &  & \multicolumn{18}{c|}{\textsc{Distinctive features}} \\ \cline{3-20}
\rotatebox[origin=c]{90}{\textsc{\textsc{CMU symbol}}} & \rotatebox[origin=c]{90}{\textsc{\textsc{IPA symbol}}} & \rotatebox[origin=c]{90}{\textsc{\textsc{syllabic}}} & \rotatebox[origin=c]{90}{\textsc{consonantal}} & \rotatebox[origin=c]{90}{\textsc{sonorant}} & \rotatebox[origin=c]{90}{\textsc{voice}} & \rotatebox[origin=c]{90}{\textsc{HSP}} & \rotatebox[origin=c]{90}{\textsc{continuant}} & \rotatebox[origin=c]{90}{\textsc{nasal}} & \rotatebox[origin=c]{90}{\textsc{lateral}} & \rotatebox[origin=c]{90}{\textsc{strident}} & \rotatebox[origin=c]{90}{\textsc{del. release}} & \rotatebox[origin=c]{90}{\textsc{anterior}} & \rotatebox[origin=c]{90}{\textsc{coronal}} & \rotatebox[origin=c]{90}{\textsc{  distributed  }} & \rotatebox[origin=c]{90}{\textsc{high}} & \rotatebox[origin=c]{90}{\textsc{low}} & \rotatebox[origin=c]{90}{\textsc{back}} & \rotatebox[origin=c]{90}{\textsc{ATR}} & \rotatebox[origin=c]{90}{\textsc{round}} \\ \hline
a & [\textipa{a}] & + & - & + & + & - & + & - & - & - & - & - & - & - & - & + & + & + & -\\[-4pt]
a$\sim$ & [\textipa{\~a}] & + & - & + & + & - & + & + & - & - & - & - & - & - & - & + & + & + & -\\[-4pt]
b & [\textipa{b}] & - & + & - & + & - & - & - & - & - & - & + & - & - & - & - & - & - & -\\[-4pt]
d & [\textipa{d}] & - & + & - & + & - & - & - & - & - & - & + & + & - & - & - & - & - & -\\[-4pt]
dZ & [\textipa{dZ}] & - & + & - & + & - & - & - & - & + & + & - & - & - & - & - & - & - & -\\[-2pt] \hline
E & [\textipa{E}] & + & - & - & + & - & + & - & - & - & - & - & - & - & - & - & - & - & -\\[-4pt]
e & [\textipa{e}] & + & - & + & + & - & + & - & - & - & - & - & - & - & - & - & - & + & -\\[-4pt]
e$\sim$ & [\textipa{\~e}] & + & - & + & + & - & + & + & - & - & - & - & - & - & - & - & - & + & -\\[-4pt]
f & [\textipa{f}] & - & + & - & - & - & + & - & - & - & - & + & - & - & - & - & - & - & -\\[-4pt]
g & [\textipa{g}] & - & + & - & + & - & - & - & - & - & - & - & - & - & - & - & - & - & -\\[-2pt] \hline
G & [\textipa{G}] & - & + & - & + & - & + & - & - & - & - & - & - & + & - & - & - & + & -\\[-4pt]
i & [\textipa{i}] & + & - & + & + & - & + & - & - & - & - & - & - & - & + & - & - & + & -\\[-4pt]
I & [\textipa{I}] & + & - & + & + & - & + & - & - & - & - & - & - & - & + & - & - & - & -\\[-4pt]
i$\sim$ & [\textipa{\~i}] & + & - & + & + & - & + & + & - & - & - & - & - & - & + & - & - & + & -\\[-4pt]
J & [\textipa{\textltailn}] & - & + & + & + & - & - & + & - & - & - & - & - & + & - & - & - & - & -\\[-2pt] \hline
j & [\textipa{y}] & - & - & + & + & - & + & - & - & - & - & - & - & - & - & - & - & - & -\\[-4pt]
j$\sim$ & [\textipa{\~y}] & - & - & + & + & - & + & + & - & - & - & - & - & - & - & - & - & - & -\\[-4pt]
k & [\textipa{k}] & - & + & - & - & - & - & - & - & - & - & - & - & - & - & - & - & - & -\\[-4pt]
l & [\textipa{l}] & - & + & + & + & - & + & - & + & - & - & + & + & - & - & - & - & - & -\\[-4pt]
L & [\textipa{L}] & - & + & + & + & - & + & - & + & - & - & + & + & + & - & - & - & - & -\\[-2pt] \hline
m & [\textipa{m}] & - & + & + & + & - & - & + & - & - & - & + & - & - & - & - & - & - & -\\[-4pt]
n & [\textipa{n}] & - & + & + & + & - & - & + & - & - & - & + & + & - & - & - & - & - & -\\[-4pt]
O & [\textipa{O}] & + & - & + & + & - & + & - & - & - & - & - & - & - & - & + & + & - & +\\[-4pt]
o & [\textipa{o}] & + & - & + & + & - & + & - & - & - & - & - & - & - & - & - & + & + & +\\[-4pt]
o$\sim$ & [\textipa{\~o}] & + & - & + & + & - & + & + & - & - & - & - & - & - & - & - & + & + & +\\[-2pt] \hline
p & [\textipa{p}] & - & + & - & - & - & - & - & - & - & - & + & - & - & - & - & - & - & -\\[-4pt]
s & [\textipa{s}] & - & + & - & - & - & + & - & - & + & - & + & + & - & - & - & - & - & -\\[-4pt]
S & [\textipa{s}] & - & + & - & - & - & + & - & - & + & - & - & - & + & + & - & - & - & -\\[-4pt]
t & [\textipa{t}] & - & + & - & - & - & - & - & - & - & - & + & + & - & - & - & - & - & -\\[-4pt]
tS & [\textipa{tS}] & - & + & - & - & - & - & - & - & + & + & - & - & - & - & - & - & - & -\\[-2pt] \hline
u & [\textipa{u}] & + & - & + & + & - & + & - & - & - & - & - & - & - & + & - & + & + & +\\[-4pt]
U & [\textipa{U}] & + & - & + & + & - & + & - & - & - & - & - & - & - & + & - & + & - & +\\[-4pt]
u$\sim$ & [\textipa{\~u}] & + & - & + & + & - & + & + & - & - & - & - & - & - & + & - & + & + & +\\[-4pt]
v & [\textipa{v}] & - & + & - & + & - & + & - & - & - & - & + & - & - & - & - & - & - & -\\[-4pt]
w & [\textipa{w}] & - & - & + & + & - & + & - & - & - & - & - & - & - & - & - & - & - & -\\[-2pt] \hline
w$\sim$ & [\textipa{\~w}] & - & - & + & + & - & + & + & - & - & - & - & - & - & - & - & - & - & -\\[-4pt]
x & [\textipa{x}] & - & + & - & - & - & + & - & - & - & - & - & - & + & - & - & - & + & -\\[-4pt]
z & [\textipa{z}] & - & + & - & + & - & + & - & - & + & - & + & + & - & - & - & - & - & -\\[-4pt]
Z & [\textipa{z}] & - & + & - & + & - & - & - & - & + & - & - & - & - & - & - & - & - & -\\[-4pt]
4 & [\textipa{R}] & - & + & + & + & + & + & - & - & - & - & + & - & - & - & - & - & - & -\\[-2pt] \hline
@ & [\textipa{@}] & + & - & + & + & - & + & - & - & - & - & - & - & - & - & + & + & - & -\\ \hline
\end{tabular}
\end{center}
\label{tab:dist-features-bp}
\end{table}

\renewcommand{\arraystretch}{0.55}% Tighter
\tabcolsep=0.15cm
\begin{table}[htbp]
\caption{Distinctive features chart for Listener phones.}
\begin{center}
\begin{tabular}{|ccc|cccccccccccccccccc|}\hline
 & &  & \multicolumn{18}{c|}{\textsc{Distinctive features}} \\ \cline{4-21}
\# & \rotatebox[origin=c]{90}{\textsc{\textsc{Listener symbol}}} & \rotatebox[origin=c]{90}{\textsc{\textsc{IPA symbol}}} & \rotatebox[origin=c]{90}{\textsc{\textsc{syllabic}}} & \rotatebox[origin=c]{90}{\textsc{consonantal}} & \rotatebox[origin=c]{90}{\textsc{sonorant}} & \rotatebox[origin=c]{90}{\textsc{voice}} & \rotatebox[origin=c]{90}{\textsc{HSP}} & \rotatebox[origin=c]{90}{\textsc{continuant}} & \rotatebox[origin=c]{90}{\textsc{nasal}} & \rotatebox[origin=c]{90}{\textsc{lateral}} & \rotatebox[origin=c]{90}{\textsc{strident}} & \rotatebox[origin=c]{90}{\textsc{del. release}} & \rotatebox[origin=c]{90}{\textsc{anterior}} & \rotatebox[origin=c]{90}{\textsc{coronal}} & \rotatebox[origin=c]{90}{\textsc{  distributed  }} & \rotatebox[origin=c]{90}{\textsc{high}} & \rotatebox[origin=c]{90}{\textsc{low}} & \rotatebox[origin=c]{90}{\textsc{back}} & \rotatebox[origin=c]{90}{\textsc{ATR}} & \rotatebox[origin=c]{90}{\textsc{round}} \\ \hline
\footnotesize 1 & \small a & \footnotesize [\textipa{@}] & \footnotesize + & \footnotesize - & \footnotesize + & \footnotesize + & \footnotesize - & \footnotesize + & \footnotesize - & \footnotesize - & \footnotesize - & \footnotesize - & \footnotesize - & \footnotesize - & \footnotesize - & \footnotesize - & \footnotesize + & \footnotesize + & \footnotesize - & \footnotesize -\\ 
\footnotesize 2 & \small aa & \footnotesize [\textipa{A}] & \footnotesize + & \footnotesize - & \footnotesize + & \footnotesize + & \footnotesize - & \footnotesize + & \footnotesize - & \footnotesize - & \footnotesize - & \footnotesize - & \footnotesize - & \footnotesize - & \footnotesize - & \footnotesize - & \footnotesize + & \footnotesize + & \footnotesize + & \footnotesize - \\
\footnotesize 3 & \small aaa & \footnotesize [\textipa{a}] & \footnotesize + & \footnotesize - & \footnotesize + & \footnotesize + & \footnotesize - & \footnotesize + & \footnotesize - & \footnotesize - & \footnotesize - & \footnotesize - & \footnotesize - & \footnotesize - & \footnotesize - & \footnotesize - & \footnotesize + & \footnotesize + & \footnotesize + & \footnotesize -\\
\footnotesize 4 & \small ae & \footnotesize [\textipa{ae}] & \footnotesize + & \footnotesize - & \footnotesize + & \footnotesize + & \footnotesize - & \footnotesize + & \footnotesize - & \footnotesize - & \footnotesize - & \footnotesize - & \footnotesize - & \footnotesize - & \footnotesize - & \footnotesize - & \footnotesize + & \footnotesize - & \footnotesize - & \footnotesize - \\
\footnotesize 5 & \small ah & \footnotesize [\textipa{@}] & \footnotesize + & \footnotesize - & \footnotesize + & \footnotesize + & \footnotesize - & \footnotesize + & \footnotesize - & \footnotesize - & \footnotesize - & \footnotesize - & \footnotesize - & \footnotesize - & \footnotesize - & \footnotesize - & \footnotesize - & \footnotesize + & \footnotesize - & \footnotesize - \\ \hline
\footnotesize 6 & \small am & \footnotesize [\textipa{\~a}] & \footnotesize + & \footnotesize - & \footnotesize + & \footnotesize + & \footnotesize - & \footnotesize + & \footnotesize + & \footnotesize - & \footnotesize - & \footnotesize - & \footnotesize - & \footnotesize - & \footnotesize - & \footnotesize - & \footnotesize + & \footnotesize + & \footnotesize + & \footnotesize -\\
\footnotesize 7 & \small ao & \footnotesize [\textipa{O}] & \footnotesize + & \footnotesize - & \footnotesize + & \footnotesize + & \footnotesize - & \footnotesize + & \footnotesize - & \footnotesize - & \footnotesize - & \footnotesize - & \footnotesize - & \footnotesize - & \footnotesize - & \footnotesize - & \footnotesize + & \footnotesize + & \footnotesize - & \footnotesize + \\
\footnotesize 8 & \small aw & \footnotesize [\textipa{aU}] & \footnotesize + & \footnotesize - & \footnotesize + & \footnotesize + & \footnotesize - & \footnotesize + & \footnotesize - & \footnotesize - & \footnotesize - & \footnotesize - & \footnotesize - & \footnotesize - & \footnotesize - & \footnotesize - & \footnotesize - & \footnotesize - & \footnotesize - & \footnotesize - \\ 
\footnotesize 9 & \small awm & \footnotesize [\textipa{\~w}] & \footnotesize - & \footnotesize - & \footnotesize + & \footnotesize + & \footnotesize - & \footnotesize + & \footnotesize + & \footnotesize - & \footnotesize - & \footnotesize - & \footnotesize - & \footnotesize - & \footnotesize - & \footnotesize - & \footnotesize - & \footnotesize - & \footnotesize - & \footnotesize -\\
\footnotesize 10 & \small ay & \footnotesize [\textipa{aI}] & \footnotesize + & \footnotesize - & \footnotesize + & \footnotesize + & \footnotesize - & \footnotesize + & \footnotesize - & \footnotesize - & \footnotesize - & \footnotesize - & \footnotesize - & \footnotesize - & \footnotesize - & \footnotesize - & \footnotesize - & \footnotesize - & \footnotesize - & \footnotesize - \\  \hline
\footnotesize 11 & \small aym & \footnotesize [\textipa{\~y}] & \footnotesize - & \footnotesize - & \footnotesize + & \footnotesize + & \footnotesize - & \footnotesize + & \footnotesize + & \footnotesize - & \footnotesize - & \footnotesize - & \footnotesize - & \footnotesize - & \footnotesize - & \footnotesize - & \footnotesize - & \footnotesize - & \footnotesize - & \footnotesize -\\
\footnotesize 12 & \small b & \footnotesize [\textipa{b}] & \footnotesize - & \footnotesize + & \footnotesize - & \footnotesize + & \footnotesize - & \footnotesize - & \footnotesize - & \footnotesize - & \footnotesize - & \footnotesize - & \footnotesize + & \footnotesize - & \footnotesize - & \footnotesize - & \footnotesize - & \footnotesize - & \footnotesize - & \footnotesize - \\ 
\footnotesize 13 & \small ch & \footnotesize [\textipa{tS}] & \footnotesize - & \footnotesize + & \footnotesize - & \footnotesize - & \footnotesize - & \footnotesize - & \footnotesize - & \footnotesize - & \footnotesize + & \footnotesize + & \footnotesize - & \footnotesize - & \footnotesize - & \footnotesize - & \footnotesize - & \footnotesize - & \footnotesize - & \footnotesize - \\ 
\footnotesize 15 & \small d & \footnotesize [\textipa{d}] & \footnotesize - & \footnotesize + & \footnotesize - & \footnotesize + & \footnotesize - & \footnotesize - & \footnotesize - & \footnotesize - & \footnotesize - & \footnotesize - & \footnotesize + & \footnotesize + & \footnotesize - & \footnotesize - & \footnotesize - & \footnotesize - & \footnotesize - & \footnotesize -\\  \hline
\footnotesize 16 & \small dh & \footnotesize [\textipa{D}] & \footnotesize - & \footnotesize + & \footnotesize - & \footnotesize + & \footnotesize - & \footnotesize + & \footnotesize - & \footnotesize - & \footnotesize - & \footnotesize - & \footnotesize + & \footnotesize + & \footnotesize - & \footnotesize - & \footnotesize - & \footnotesize - & \footnotesize - & \footnotesize - \\ 
\footnotesize 17 & \small e & \footnotesize [\textipa{e}] & \footnotesize + & \footnotesize - & \footnotesize + & \footnotesize + & \footnotesize - & \footnotesize + & \footnotesize - & \footnotesize - & \footnotesize - & \footnotesize - & \footnotesize - & \footnotesize - & \footnotesize - & \footnotesize - & \footnotesize - & \footnotesize - & \footnotesize + & \footnotesize -\\
\footnotesize 18 & \small eh & \footnotesize [\textipa{E}] & \footnotesize + & \footnotesize - & \footnotesize - & \footnotesize + & \footnotesize - & \footnotesize + & \footnotesize - & \footnotesize - & \footnotesize - & \footnotesize - & \footnotesize - & \footnotesize - & \footnotesize - & \footnotesize - & \footnotesize - & \footnotesize - & \footnotesize - & \footnotesize - \\ 
\footnotesize 19 & \small em & \footnotesize [\textipa{\~e}] & \footnotesize + & \footnotesize - & \footnotesize + & \footnotesize + & \footnotesize - & \footnotesize + & \footnotesize + & \footnotesize - & \footnotesize - & \footnotesize - & \footnotesize - & \footnotesize - & \footnotesize - & \footnotesize - & \footnotesize - & \footnotesize - & \footnotesize + & \footnotesize -\\
\footnotesize 20 & \small ey & \footnotesize [\textipa{A\*r}] & \footnotesize + & \footnotesize - & \footnotesize + & \footnotesize + & \footnotesize - & \footnotesize + & \footnotesize - & \footnotesize - & \footnotesize - & \footnotesize - & \footnotesize - & \footnotesize - & \footnotesize - & \footnotesize - & \footnotesize - & \footnotesize - & \footnotesize + & \footnotesize - \\  \hline
\footnotesize 21 & \small eym & \footnotesize [\textipa{\~y}] & \footnotesize - & \footnotesize - & \footnotesize + & \footnotesize + & \footnotesize - & \footnotesize + & \footnotesize + & \footnotesize - & \footnotesize - & \footnotesize - & \footnotesize - & \footnotesize - & \footnotesize - & \footnotesize - & \footnotesize - & \footnotesize - & \footnotesize - & \footnotesize -\\
\footnotesize 22 & \small f & \footnotesize [\textipa{f}] & \footnotesize - & \footnotesize + & \footnotesize - & \footnotesize - & \footnotesize - & \footnotesize + & \footnotesize - & \footnotesize - & \footnotesize - & \footnotesize - & \footnotesize + & \footnotesize - & \footnotesize - & \footnotesize - & \footnotesize - & \footnotesize - & \footnotesize - & \footnotesize - \\ 
\footnotesize 23 & \small g & \footnotesize [\textipa{g}] & \footnotesize - & \footnotesize + & \footnotesize - & \footnotesize + & \footnotesize - & \footnotesize - & \footnotesize - & \footnotesize - & \footnotesize - & \footnotesize - & \footnotesize - & \footnotesize - & \footnotesize - & \footnotesize - & \footnotesize - & \footnotesize - & \footnotesize - & \footnotesize - \\ 
\footnotesize 24 & \small hh & \footnotesize [\textipa{h}] & \footnotesize - & \footnotesize + & \footnotesize - & \footnotesize - & \footnotesize - & \footnotesize + & \footnotesize - & \footnotesize - & \footnotesize - & \footnotesize - & \footnotesize - & \footnotesize - & \footnotesize + & \footnotesize - & \footnotesize - & \footnotesize - & \footnotesize - & \footnotesize - \\ 
\footnotesize 25 & \small i & \footnotesize [\textipa{i}] & \footnotesize + & \footnotesize - & \footnotesize + & \footnotesize + & \footnotesize - & \footnotesize + & \footnotesize - & \footnotesize - & \footnotesize - & \footnotesize - & \footnotesize - & \footnotesize - & \footnotesize - & \footnotesize + & \footnotesize - & \footnotesize - & \footnotesize + & \footnotesize -\\  \hline
\footnotesize 26 & \small ih & \footnotesize [\textipa{I}] & \footnotesize + & \footnotesize - & \footnotesize + & \footnotesize + & \footnotesize - & \footnotesize + & \footnotesize - & \footnotesize - & \footnotesize - & \footnotesize - & \footnotesize - & \footnotesize - & \footnotesize - & \footnotesize + & \footnotesize - & \footnotesize - & \footnotesize - & \footnotesize - \\ 
\footnotesize 27 & \small im & \footnotesize [\textipa{\~i}] & \footnotesize + & \footnotesize - & \footnotesize + & \footnotesize + & \footnotesize - & \footnotesize + & \footnotesize + & \footnotesize - & \footnotesize - & \footnotesize - & \footnotesize - & \footnotesize - & \footnotesize - & \footnotesize + & \footnotesize - & \footnotesize - & \footnotesize + & \footnotesize -\\
\footnotesize 28 & \small iy & \footnotesize [\textipa{i}] & \footnotesize + & \footnotesize - & \footnotesize + & \footnotesize + & \footnotesize - & \footnotesize + & \footnotesize - & \footnotesize - & \footnotesize - & \footnotesize - & \footnotesize - & \footnotesize - & \footnotesize - & \footnotesize + & \footnotesize - & \footnotesize - & \footnotesize + & \footnotesize - \\ 
\footnotesize 29 & \small jh & \footnotesize [\textipa{dZ}] & \footnotesize - & \footnotesize + & \footnotesize - & \footnotesize + & \footnotesize - & \footnotesize - & \footnotesize - & \footnotesize - & \footnotesize + & \footnotesize + & \footnotesize - & \footnotesize - & \footnotesize - & \footnotesize - & \footnotesize - & \footnotesize - & \footnotesize - & \footnotesize - \\ 
\footnotesize 30 & \small k & \footnotesize [\textipa{k}] & \footnotesize - & \footnotesize + & \footnotesize - & \footnotesize - & \footnotesize + & \footnotesize - & \footnotesize - & \footnotesize - & \footnotesize - & \footnotesize - & \footnotesize - & \footnotesize - & \footnotesize - & \footnotesize - & \footnotesize - & \footnotesize - & \footnotesize - & \footnotesize - \\  \hline
\footnotesize 31 & \small kk & \footnotesize [\textipa{k}] & \footnotesize - & \footnotesize + & \footnotesize - & \footnotesize - & \footnotesize - & \footnotesize - & \footnotesize - & \footnotesize - & \footnotesize - & \footnotesize - & \footnotesize - & \footnotesize - & \footnotesize - & \footnotesize - & \footnotesize - & \footnotesize - & \footnotesize - & \footnotesize -\\
\footnotesize 32 & \small l & \footnotesize [\textipa{l}] & \footnotesize - & \footnotesize + & \footnotesize + & \footnotesize + & \footnotesize - & \footnotesize + & \footnotesize - & \footnotesize + & \footnotesize - & \footnotesize - & \footnotesize + & \footnotesize + & \footnotesize - & \footnotesize - & \footnotesize - & \footnotesize - & \footnotesize - & \footnotesize - \\ 
\footnotesize 33 & \small lh & \footnotesize [\textipa{L}] & \footnotesize - & \footnotesize + & \footnotesize + & \footnotesize + & \footnotesize - & \footnotesize + & \footnotesize - & \footnotesize + & \footnotesize - & \footnotesize - & \footnotesize + & \footnotesize + & \footnotesize + & \footnotesize - & \footnotesize - & \footnotesize - & \footnotesize - & \footnotesize -\\ 
\footnotesize 34 & \small m & \footnotesize [\textipa{m}] & \footnotesize - & \footnotesize + & \footnotesize + & \footnotesize + & \footnotesize - & \footnotesize - & \footnotesize + & \footnotesize - & \footnotesize - & \footnotesize - & \footnotesize + & \footnotesize - & \footnotesize - & \footnotesize - & \footnotesize - & \footnotesize - & \footnotesize - & \footnotesize - \\ 
\footnotesize 35 & \small n & \footnotesize [\textipa{n}] & \footnotesize - & \footnotesize + & \footnotesize + & \footnotesize + & \footnotesize - & \footnotesize - & \footnotesize + & \footnotesize - & \footnotesize - & \footnotesize - & \footnotesize + & \footnotesize + & \footnotesize - & \footnotesize - & \footnotesize - & \footnotesize - & \footnotesize - & \footnotesize - \\  \hline
\footnotesize 36 & \small ng & \footnotesize [\textipa{N}] & \footnotesize - & \footnotesize + & \footnotesize + & \footnotesize + & \footnotesize - & \footnotesize - & \footnotesize + & \footnotesize - & \footnotesize - & \footnotesize - & \footnotesize - & \footnotesize - & \footnotesize - & \footnotesize - & \footnotesize - & \footnotesize - & \footnotesize - & \footnotesize - \\ 
\footnotesize 37 & \small nh & \footnotesize [\textipa{\textltailn}] & \footnotesize - & \footnotesize + & \footnotesize + & \footnotesize + & \footnotesize - & \footnotesize - & \footnotesize + & \footnotesize - & \footnotesize - & \footnotesize - & \footnotesize - & \footnotesize - & \footnotesize + & \footnotesize - & \footnotesize - & \footnotesize - & \footnotesize - & \footnotesize -\\ 
\footnotesize 38 & \small o & \footnotesize [\textipa{o}] & \footnotesize + & \footnotesize - & \footnotesize + & \footnotesize + & \footnotesize - & \footnotesize + & \footnotesize - & \footnotesize - & \footnotesize - & \footnotesize - & \footnotesize - & \footnotesize - & \footnotesize - & \footnotesize - & \footnotesize - & \footnotesize + & \footnotesize + & \footnotesize +\\
\footnotesize 39 & \small om & \footnotesize [\textipa{\~o}] & \footnotesize + & \footnotesize - & \footnotesize + & \footnotesize + & \footnotesize - & \footnotesize + & \footnotesize + & \footnotesize - & \footnotesize - & \footnotesize - & \footnotesize - & \footnotesize - & \footnotesize - & \footnotesize - & \footnotesize - & \footnotesize + & \footnotesize + & \footnotesize +\\ 
\footnotesize 40 & \small ow & \footnotesize [\textipa{oU}] & \footnotesize + & \footnotesize - & \footnotesize + & \footnotesize + & \footnotesize - & \footnotesize + & \footnotesize - & \footnotesize - & \footnotesize - & \footnotesize - & \footnotesize - & \footnotesize - & \footnotesize - & \footnotesize - & \footnotesize - & \footnotesize + & \footnotesize + & \footnotesize + \\  \hline
\footnotesize 41 & \small oy & \footnotesize [\textipa{OI}] & \footnotesize + & \footnotesize - & \footnotesize + & \footnotesize + & \footnotesize - & \footnotesize + & \footnotesize - & \footnotesize - & \footnotesize - & \footnotesize - & \footnotesize - & \footnotesize - & \footnotesize - & \footnotesize - & \footnotesize - & \footnotesize + & \footnotesize - & \footnotesize + \\ 
\footnotesize 42 & \small oym & \footnotesize [\textipa{\~y}] & \footnotesize - & \footnotesize - & \footnotesize + & \footnotesize + & \footnotesize - & \footnotesize + & \footnotesize + & \footnotesize - & \footnotesize - & \footnotesize - & \footnotesize - & \footnotesize - & \footnotesize - & \footnotesize - & \footnotesize - & \footnotesize - & \footnotesize - & \footnotesize -\\
\footnotesize 43 & \small p & \footnotesize [\textipa{p}] & \footnotesize - & \footnotesize + & \footnotesize - & \footnotesize - & \footnotesize + & \footnotesize - & \footnotesize - & \footnotesize - & \footnotesize - & \footnotesize - & \footnotesize + & \footnotesize - & \footnotesize - & \footnotesize - & \footnotesize - & \footnotesize - & \footnotesize - & \footnotesize - \\ 
\footnotesize 44 & \small pp & \footnotesize [\textipa{p}] & \footnotesize - & \footnotesize + & \footnotesize - & \footnotesize - & \footnotesize - & \footnotesize - & \footnotesize - & \footnotesize - & \footnotesize - & \footnotesize - & \footnotesize + & \footnotesize - & \footnotesize - & \footnotesize - & \footnotesize - & \footnotesize - & \footnotesize - & \footnotesize -\\
\footnotesize 45 & \small r & \footnotesize [\textipa{\*r}] & \footnotesize - & \footnotesize + & \footnotesize + & \footnotesize + & \footnotesize - & \footnotesize + & \footnotesize - & \footnotesize - & \footnotesize - & \footnotesize - & \footnotesize + & \footnotesize - & \footnotesize - & \footnotesize - & \footnotesize - & \footnotesize - & \footnotesize - & \footnotesize - \\  \hline
\footnotesize 46 & \small rd & \footnotesize [\textipa{R}] & \footnotesize - & \footnotesize + & \footnotesize + & \footnotesize + & \footnotesize + & \footnotesize + & \footnotesize - & \footnotesize - & \footnotesize - & \footnotesize - & \footnotesize + & \footnotesize - & \footnotesize - & \footnotesize - & \footnotesize - & \footnotesize - & \footnotesize - & \footnotesize -\\ 
\footnotesize 47 & \small s & \footnotesize [\textipa{s}] & \footnotesize - & \footnotesize + & \footnotesize - & \footnotesize - & \footnotesize - & \footnotesize + & \footnotesize - & \footnotesize - & \footnotesize + & \footnotesize - & \footnotesize + & \footnotesize + & \footnotesize - & \footnotesize - & \footnotesize - & \footnotesize - & \footnotesize - & \footnotesize - \\ 
\footnotesize 48 & \small sh & \footnotesize [\textipa{S}] & \footnotesize - & \footnotesize + & \footnotesize - & \footnotesize - & \footnotesize - & \footnotesize + & \footnotesize - & \footnotesize - & \footnotesize + & \footnotesize - & \footnotesize - & \footnotesize - & \footnotesize + & \footnotesize + & \footnotesize - & \footnotesize - & \footnotesize - & \footnotesize - \\ 
\footnotesize 49 & \small t & \footnotesize [\textipa{t}] & \footnotesize - & \footnotesize + & \footnotesize - & \footnotesize - & \footnotesize + & \footnotesize - & \footnotesize - & \footnotesize - & \footnotesize - & \footnotesize - & \footnotesize + & \footnotesize + & \footnotesize - & \footnotesize - & \footnotesize - & \footnotesize - & \footnotesize - & \footnotesize - \\ 
\footnotesize 50 & \small th & \footnotesize [\textipa{T}] & \footnotesize - & \footnotesize + & \footnotesize - & \footnotesize - & \footnotesize - & \footnotesize + & \footnotesize - & \footnotesize - & \footnotesize - & \footnotesize - & \footnotesize + & \footnotesize + & \footnotesize - & \footnotesize - & \footnotesize - & \footnotesize - & \footnotesize - & \footnotesize - \\  \hline
\footnotesize 51 & \small tt & \footnotesize [\textipa{t}] & \footnotesize - & \footnotesize + & \footnotesize - & \footnotesize - & \footnotesize - & \footnotesize - & \footnotesize - & \footnotesize - & \footnotesize - & \footnotesize - & \footnotesize + & \footnotesize + & \footnotesize - & \footnotesize - & \footnotesize - & \footnotesize - & \footnotesize - & \footnotesize -\\
\footnotesize 52 & \small u & \footnotesize [\textipa{u}] & \footnotesize + & \footnotesize - & \footnotesize + & \footnotesize + & \footnotesize - & \footnotesize + & \footnotesize - & \footnotesize - & \footnotesize - & \footnotesize - & \footnotesize - & \footnotesize - & \footnotesize - & \footnotesize + & \footnotesize - & \footnotesize + & \footnotesize + & \footnotesize +\\
\footnotesize 53 & \small uh & \footnotesize [\textipa{U}] & \footnotesize + & \footnotesize - & \footnotesize + & \footnotesize + & \footnotesize - & \footnotesize + & \footnotesize - & \footnotesize - & \footnotesize - & \footnotesize - & \footnotesize - & \footnotesize - & \footnotesize - & \footnotesize + & \footnotesize - & \footnotesize + & \footnotesize - & \footnotesize + \\ 
\footnotesize 54 & \small um & \footnotesize [\textipa{\~u}] & \footnotesize + & \footnotesize - & \footnotesize + & \footnotesize + & \footnotesize - & \footnotesize + & \footnotesize + & \footnotesize - & \footnotesize - & \footnotesize - & \footnotesize - & \footnotesize - & \footnotesize - & \footnotesize + & \footnotesize - & \footnotesize + & \footnotesize + & \footnotesize +\\
\footnotesize 55 & \small uw & \footnotesize [\textipa{u}] & \footnotesize + & \footnotesize - & \footnotesize + & \footnotesize + & \footnotesize - & \footnotesize + & \footnotesize - & \footnotesize - & \footnotesize - & \footnotesize - & \footnotesize - & \footnotesize - & \footnotesize - & \footnotesize + & \footnotesize - & \footnotesize + & \footnotesize + & \footnotesize + \\  \hline
\footnotesize 56 & \small v & \footnotesize [\textipa{v}] & \footnotesize - & \footnotesize + & \footnotesize - & \footnotesize + & \footnotesize - & \footnotesize + & \footnotesize - & \footnotesize - & \footnotesize - & \footnotesize - & \footnotesize + & \footnotesize - & \footnotesize - & \footnotesize - & \footnotesize - & \footnotesize - & \footnotesize - & \footnotesize - \\ 
\footnotesize 57 & \small w & \footnotesize [\textipa{w}] & \footnotesize - & \footnotesize - & \footnotesize + & \footnotesize + & \footnotesize - & \footnotesize + & \footnotesize - & \footnotesize - & \footnotesize - & \footnotesize - & \footnotesize - & \footnotesize - & \footnotesize - & \footnotesize - & \footnotesize - & \footnotesize - & \footnotesize - & \footnotesize - \\ 
\footnotesize 58 & \small y & \footnotesize [\textipa{y}] & \footnotesize - & \footnotesize - & \footnotesize + & \footnotesize + & \footnotesize - & \footnotesize + & \footnotesize - & \footnotesize - & \footnotesize - & \footnotesize - & \footnotesize - & \footnotesize - & \footnotesize - & \footnotesize - & \footnotesize - & \footnotesize - & \footnotesize - & \footnotesize - \\ 
\footnotesize 59 & \small z & \footnotesize [\textipa{z}] & \footnotesize - & \footnotesize + & \footnotesize - & \footnotesize + & \footnotesize - & \footnotesize + & \footnotesize - & \footnotesize - & \footnotesize + & \footnotesize - & \footnotesize + & \footnotesize + & \footnotesize - & \footnotesize - & \footnotesize - & \footnotesize - & \footnotesize - & \footnotesize - \\ 
\footnotesize 60 & \small zh & \footnotesize [\textipa{Z}] & \footnotesize - & \footnotesize + & \footnotesize - & \footnotesize + & \footnotesize - & \footnotesize - & \footnotesize - & \footnotesize - & \footnotesize + & \footnotesize - & \footnotesize - & \footnotesize - & \footnotesize - & \footnotesize - & \footnotesize - & \footnotesize - & \footnotesize - & \footnotesize - \\ \hline
\end{tabular}
\end{center}
\label{tab:dist-features-listener}
\end{table}
\renewcommand{\arraystretch}{1.0}% Normal


\clearpage
\section{Building the Pronunciation Model}

To the best of our knowledge, no previous research has addressed the problem of generating brazilian-accented 
transcriptions for \ac{ASR} purposes or has described the mispronunciations phenomena from a computational perspective. 
Therefore we had to develop our own pronunciation model
\footnote{``Pronunciation models'' are also called ``pronunciation dictionaries''. In this thesis, we are going to
use both terms interchargeably, without any distinction.}. There are basically 
three main approaches we could use to build such model: rule-based methods (XXX CITATION), machine learning methods (XXX CITATION)
and hybrid ones (XXX CITATION). For achieving good performance through machine learning or hybrid approaches, one necessary 
needs a large annotated corpus. That is not the case though. The only Brazilian-accented transcribed corpus we have access
is the Listener Corpus, but we carried out some pilot experiments that showed it was not robust enough for the task.

That being so, we decided to make use of a rule-based approeach for building the pronunciation model. We reviewed all papers 
described in \autoref{ch:second-language}, that deal with the mispronunciation of English phones by brazilians, in order to find 
interlingual allophonies, such as the English [\textipa{T}] usually becomes [\textipa{t}], [\textipa{t\super h}] or [\textipa{f}] in 
beginners' speech \citep{Reis2006}. 

By knowing the allophonies, we developed rules in order to generate the mispronunciations in the dictionary. The mispronunciation contexts
specified through rewrite rules are thus a contribution of this thesis. It is worth mentioning that, for creating the rules, we
took into account the frequency of occurrence of each mispronunciation (when this information was available 
on the papers). On that account mispronunciations which were reported as being very rare or with no significant probability were excluded.

We implemented the rules through a Python script which made a large usage of the \ac{regex} library. As one familiar 
with \ac{regex} knows, \ac{regex} rules might get too clumsy and hard to understand. For instance, one of the rules in our Python script is:

\begin{lstlisting}[float=!h,caption=Example of a fully-specified regex rule.]
re.sub('(ih|iy) (m|n|ng) (#|[^aeiouyw])', r'i~ \3', pron)
\end{lstlisting}

To one not familiar with how the dictionary is structured, this rule, in our opinion, could be somewhat meaningless. Therefore,  to 
render the text easier to read, we are going to describe the rules in a pseudocode with a rather flexible notation
\footnote{For those who are insterested in checking the code itself, it can be downloaded on the site of the project:
\url{http://nilc.icmc.usp.br/listener}}
.
Whenever possible, we try to simplify the rules' contexts, so anyone not acquainted with the dictionary format or the
\ac{regex} syntax might understand their content.

\section{Building the Grammars and the Language Model}
\section{Results}



%*****************************************
%*****************************************
%*****************************************
%*****************************************
%*****************************************


%*****************************************
%*****************************************
%*****************************************
%*****************************************
%*****************************************