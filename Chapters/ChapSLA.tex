%*****************************************
\chapter{Second Language Acquisition}\label{ch:second-language}
%*****************************************

\graffito{Definition.}
Illo principalmente su nos. Non message \emph{occidental} angloromanic
da. Debitas effortio simplificate sia se, auxiliar summarios da que,
se avantiate publicationes via. Pan in terra summarios, capital
interlingua se que. Al via multo esser specimen, campo responder que
da. Le usate medical addresses pro, europa origine sanctificate nos
se.

\section{PB}

Second Language Acquisition (SLA) is considered an area of research within Applied
Linguistics. Much of its efforts are dedicated to the interaction between 
one's native language (called \ac{L1}) and second language (L2), while one
is learning an additional language. It is known that \ac{L2} acquisition inevitably
encompasses negative transfer from L1 to L2, \cite{Wells2000} sums up the problem:

Quando nos deparamos com uma l\'ingua estrangeira, a tend\^encia natural \'e
que interpretemos seus sons a partir dos sons de nossa pr\'opria l\'ingua.
Analogamente, quando falamos uma l\'ingua estrangeira, tendemos a utilizar
os sons e os padr\~oes sonoros de nossa l\'ingua nativa.

In what regard to the process of learning an additional language, \cite{Lenneberg1967}
proposed the well-known Critical Period Hypothesis to explain the different levels
of proffiency that people show. The hypothesis claim that the ability to acquire 
language is biologically linked to age, in a way that thtere is an ideal time window
to acquire a language, after which language acquisition becomes difficult and deteriorated.
In its initial formulation, the critical period was set to the age between two years and puberty.

Diversos pontos da formula\c{c}\~ao inicial da hip\'otese do per\'iodo cr\'itico j\'a
foram rebatidos; seu cerne, isto \'e, a ideia de que haja um tempo ideal
espec\'ifico para a aquisi\c{c}\~ao de l\'ingua adicional, j\'a foi revisada e a
Hip\'otese do Per\'iodo Cr\'itico \'e, hoje, reinterpretada (Hylternstam \&
Abrahamsson, 2000). No que se cr\^e, atualmente, \'e que restri\c{c}\~oes
maturacionais, aliadas a fatores s\'ocio-psicol\'ogicos, podem atuar de modo
a tornar o aprendizado mais lento ap\'os a puberdade. Qualquer um,
portanto, que se proponha a aprender uma l\'ingua ap\'os a puberdade,
tender\'a a desenvolver um sotaque estrangeiro. Esse sotaque \'e
caracterizado, principalmente, pela transfer\^encia de padr\~oes do sistema
fonol\'ogico da L1 para a L2 e, tamb\'em, pela transfer\^encia de padr\~oes de
correspond\^encia entre letra e som da L1 para a L2 (Zimmer \& Alves,
2006). O aprendiz tende a produzir na L2 padr\~oes ac\'ustico-articulat\'orios
id\^enticos ou semelhantes aos de sua L1, al\'em de tender a tratar as
unidades ac\'ustico-articulat\'orias da L2 como se fossem as da L1 (Zimmer,
2004).

Como exemplo, considere-se a realiza\c{c}\~ao das consoantes  oclusivas  [p,
t, k{]} no ingl\^es e no PB. No ingl\^es, tais consoantes al\'em de ocorrerem
em onset sil\'abico{[}4{]}, tamb\'em podem ocorrer posi\c{c}\~ao de coda{[}5{]},
de modo a compor uma s\'ilaba travada. H\'a, portanto, palavras como
{[}'pi?s{]} `piece', {[}'ta?m{]} `time' e {[}'kæn{]} `can', bem como
{[}'b?k{]} `book', {[}'st??rt{]} `start' e {[}'w???k{]} `work'. No PB,
por sua vez, tais oclusivas ocorrem apenas em onset sil\'abico, de modo
que o aprendiz de L2, ao lidar com oclusivas em final de s\'ilaba, tende a
transferir as caracter\'isticas de sua L1 para L2, realizando, assim,
ep\^enteses e processos de ressilabifica\c{c}\~ao no prop\'osito de reorganizar a
estrutura sil\'abica. Por exemplo, a Figura 1 apresenta a representa\c{c}\~ao
autossegmental (Selkirk, 1982) da palavra `book' na pron\'uncia padr\~ao do
ingl\^es e na pron\'uncia com transfer\^encia do PB para o ingl\^es.

Figura 1: Realiza\c{c}\~ao da palavra `book' na pron\'uncia padr\~ao do ingl\^es
{[}1{]}; realiza\c{c}\~ao da palavra `book' com transfer\^encia do PB para o
ingl\^es {[}2{]}

Como se observa, a pron\'uncia padr\~ao na l\'ingua inglesa da palavra `book'
\'e {[}'b?k{]}. No entanto, como no PB oclusivas n\~ao ocorrem em posi\c{c}\~ao de
coda, o aprendiz tende a realizar a palavra a partir dos padr\~oes
fonol\'ogicos que conhece em L1, efetuando a ep\^entese do {[}i, ?{]} e
ressilabificando a palavra, de modo a transform\'a-la de monoss\'ilaba a
diss\'ilaba: {[}'b?k{]} \textgreater{} {[}'bu.k?{]}. No processo de
comunica\c{c}\~ao entre um nativo e um aprendiz, \'e como se o nativo tivesse
como representa\c{c}\~ao mental /'b?k/ e realizasse na fala {[}'b?k{]}, mas o
aprendiz percebesse tal realiza\c{c}\~ao como /'bu.k?/ e, ent\~ao, realizasse em
sua fala {[}'bu.k?{]} (cf.~Figura 2).

Figura 2: Esquema de um processo dial\'ogico entre um nativo e um
aprendiz.

O sotaque estrangeiro, fruto dessa transfer\^encia de padr\~oes de L1 para a
L2, pode trazer preju\'izo ao processo comunicativo. No exemplo ilustrado
pela Figura 2, o aprendiz altera a qualidade da vogal esperada {[}?{]} e
modifica a estrutura sil\'abica da palavra, realizando uma palavra
monossil\'abica como dissil\'abica. A consequ\^encia disso \'e que o nativo se
depara com uma sequ\^encia de fones, {[}'bu.k?{]}, que n\~ao \'e prevista em
sua l\'ingua, e a ele, ent\~ao, cabe a tarefa de decodificar essa sequ\^encia
de fones, mapeando-a numa sequ\^encia que apresente um padr\~ao fon\'etico
similar e existente em sua l\'ingua, no caso, {[}'b?k{]}. Esse processo,
no entanto, nem sempre \'e efetivo. Em muitas vezes, o padr\~ao de pron\'uncia
apresentado pelo aprendiz \'e t\~ao distinto do esperado, que o interlocutor
\'e incapaz de decodificar a mensagem.

Al\'em disso, o preju\'izo na comunica\c{c}\~ao ocorre n\~ao apenas em processos
dial\'ogicos de aprendizes e nativos, mas tamb\'em entre aprendizes que
possuem l\'inguas-nativas distintas. Em um estudo de Major et al. (2002),
falantes n\~ao-nativos de ingl\^es foram avaliados em tarefas de listening,
ao ouvir \'audios de falantes nativos e de aprendizes com diferentes L1 de
background. O melhor desempenho na tarefa se deu quando os sujeitos eram
expostos a \'audios de falantes nativos. Os sujeitos tamb\'em desempenharam
melhor quando ouviam aprendizes que possu\'iam a mesma L1 (por exemplo,
chineses compreendiam melhor o ingl\^es falado por outros chineses, que o
ingl\^es falado por um espanhol). O problema do sotaque estrangeiro \'e que,
em muitas vezes, os aprendizes realizam tantos processos de
transfer\^encia de L1 para L2, que se torna dif\'icil a decodifica\c{c}\~ao da
mensagem, seja por um nativo ou por aprendiz que possua outra L1 como
base. A t\'itulo de exemplo, tome-se a pron\'uncia da palavra `smooth' por
brasileiros com pouca profici\^encia em ingl\^es. Devido à transfer\^encia de
padr\~oes fon\'etico-articulat\'orios e tamb\'em de correspond\^encia
grafo-fon\^emica, uma poss\'ivel pron\'uncia de `smooth', por esses
indiv\'iduos, seria {[}iz'mu.f?{]}, sendo que o padr\~ao esperado \'e
{[}'smu:ð{]}. Isto \'e, a sequ\^encia de fones \'e modificada quase por
completo, consequentemente, o grau de inteligibilidade da comunica\c{c}\~ao
decai, uma vez que se torna improv\'avel que o interlocutor seja capaz de
decodificar {[}'smu:ð{]} a partir de {[}iz'mu.f?{]}.

N\~ao bastasse isso, o sotaque estrangeiro afeta n\~ao apenas a
inteligibilidade do discurso, mas tamb\'em a forma como o indiv\'iduo \'e
percebido por seu interlocutor. Segundo Fuertes et al. (2002), o sotaque
tem fun\c{c}\~ao s\'ocio-cultural, impactando, em uma situa\c{c}\~ao de di\'alogo, na
representa\c{c}\~ao que os falantes criam uns dos outros, seja no que diz
respeito ao status do interlocutor (intelig\^encia, escolaridade, classe
social e \^exito profissional) ou de seu n\'ivel de solidariedade (simpatia,
confiabilidade e bondade).

Um maior n\'ivel de profici\^encia, portanto, \'e de interesse de modo a
facilitar a comunica\c{c}\~ao e a aumentar o n\'ivel de prest\'igio do aprendiz a
partir de seu sotaque. Caba\~nero e Alves (2008) ressaltam que, no que
concerne à aprendizagem de padr\~oes fon\'etico-fonol\'ogicos da l\'ingua-alvo,
a instru\c{c}\~ao expl\'icita facilita o processamento do input, sendo capaz de
tornar o aprendiz consciente da transfer\^encia, dessa forma, contribuindo
para uma diminui\c{c}\~ao do refor\c{c}o do padr\~ao de sua L1. Em outras palavras,
\'e preciso que o aprendiz seja informado do que, em sua pron\'uncia foge ao
padr\~ao, de forma a poder corrigi-la. No exemplo da Figura 2, o
brasileiro aprendiz necessita de ser informado, de forma expl\'icita,
sobre a altera\c{c}\~ao da qualidade voc\'alica de {[}?{]} para {[}u{]} e,
tamb\'em, da inser\c{c}\~ao da vogal final em sua pron\'uncia da palavra ``book''.
Somente assim ele ser\'a capaz de ter consci\^encia da exist\^encia do
fen\^omeno e, a partir disso, poder modificar sua pron\'uncia.

Celce-Murcia et al. (1996) prop\~oem que o ensino de pron\'uncia de L2 deve
ser constitu\'ido por cinco fases: (i) descri\c{c}\~ao e an\'alise; (ii) audi\c{c}\~ao
discriminativa; (iii) produ\c{c}\~ao controlada com feedback; (iv) produ\c{c}\~ao
guiada com feedback; (v) produ\c{c}\~ao em contexto comunicativo com feedback.
As duas fases iniciais dizem respeito à percep\c{c}\~ao do fen\^omeno pelo
aluno, as demais referem-se à sua realiza\c{c}\~ao. No quadro proposto pelas
autoras, o aprendizado inicia-se na descri\c{c}\~ao e an\'alise do fen\^omeno,
quando o aprendiz \'e posto em contato com textos que descrevem a
exist\^encia do fen\^omeno de pron\'uncia em quest\~ao, suas caracter\'isticas
ac\'ustico-articulat\'orias e, tamb\'em, o contexto em que ele ocorre. A
seguir, passa-se à audi\c{c}\~ao discriminativa do fen\^omeno. Nessa fase,
\'audios s\~ao apresentados ao aprendiz e a ele cabe a tarefa de discernir
em quais deles o fen\^omeno de pron\'uncia ocorre. Tendo o aprendiz
desenvolvido consci\^encia do fen\^omeno, iniciam-se as tr\^es fases de
produ\c{c}\~ao. A inten\c{c}\~ao \'e desenvolver, gradativamente, a capacidade do
aprendiz de produzir o fen\^omeno, partindo-se de um contexto controlado
(palavras e senten\c{c}as em isolamento), passando por atividades guiadas
(em que temas ou situa\c{c}\~oes de d\'ialogo s\~ao simuladas) at\'e chegar a
situa\c{c}\~oes comunicativas reais.

2 Ensino de Pron\'uncia Espec\'ifico para Falantes do Portugu\^es Brasileiro

\'E extensa a literatura existente para o ensino da pron\'uncia do ingl\^es,
em suas diversas variantes (Halliday, 1970; Jones, 1976; O'Connor, 1980;
Clifford, 1985; Kreidler, 1989; Ladefoged, 1993; Dalton \& Seidlhofer,
1994; Gilbert, 2000; Kenworthy, 2000; Staun, 2010; Ogden, 2012). Embora
haja um grande n\'umero de obras publicadas sobre o assunto, a grande
maioria dos trabalhos publicados desconsidera a l\'ingua nativa do
aprendiz e, consequentemente, todo o conhecimento lingu\'istico impl\'icito
que adv\'em desse fato (Crist\'ofaro-Silva, 2012).

As obras mencionadas s\~ao, em sua  maioria,  fruto  de  publica\c{c}\~oes  de
editoras com grande entrada no mercado internacional, que vendem o mesmo
livro de pron\'uncia, sem adapta\c{c}\~oes, seja no Brasil, na China, na Fran\c{c}a,
na Alemanha, na R\'ussia ou onde quer que seja. No entanto, dada a
diferen\c{c}a entre as l\'inguas, o conhecimento lingu\'istico impl\'icito que um
brasileiro possui \'e muito diferente daquele que um chin\^es falante de
Mandarim possui, por exemplo. A t\'itulo de ilustra\c{c}\~ao: o PB \'e uma l\'ingua
indo-europeia, rom\^anica, flexiva, com distin\c{c}\~ao de g\^enero morfol\'ogico
(masculino e feminino) e vogais nasais com status fon\^emico; por sua vez,
o Mandarim \'e uma l\'ingua sino-tibetana, chinesa, isolante, com seis tipos
de classificadores e tons com status fon\^emico (Weinberger, 2013; Lewis
et al., 2013). Dado este cen\'ario \'e natural que, caso um brasileiro e um
chin\^es decidam aprender ingl\^es, aspectos distintos da l\'ingua inglesa
devem ser enfatizados para cada um deles. Sendo assim, o ensino de
l\'ingua estrangeira precisa considerar o conhecimento lingu\'istico que o
falante j\'a possui em raz\~ao de sua l\'ingua nativa, buscando apresentar as
caracter\'isticas da l\'ingua adicional que s\~ao comuns à sua l\'ingua nativa e
enfatizar os aspectos que lhe s\~ao diferentes, a fim de aumentar a
capacidade comunicativa do aprendiz.

No Brasil, s\~ao poucos os trabalhos publicados na  \'area  de  ensino  de
pron\'uncia de ingl\^es que estabelecem um m\'etodo de ensino baseada no
conhecimento de l\'ingua que o falante do PB j\'a possui. Destacam-se as
iniciativas de Godoy et al. (2006), Zimmer et al. (2009) e
Crist\'ofaro-Silva (2012).

3 Fon\'etica, Fonologia e as Unidades de Descri\c{c}\~ao da Fala

A Fon\'etica e a Fonologia s\~ao os ramos de estudo da Lingu\'istica que
investigam a sonoridade da fala (Crist\'ofaro-Silva \& Yehia, 2008). A
fala humana \'e, por natureza, um fen\^omeno cont\'inuo. Quer se a tome em seu
n\'ivel de produ\c{c}\~ao articulat\'oria, por meio do estudo da movimenta\c{c}\~ao de
articuladores do trato vocal, quer em seu n\'ivel de percep\c{c}\~ao ac\'ustica,
por meio do estudo de varia\c{c}\~oes na press\~ao do ar, a fala sempre se
manifesta como um fen\^omeno que se desenvolve continuamente no tempo.
Todavia, a Fon\'etica e a Fonologia{[}6{]} postular\~ao que \'e poss\'ivel
descrev\^e-la por meio de unidades discretas.

A Fon\'etica baseia-se no estudo dos fones que comp\~oem a fala. Um fone
constitui a menor unidade discreta percept\'ivel do cont\'inuo da fala
(Crystal, 2008). Fones s\~ao unidades concretas, reais, que podem ser
descritas a partir de suas propriedades ac\'ustico-articulat\'orias.
Usualmente, os fones s\~ao representados atrav\'es dos s\'imbolos do Alfabeto
Fon\'etico Internacional (IPA, 2005), que arrola todos os sons capazes de
serem produzidos pelo aparelho fonador humano (Figura 3).

A t\'itulo de exemplo: a palavra ``ci\^encia'', tal como \'e usualmente
pronunciada no PB, poderia ser transcrita pela sequ\^encia de fones
{[}si'e?si?{]}{[}7{]}, isto \'e, uma consoante fricativa alveolar
desvozeada {[}s{]}; uma vogal alta anterior n\~ao-arredondada {[}i{]}; uma
vogal m\'edia-alta anterior n\~ao-arredondada nasalizada {[}e?{]}, com
acento prim\'ario; uma consoante fricativa alveolar desvozeada {[}s{]};
uma vogal alta anterior n\~ao-arredondada {[}i{]}; e, por fim, um schwa
{[}?{]}.


Figura 3: Alfabeto Fon\'etico Internacional (IPA, 2005).

A Fonologia, por sua vez, fundamenta-se no estudo dos fonemas da fala,
tendo por base o chamado princ\'ipio fon\^emico. Tal princ\'ipio foi
sistematizado por Swadesh (1934) e estabelece que:

cada l\'ingua possui um n\'umero limitado de sons elementares, os quais
recebem o nome de fonemas, que formam, em conjunto, o chamado invent\'ario
fonem\'atico das l\'inguas;

cada som produzido ao se falar possui correspond\^encia com um desses sons
elementares, com um fonema da l\'ingua;

os fonemas possuem valor significativo nas l\'inguas, isto \'e, possuem
capacidade de distinguir o significado das palavras.

A Fonologia prop\~oe que as l\'inguas do mundo s\~ao formadas por um
invent\'ario fechado de sons com valor significativo, os fonemas. Fonemas
s\~ao, portanto, unidades abstratas, n\~ao realizadas, que s\~ao descritas a
partir de sua fun\c{c}\~ao significativa em uma dada l\'ingua. O m\'etodo
utilizado para a identifica\c{c}\~ao \'e o ``teste da comuta\c{c}\~ao'' (Fages, 1967).
Basicamente, tal teste consiste em gerar pegar um determinado enunciado,
mudar, artificialmente, parte dele e observar se a mudan\c{c}a gerou tamb\'em
uma mudan\c{c}a no significado. Em outras palavras, consiste em modificar
fones de uma palavra, observar se com essa modifica\c{c}\~ao houve mudan\c{c}a no
significado e, ent\~ao, a partir dessa mudan\c{c}a ou n\~ao no significado,
concluir se os fones modificados s\~ao fonemas da l\'ingua.

Assim, para o portugu\^es brasileiro, o teste da comuta\c{c}\~ao prev\^e que o
fato de se poder mudar o sentido da palavra {[}'gat?{]} ao enunci\'a-la
trocando seu som inicial {[}g{]} por {[}p{]}, obtendo-se {[}'pat?{]},
implica que {[}g{]} e {[}p{]} t\^em valor distintivo na l\'ingua e que esses
sons s\~ao, portanto, fonemas, logo /g/ e /p/. Por outro lado, a
realiza\c{c}\~ao da palavra `mar' como {[}'mar{]} ou {[}'mah{]} n\~ao altera o
significado da palavra, tanto {[}'mar{]} quanto {[}'mah{]} referem-se à
mesma por\c{c}\~ao de \'agua. Infere-se, a partir disso, que, como a
substitui\c{c}\~ao de {[}r{]} por {[}h{]} n\~ao alterou o sentido da palavra,
n\~ao constituem dois fonemas distintos, mas sim dois alofones de um mesmo
fonema, no caso, o comum para o PB \'e consider\'a-los pertencentes ao
arquifonema /R/.

4 Descri\c{c}\~ao do Invent\'ario Fon\'etico do Portugu\^es Brasileiro e do Ingl\^es
Americano

Na estrutura do Listener, ser\'a assumido, para o ingl\^es americano, o
invent\'ario fon\'etico descrito por Ogden (2012); e para o PB, o descrito
por Crist\'ofaro-Silva (2005). Os Quadro 1 a 4 sintetizam o conjunto de
fones que tais autores prop\~oem para cada uma das l\'inguas.{[}8{]}

        Quadro 1: Fones consonantais do portugu\^es brasileiro

                                [pic]

Conforme se observa no Quadro 1, o invent\'ario fon\'etico consonantal do PB
\'e composto por 26 sons: seis oclusivos {[}p, b, t, d, k, g{]}, dois
africados {[}t?, d?{]}, tr\^es nasais {[}m, n, ?{]}, um tepe {[}?{]}, dez
fricativos {[}f, v, s, z, ?, ?, x, ?, h, ?{]}, dois aproximantes {[}j,
w{]} e dois laterais {[}l, ?{]}.

          Quadro 2: Fones consonantais do ingl\^es americano.

                                [pic]

De acordo com a classifica\c{c}\~ao proposta por Ogden (2012), que consta no
Quadro 2, as consoantes do ingl\^es podem ser descritas a partir de um
conjunto de 24 fones, a saber: seis oclusivos {[}p, b, t, d, k, g{]},
dois africados {[}t?, d?{]}, tr\^es nasais {[}m, n, ?{]}, nove fricativos
{[}f, v, s, z, ?, ?, h{]}, tr\^es aproximantes {[}r, j, w{]} e um lateral
{[}l{]}.

Segundo Crist\'ofaro-Silva (2005), as vogais do PB contabilizam quinze
segmentos, sendo dez orais {[}a, ?, ?, e, i, ?, ?, o, u, ?{]} e cinco
nasais {[}\~a, e?, i?, o?, u?{]} (Quadro 3).

              Quadro 3: Vogais do portugu\^es brasileiro

                                [pic]

A distribui\c{c}\~ao pode ser feita a partir de quatro n\'iveis, dados pela
posi\c{c}\~ao da l\'ingua. H\'a seis vogais com articula\c{c}\~ao alta {[}i, ?, i?, u,
?, u?{]}, quatro com m\'edia-alta {[}e, e?, o, o?{]}, duas com m\'edia-baixa
{[}?, ?{]} e tr\^es com articula\c{c}\~ao da l\'ingua em posi\c{c}\~ao baixa {[}a, ?,
\~a{]}. Todas as vogais anteriores e centrais {[}a, ?, \~a, ?, e, e?, i, ?,
i?{]} s\~ao n\~ao-arredondadas, j\'a as posteriores {[}?, o, o?, u, ?, u?{]}
s\~ao realizadas com arredondamento dos l\'abios.

                Quadro 4: Vogais do ingl\^es americano

                                [pic]

Para o ingl\^es americano, h\'a apenas vogais orais (Quadro 4). O n\'umero de
segmentos totaliza onze, sendo que destes cinco s\~ao longos {[}a?, ??,
i?, ??, u?{]} e seis breves {[}æ, ?, ?, ?, ?, ?{]}. H\'a quatro classes de
vogais, definidas a partir da posi\c{c}\~ao da l\'ingua: quatro vogais altas
{[}i?, ?, u?, ?{]}, uma m\'edia {[}?{]}, quatro m\'edias-baixas {[}?, ?, ??,
??{]} e duas baixas {[}æ, a?{]}.

  Quanto aos  ditongos,  o  PB  apresenta  um  total  de  vinte  e  tr\^es

ditongos, entre orais e nasais. H\'a onze ditongos orais decrescentes,
seis dos quais terminados {[}?{]}: {[}a?, ??, e?, ??, o?, u?{]}, e cinco
em {[}?{]}: {[}a?, ??, e?, o?, i?{]}. Os ditongos crescentes s\~ao sete,
quatro iniciados por {[}?{]}: {[}??, ?i, ?o, ??{]}, e tr\^es por {[}?{]}:
{[}??, ??, ?u{]}. Por sua vez, os ditongos nasais somam cinco: {[}\~a?,
??, \~o?, ??, \~a?{]}. No ingl\^es, h\'a apenas ditongos orais e todos
decrescentes. Tr\^es deles t\^em {[}?{]} como semivogal: {[}a?, e?, ??{]}, e
dois {[}?{]}: {[}a?, o?{]}.

5 Levantamento dos Desvios de Pron\'uncia

Na classifica\c{c}\~ao dos erros de pron\'uncia, deu-se prioridade,
especialmente, aos erros de pron\'uncia que afetam a compreens\~ao e que s\~ao
apresentados em trabalhos que consideram, no ensino da pron\'uncia do
ingl\^es, a transfer\^encia de padr\~oes sonoros de L1 para L2.

  A listagem dos erros de pron\'uncia a serem considerados  pelo  Listener

foi obtida a partir da consulta aos trabalhos de Zimmer (2004), Godoy
(2005), Zimmer et al. (2009) e Crist\'ofaro-Silva (2012). Tais trabalhos
analisam, de forma ampla, os aspectos de transfer\^encia de L1 para L2 que
afetam a pron\'uncia de brasileiros aprendizes de ingl\^es. No verificador
de pron\'uncia, optou-se por utilizar os nove tipos de erros elencados em
Zimmer et al. (2009), por se tratar, ao nosso ver, da investiga\c{c}\~ao mais
abrangente sobre o assunto. Os desvios de pron\'uncia selecionados est\~ao
descritos e exemplificados no Quadro 5.

{[}pic{]}

   Quadro 5: Desvios de pron\'uncias a ser analisados pelo Listener.

Nas Se\c{c}\~oes de 2.1.1.4.1 a 2.1.1.4.9, ser\'a apresentado, em maior n\'ivel de
detalhe, cada um desses desvios de pron\'uncia.

6 Simplifica\c{c}\~ao sil\'abica

Definimos como simplifica\c{c}\~ao sil\'abica os processos que ocorrem, na
interl\'ingua, de modo a simplificar encontros consonantais complexos,
atrav\'es da ep\^entese de {[}i{]} ou {[}?{]} e da consequente
ressilabifica\c{c}\~ao da s\'ilaba original. A simplifica\c{c}\~ao sil\'abica envolve os
seguintes contextos: quando /p/, /t/, /k/, /b/, /d/ ou /g/ ocupam
posi\c{c}\~ao de coda; e quando a palavra se inicia por um cluster do tipo
/sC/.

Na l\'ingua inglesa, todas as consoantes, exceto /h/, podem ocorrem em
posi\c{c}\~ao final de s\'ilaba ou palavra; comparativamente, no PB, apenas um
invent\'ario limitado de consoantes pode ocupar posi\c{c}\~oes finais s\'ilaba ou
palavra: /r/ e seus alofones, a lateral /l/, as nasais /m/, /n/ e /?/ e
as sibilantes /s/ e /z/ (Silveira 2012). N\~ao bastasse isso, no PB, esses
fonemas est\~ao sujeitos a processos fonol\'ogicos em contexto final de
s\'ilaba, de modo a limitar ainda mais a distribui\c{c}\~ao: /r/ pode ser
apagado ``sair'' {[}sa'i{]}, /l/ sofre vocaliza\c{c}\~ao ``sal'' {[}'saw{]},
as nasais nasalizam a vogal anterior e perdem seu tra\c{c}o consonantal
``som'' {[}'s\~o{]}, e a sibilante /z/ se torna desvozeada ``voz''
{[}'v?s{]}. Por essa raz\~ao, os aprendizes tendem a realizar, na
interl\'ingua, processos de simplifica\c{c}\~ao sil\'abica, de modo a evitar
consoantes n\~ao permitidas em coda no PB e, tamb\'em, encontros
consonantais tautossil\'abicos que n\~ao ocorrem em sua l\'ingua nativa. Post
(2010) refere-se à simplifica\c{c}\~ao sil\'abica como uma estrat\'egia de reparo:
os padr\~oes da L2 que s\~ao proibidos na L1 s\~ao alterados pelo aprendiz, na
interl\'ingua, de modo a condizerem com padr\~oes existentes na L1.

A simplifica\c{c}\~ao sil\'abica na interl\'ingua envolve a ep\^entese de {[}i{]} ou
{[}?{]} e a ressilabifica\c{c}\~ao da s\'ilaba original. Tome-se como exemplo a
palavra inglesa monossil\'abica ``dog'', cuja pron\'uncia can\^onica \'e
{[}'d?g{]}. Como a consoante {[}g{]} n\~ao ocorre em coda no PB, o
aprendiz acaba por inserir uma vogal epent\'etica no final da palavra e
por ressilabific\'a-la, realizando o diss\'ilabo {[}'d?.g?{]}. A pron\'uncia
{[}'d?.g?{]}, portanto, obedece aos padr\~oes fonot\'aticos do PB, que n\~ao
permitem a ocorr\^encia da consoante {[}g{]} em coda.
ê
Segundo Silveira (2012), a simplifica\c{c}\~ao sil\'abica ocorre n\~ao apenas de
modo a evitar consoantes proibidas em coda ou encontros consonantais
tautossil\'abicos, h\'a tamb\'em casos de simplifica\c{c}\~ao por transfer\^encia do
conhecimento de decodifica\c{c}\~ao letra-som de L1 para L2. Palavras com um
mesmo contexto fon\'etico na l\'ingua-alvo, como ``ham'' {[}'ham{]} e
``name'' {[}'ne?m{]}, tiveram realiza\c{c}\~oes distintas pelos aprendizes em
virtude do ortogr\'afico final. A primeira foi realizada com nasaliza\c{c}\~ao
da vogal anterior e perda do tra\c{c}o consonantal da consoante: {[}'h\~a{]},
enquanto a segunda foi realizada com ep\^entese da vogal {[}?{]} seguida
de ressibilafica\c{c}\~ao: {[}'ne?.m?{]}. Como na escrita do PB, um final
indica a realiza\c{c}\~ao da vogal {[}?{]}, o aprendiz transfere esse
conhecimento para a interl\'ingua e isso interfere na sua pron\'uncia.
Analisando palavras terminadas foneticamente em {[}m{]}, {[}n{]} e
{[}l{]}; e ortograficamente em , , ; Silveira (2012) constatou que cerca
de 10,1\% das realiza\c{c}\~oes continham ep\^entese (n = 930), sendo que as
palavras terminadas em \textless{}-e\textgreater{}, o percentual foi de
33,0\% (n = 130).

A baixa taxa de realiza\c{c}\~ao de simplifica\c{c}\~ao sil\'abica poderia ser
interpretada tendo em vista a popula\c{c}\~ao analisada. Os sujeitos
analisados por Silveira (2012) possu\'iam profici\^encia avan\c{c}ada em ingl\^es
e moravam, em m\'edia, havia 7,5 anos nos Estados Unidos. Todavia,
resultados semelhantes s\~ao descritos em Zimmer (2009), que analisou
casos de simplifica\c{c}\~ao sil\'abica por aprendizes de v\'arios n\'iveis de
profici\^encia, em tarefas de leitura de palavras e n\~ao-palavras. Zimmer
(2009) verificou que a simplifica\c{c}\~ao sil\'abica ocorreu em 7,9\% (n = 936)
dos dados. No n\'ivel iniciante, a simplifica\c{c}\~ao sil\'abica ocorreu em
16,7\% das realiza\c{c}\~oes dos sujeitos; j\'a no avan\c{c}ado, nenhum caso de
ep\^entese foi registrado.

Delatorre (2009) investigou casos de simplifica\c{c}\~ao sil\'abica no morfema
verbal regular de passado \{-ed\}{[}9{]}, com sujeitos de profici\^encia
intermedi\'aria em ingl\^es. Em tarefas de leitura, a ep\^entese ocorreu em
71,8\% das realiza\c{c}\~oes dos aprendizes (n = 1927); j\'a em situa\c{c}\~oes de
di\'alogo, a taxa foi de 61,8\% (n = 199). Como o morfema \{-ed\} envolve
outros processos fonol\'ogicos, al\'em da simplifica\c{c}\~ao sil\'abica, optamos
por trat\'a-lo separadamente, na Se\c{c}\~ao 2.1.1.4.9.

Rauber e Baptista (2004) tamb\'em constataram estrat\'egias de simplifica\c{c}\~ao
sil\'abica por aprendizes na realiza\c{c}\~ao de clusters consonantais iniciais
do tipo /sC(C)/, como em ``star'' {[}'st?r{]} ou ``strike''
{[}'str??k{]}. Como em PB n\~ao h\'a onsets complexos em in\'icio de palavra,
os aprendizes tendem a inserir um {[}i{]} epent\'etico antes de /s/,
transformando o encontro tautossil\'abico em heterossil\'abico: /sC/
\textgreater{} {[}is.C{]}. Sendo assim, ``star'' tende a ser realizado,
na interl\'ingua, como {[}is't?r{]} e ``strike'' como {[}is'tr??k{]}. No
estudo, as autoras reportaram uma taxa simplifica\c{c}\~ao sil\'abica de 29,0\%
(n = 866) para casos de /sC/, e de 38,6\% (n = 627) para casos de /sCC/.
Al\'em disso, elas indicaram que outros processos fonol\'ogicos tamb\'em foram
verificados nos dados, como o vozeamento de /s/ diante de consoantes
vozeadas, passando a {[}z{]}, a exemplo de ``small'' {[}iz'm?l{]}. Os
participantes foram estudantes de Letras do bacharelado em Ingl\^es, de
conhecimento intermedi\'ario a avan\c{c}ado da l\'ingua inglesa, os quais
cursavam o segundo ou o terceiro ano da gradua\c{c}\~ao.

Rebello e Baptista (2007) analisaram, tamb\'em,o contexto /sC(C)/ inicial,
todavia, reportaram taxas de ocorr\^encia do fen\^omeno consideravelmente
mais altas: 54,3\% para clusters iniciais do tipo /sC/ (n = 460) e
59,0\% para /sCC/ (n = 768). As diferen\c{c}as podem ser justificadas em
virtude da popula\c{c}\~ao analisada em cada um dos estudos, Rebello e
Baptista (2007) lidaram com sujeitos de profici\^encia mais baixa que
Rauber e Baptista (2004).

7 Substitui\c{c}\~ao consonantal

Denominamos substitui\c{c}\~ao consonantal os casos envolvendo a substitui\c{c}\~ao
do par de interdentais {[}?{]} e {[}ð{]} do ingl\^es, por {[}f{]},
{[}v{]}, {[}s{]}, {[}z{]}, {[}t{]}, {[}d{]} ou correspondentes; e,
tamb\'em, a substitui\c{c}\~ao da aproximante {[}?{]} por um r\'otico an\'alogo no
PB: {[}x{]}, {[}?{]}, {[}h{]}, {[}?{]} ou {[}?{]}.

A substitui\c{c}\~ao consonantal de {[}?{]} e {[}ð{]} ocorre em raz\~ao de tais
fones inexistirem no invent\'ario fon\'etico do PB, dessa maneira, o
aprendiz tende a perceber e a produzir esses sons pelo vi\'es de sua
l\'ingua nativa, o invent\'ario fon\'etico do PB. A articula\c{c}\~ao de {[}?{]} e
{[}ð{]} \'e considerada complexa n\~ao apenas por aprendizes de ingl\^es como
l\'ingua estrangeira. Vihman (1996) pesquisou a aquisi\c{c}\~ao fonol\'ogica do
ingl\^es por crian\c{c}as norte- americanas, tendo constatado que as
interdentais {[}?{]} e {[}ð{]} constituem o par de fones que as crian\c{c}as
mais demoram a adquirir, dada sua complexidade articulat\'oria. Seguindo a
Teoria da Marca\c{c}\~ao de Eckman (1977), as interdentais {[}?{]} e {[}ð{]}
podem ser consideradas fones marcados, uma vez que s\~ao pouco frequentes
nas l\'inguas do mundo e disso adv\'em a dificuldade de articul\'a-las.

\'E interessante ressaltar que o aprendiz brasileiro de ingl\^es mapeia
{[}?{]} e {[}ð{]} em fones j\'a existentes no PB n\~ao de modo aleat\'orio,
mas de modo a maximizar a semelhan\c{c}a ac\'ustica e articulat\'oria. As
consoantes {[}?{]} e {[}ð{]} s\~ao fricativas interdentais, e como se
mostrar\'a a seguir, elas tendem a ser substitu\'idas por consoantes do PB
que mant\^em o mesmo modo e/ou ponto de articula\c{c}\~ao, a exemplo de outras
fricativas anteriores, como as labiodentais {[}f{]} e {[}v{]}, ou as
alveolares {[}s{]} e {[}z{]}; ou a exemplo das oclusivas alveolares
{[}t{]} e {[}d{]}.

Schadech e Silveira (2013) avaliaram o quanto a produ\c{c}\~ao de {[}?{]} e
{[}ð{]} por aprendizes afeta a inteligibilidade da mensagem por nativos.
As autoras realizaram um experimento em que tocaram grava\c{c}\~oes de
brasileiros aprendizes de ingl\^es, pronunciando palavras contendo {[}?{]}
e {[}ð{]}, para dez falantes nativos de ingl\^es. Muitas das grava\c{c}\~oes
continham pron\'uncias com influ\^encia de L1 em L2, de maneira que os
aprendizes substitu\'iam {[}?{]} e {[}ð{]} por {[}f{]}, {[}v{]}, {[}s{]},
{[}z{]}, {[}t{]} ou {[}d{]}. O grau de inteligibilidade foi mensurado
pelos nativos atrav\'es de question\'arios, em que deviam marcar, em uma
escala variando de ``muito f\'acil'' a ``muito dif\'icil'', qual o grau de
inteligibilidade da grava\c{c}\~ao. Os resultados indicaram que, de acordo com
os nativos, a substitui\c{c}\~ao de {[}?{]} tem mais impacto na
inteligibilidade que a substitui\c{c}\~ao {[}ð{]}: {[}?{]} foi classificado
como de compreens\~ao ``n\~ao muito f\'acil'' e de {[}ð{]} como ``f\'acil''.

Reis (2006) investigou a produ\c{c}\~ao das interdentais {[}?{]} e {[}ð{]} por
aprendizes de profici\^encia intermedi\'aria-baixa e avan\c{c}ada, em tarefas de
leitura de senten\c{c}as, textos e retelling, constatando baixas taxas de
acerto para ambos os fones. Considerando as tr\^es tarefas, os sujeitos de
n\'ivel intermedi\'ario-baixo realizaram {[}?{]} em 16,6\% dos contextos (n
= 489) e {[}ð{]} em apenas 0,1\% dos casos (n = 494). J\'a os de n\'ivel
avan\c{c}ado conseguiram produzir corretamente {[}?{]} em 41,3\% dos casos
(n = 499) e {[}ð{]} em 7,5\% (n = 610). Embora os resultados tenham se
mostrado estatisticamente significativos, a autora salienta que as
baixas taxas de realiza\c{c}\~ao de {[}ð{]} podem ter sido enviesadas em
virtude do n\'umero reduzido de participantes no estudo: havia 16
informantes de profici\^encia intermedi\'aria-baixa e 8 de avan\c{c}ada. De todo
modo, ainda que n\~ao se possa ter precis\~ao sobre o percentual de acerto
dos fones, os resultados indicam que os aprendizes t\^em altas taxas de
erro na produ\c{c}\~ao das interdentais e que se trata, portanto, de uma
dificuldade de pron\'uncia dos aprendizes brasileiros. Reis (2006)
verificou substitui\c{c}\~oes de {[}?{]} por {[}t{]}, {[}f{]}, {[}d{]},
{[}t?{]}, {[}s{]} e {[}t?{]}; sendo as mais frequentes {[}t{]} (45,8\%),
{[}t?{]} (7,5\%) e {[}f{]} (6,9\%); e substitui\c{c}\~oes de {[}ð{]} por
{[}d{]}, {[}t?{]}, {[}d?{]}, {[}d?{]}, {[}t?{]}, {[}?{]} e {[}t{]}; as
mais frequentes {[}d{]} (85,6\%) e {[}t?{]} (1,4\%){[}10{]}.

Trevisol (2010) realizou um estudo sobre a produ\c{c}\~ao da interdental
vozeada {[}ð{]} por professores de ingl\^es. O experimento consistiu da
leitura de 20 frases, as quais continham o fone {[}ð{]} em in\'icio e
final de palavra. Mesmo nessa popula\c{c}\~ao de n\'ivel avan\c{c}ado em ingl\^es, as
interdentais se mostram como uma dificuldade de pron\'uncia. Em in\'icio de
palavra, os sujeitos produziram corretamente {[}ð{]} em 51,4\% das
vezes, tendo trocado {[}ð{]} por {[}d{]} nos demais 48,6\% casos (n =
220). Em final de palavra, a taxa de acerto verificada foi
consideravelmente menor: 26,0\% (n = 208). O maior n\'umero de
substitui\c{c}\~oes deu-se pela correspondente desvozeada da interdental
{[}?{]}, que contabilizou 65,5\% das realiza\c{c}\~oes (n = 208); a seguir, o
maior n\'umero de substitui\c{c}\~oes foi pela oclusiva {[}d{]}, com 6,0\% dos
casos (n = 208). Trevisol (2010), tamb\'em registrou substitui\c{c}\~oes por
{[}v{]}, {[}f{]}, {[}t{]}, {[}t?{]} e {[}Ø{]}, no entanto, todas elas
s\~ao de baixa ocorr\^encia (\textless{}1,0\%) e podem ser consideradas
esp\'urias. A autora explica a predomin\^ancia de {[}?{]} em final de
palavra, em virtude de haver restri\c{c}\~oes, no PB, para que sons fricativos
sejam vozeados em final de palavra. Esse processo ser\'a tratado à parte,
na Se\c{c}\~ao 2.1.1.4.4.

No que diz respeito ao {[}?{]}, tal fone constitui uma aproximante
alveolar vozeada e est\'a presente em diversos dialetos do PB.
Equivocadamente, a literatura lingu\'istica no Brasil convencionou a
chamar tal som de ``r retroflexo'', embora se trate, em termos
articulat\'orios, de uma aproximante (Rennicke, 2011). A Figura 4
apresenta um mapa da distribui\c{c}\~ao dos r\'oticos no Brasil, indicando as
regi\~oes em que ocorre o {[}?{]}.

                                [pic]

Figura 4: Distribui\c{c}\~ao geogr\'afica dos sons r\'oticos em coda no Brasil -
Rennicke (2011) com base em Noll (2008).

Como se nota, a maior parte dos dialetos que cont\'em a aproximante
{[}?{]} est\'a concetrada na regi\~ao Centro-Sul do Brasil. Apesar disso,
Rennicke (2011) afirma haver estudos que indicam a presen\c{c}a da
aproximante {[}?{]}, em menor concentra\c{c}\~ao, em quase todas as regi\~oes do
pa\'is. Para os dialetos que possuem a aproximante {[}?{]} como parte do
invent\'ario fon\'etico, sua percep\c{c}\~ao e produ\c{c}\~ao na interl\'ingua n\~ao
constitui problema, de forma que os aprendizes conseguem, por exemplo,
pronunciar car {[}'k??{]} e word {[}'w??d{]} sem dificuldades.

No entanto, nos dialetos do PB em que {[}?{]} n\~ao ocorre, os aprendizes
t\^em mais um obst\'aculo a vencer no aprendizado da l\'ingua inglesa.
Geralmente, eles acabam por realizar substitui\c{c}\~oes, na interl\'ingua,
mapeando a aproximante {[}?{]} em um r\'otico an\'alogo no PB: {[}x{]},
{[}?{]}, {[}h{]}, {[}?{]} ou {[}?{]} (Zimmer, 2009).

Osborne (2010) pesquisou a aquisi\c{c}\~ao da aproximante {[}?{]} por tr\^es
brasileiros aprendizes de ingl\^es, com conhecimento de ingl\^es em n\'ivel
iniciante. O experimento consistiu na leitura de senten\c{c}as em voz, as
quais continham a aproximante {[}?{]} em diversos contextos. Para onset
complexos, a exemplo da palavra travel {[}'træv.?l{]}, {[}?{]} foi
realizado como {[}?{]} em 71,7\% dos casos, como {[}?{]} em 26,4\% e
omitido em 1,9\% (n = 53). Em posi\c{c}\~ao intervoc\'alica, como America
{[}??mer.?.k?{]}, {[}?{]} foi realizado como {[}?{]} em 51,7\% das vezes
e como {[}?{]} nos demais 48,3\% (n = 29). Em posi\c{c}\~ao de coda, como em
park {[}'p??rk{]} e war {[}'w??r{]}, a aproximante {[}?{]} apresenta o
maior n\'umero de varia\c{c}\~ao, sendo realizada ora como {[}?{]}, {[}?{]},
{[}x{]}, {[}h{]} ou sendo apagado. Osborne (2010) analisa os dados de
coda em duas situa\c{c}\~oes: em meio e final de palavra. No que diz respeito
ao meio de palavra, os resultados obtidos com o {[}?{]} foram: {[}h{]}
57,6\%; {[}?{]} 18,2\%; apagamento 15,1\%; {[}x{]} 6,1\%; e {[}?{]}
3,0\% (n = 33). Em final de palavra, as realiza\c{c}\~oes s\~ao similares, mas
n\~ao se nota a ocorr\^encia do tepe {[}?{]}: apagamento 52,5\%; {[}?{]}
27,5\% {[}h{]} 15,0\% e {[}x{]} 5,0\% (n = 40). O autor tamb\'em avaliou a
realiza\c{c}\~ao do {[}h{]} em ingl\^es, em palavras como huge {[}'hju?d?{]}, e
do tepe {[}r{]}, como em city {[}?s??.i{]}; no entanto, nenhuma varia\c{c}\~ao
foi observada, tendo os aprendizes produzido o padr\~ao esperado em 100\%
dos casos.

8 Falta de aspira\c{c}\~ao de oclusivas em posi\c{c}\~ao de in\'icio de palavra ou
s\'ilaba acentuada

Definimos como falta de aspira\c{c}\~ao de oclusivas a substitui\c{c}\~ao das
consoantes {[}p?{]}, {[}t?{]} e {[}k?{]} do ingl\^es, em posi\c{c}\~ao de in\'icio
de palavra ou s\'ilaba acentuada, por suas correspondentes n\~ao-aspiradas
{[}p{]}, {[}t{]} e {[}k{]}.

A aspira\c{c}\~ao \'e um fen\^omeno restrito às consoantes obstruintes. Uma
consoante \'e considerada aspirada quando \'e articulada de modo que, ap\'os a
fase de explos\~ao dos articuladores, segue-se a libera\c{c}\~ao de um sopro de
ar. Desde Lisker e Abramson (1964), estudos envolvendo compara\c{c}\~oes entre
segmentos vozeados, desvozeados e aspirados t\^em sido realizados a partir
de medidas de voice onset time (VOT). O VOT consiste no intervalo e
entre a explos\~ao dos articuladores da consoante e o in\'icio do vozeamento
da glote. A Figura 5 ilustra as combina\c{c}\~oes de valores de VOT em uma
oclusiva bilabial.

                                [pic]

    Figura 5: Tipos de fona\c{c}\~ao e valores correspondentes de VOT.

Como se observa, assume-se como ponto de refer\^encia, ou ponto zero, a
soltura dos articuladores da consoante, a partir disso, calculam-se os
valores de VOT. Quando h\'a um intervalo entre a soltura dos articuladores
e o in\'icio do vozeamento da glote, isto \'e, VOT \textgreater{} 0, a
consoante \'e classificada como aspirada, no exemplo: {[}p?{]}. Quando o
vozeamento se inicia imediatamente ap\'os a soltura dos articuladores, no
caso de VOT = 0, a consoante \'e desvozeada: {[}p{]}. Por fim, quando o
vozeamento antecede a explos\~ao, ou seja, quando h\'a pr\'e-sonoriza\c{c}\~ao, VOT
\textless{} 0, a consoante \'e vozeada: {[}b{]}.

A rela\c{c}\~ao entre as medidas de VOT e tipo de fona\c{c}\~ao \'e dependente de
l\'ingua, por exemplo, um fone que tenha um determinado valor de VOT pode
ser considerado aspirado em uma l\'ingua, mas desvozeado em outra. A

., baseada nos dados de Yang (1993), apresenta os valores m\'edios de VOT
para as consoantes oclusivas do ingl\^es.

        Tabela 1: Valores m\'edios de VOT (em ms) de consoantes
                  oclusivas no ingl\^es (Yang 1993).

                                [pic]

Yang (1993) opta por reportar os valores de VOT para obstruintes
vozeadas na forma A/B, em que A cont\'em a m\'edia das amostras com valor
negativo de VOT e B as com positivo. O autor argumenta que a diferen\c{c}a
de sinal no VOT representa fen\^omenos diferentes: um VOT \textless{} 0
corresponde à pr\'e-sonoriza\c{c}\~ao, j\'a um VOT \textgreater{} 0 corresponde à
p\'os- sonoriza\c{c}\~ao. Tais fen\^omenos, de acordo com Yang (1993) devem ser
tratados distintamente. A partir Tabela 1, nota-se que, na l\'ingua
inglesa, o VOT est\'a relacionado ao lugar de articula\c{c}\~ao da consoante,
sendo que o par de alveolares {[}t?{]} e {[}d{]} apresenta os maiores
valores m\'edios, respectivamente 95ms e 20/-91ms. As oclusivas velares
{[}k?{]} e {[}g{]} seguem com 88ms e 32/- 78ms. Por fim, os menores
valores m\'edios de VOT s\~ao encontrados nas bilabiais {[}p?{]} e {[}b{]},
77ms e 17/-78ms.

Para o PB, a an\'alise mais extensa de VOT de que temos not\'icia foi
realizada por Klein (1999). Os resultados est\~ao resumidos na Tabela 2.

        Tabela 2: Valores m\'edios de VOT (em ms) de consoantes
                    oclusivas no PB (Klein 1999).

                                [pic]

Como se nota, no PB, as consoantes desvozeadas apresentam menores
valores m\'edios de VOT que no ingl\^es, havendo, portanto, menos aspira\c{c}\~ao.
Al\'em disso, a correla\c{c}\~ao entre o lugar de articula\c{c}\~ao da consoante e os
valores VOT \'e mais atenuada, a diferen\c{c}a de VOT entre {[}p{]} e {[}t{]}
\'e estatisticamente insignificante, sendo que apenas a oclusiva velar
{[}k{]} apresenta algum n\'ivel de aspira\c{c}\~ao. Ao se comparar as medidas de
VOT entre os pares do PB e do ingl\^es {[}p?{]} vs. {[}p{]}, {[}t?{]} vs.
{[}t{]}, e {[}k?{]} vs. {[}k{]}, \'e poss\'ivel ver que as consoantes
desvozeadas da l\'ingua inglesa apresentam valores bem mais altos de VOT,
o que evidencia, de fato, sua aspira\c{c}\~ao.

Salienta-se que, diferentemente de Yang (1993), Klein (1999) agrupa os
valores de VOT positivos e negativos das consoantes vozeadas, de forma
que a m\'edia apresentada das vozeadas acaba por se tornar menor. Ainda
assim, \'e poss\'ivel notar que as oclusivas vozeadas do PB apresentam mais
pr\'e- sonoriza\c{c}\~ao que as do ingl\^es. A maior diferen\c{c}a \'e verificada na
oclusiva velar, no PB, seu valor m\'edio \'e de -91ms, enquanto no ingl\^es \'e
de 32/-78ms. A seguir a maior diferen\c{c}a \'e notada na bilabial {[}b{]},
-87ms vs. 17/-78ms; por fim, os menores valores s\~ao encontrados na
alveolar {[}d{]}, -99 vs. 20/- 91ms. Por conseguinte, conclui-se que as
oclusivas vozeadas {[}b{]}, {[}d{]} e {[}g{]} do PB possuem mais
vozeamento que as do ingl\^es.

Al\'em disso, \'e poss\'ivel observar certa sobreposi\c{c}\~ao entre os valores
positivos de VOT das oclusivas vozeadas no ingl\^es e das oclusivas
desvozeadas no PB. Por exemplo, o valor de VOT positivo de {[}b{]} no
ingl\^es, 18ms, \'e bastante similar ao valor de {[}p{]} no PB, 17ms. Isso \'e
v\'alido tamb\'em para os demais lugares de articula\c{c}\~ao: o valor de VOT da
alveolar vozeada {[}d{]} do ingl\^es, 20 ms., \'e muito similar ao da
desvozeada {[}t{]} do PB, 17 ms.; e o mesmo para as velares {[}g{]} e
{[}k{]}, 32ms vs.~38ms. Em outras palavras, as consoantes desvozeadas no
PB s\~ao articuladas, em certos contextos, de modo muito similar às
vozeadas no ingl\^es. Isso pode trazer problemas na inteligibilidade da
pron\'uncia do aprendiz, uma vez que, caso ele transfira esse padr\~ao para
a interl\'ingua, pode, por exemplo, tentar pronunciar {[}k{]} e acabar
sendo entendido como {[}g{]}, em palavras como caught {[}'k??t{]} e got
{[}'g??t{]}; coat {[}'ko?t{]} e goat {[}'go?t{]}, etc. Cabe, portanto,
ao aprendiz dominar essa diferen\c{c}a, de modo a aumentar a
inteligibilidade de sua pron\'uncia.

Alves (2011) investigou a produ\c{c}\~ao das oclusivas aspiradas {[}p?{]},
{[}t?{]} e {[}k?{]} por brasileiros aprendizes de ingl\^es. A autora
utilizou valores de VOT na classifica\c{c}\~ao, tendo definido como aspirados
os segmentos que apresentavam VOT \textgreater{} 60ms e n\~ao-aspirados os
que apresentavam VOT \textless{} 35ms. A autora observou que {[}p?{]}
foi realizado sem aspira\c{c}\~ao pelos aprendizes em 59\% das vezes (n = 41),
{[}t?{]} em 33\% das vezes (n = 51) e {[}k?{]} em 10\% das vezes (n =
96). A diferen\c{c}a no desempenho pode ser explicada em virtude do m\'etodo
de classifica\c{c}\~ao utilizado: apenas duas faixas de valores, VOT
\textless{} 35ms ou VOT \textgreater{} 60ms; e tamb\'em em virtude de, no
PB, a consoante {[}k{]} j\'a possuir maior VOT, dado o lugar de
articula\c{c}\~ao. Entretanto, o estudo contou com um n\'umero muito reduzido de
participantes (tr\^es), de modo que os resultados obtidos devem ser
considerados apenas indicativos e n\~ao conclusivos.

Prestes (2012) investigou a produ\c{c}\~ao de oclusivas surdas e sonoras por
aprendizes brasileiros e falantes nativos de ingl\^es. A autora emprega
medidas de VOT na classifica\c{c}\~ao dos segmentos e adota uma postura de
tratar o fen\^omeno em sua gradi\^encia, isto \'e, considerando-se apenas as
medidas de VOT, sem dizer se uma determinada consoante \'e ou n\~ao
aspirada. Valores relativos de VOT foram utilizados na an\'alise, mais
especificamente, a raz\~ao entre a dura\c{c}\~ao do VOT e o tempo total da
consoante. Prestes (2012) concluiu que as realiza\c{c}\~oes surdas dos
aprendizes apresentaram menor VOT (5,87\% VOT/consoante) em rela\c{c}\~ao às
dos nativos (2,11\% VOT/consoante); e que as vozeadas dos aprendizes
apresentam maior VOT (10,15\% VOT/consoante) em compara\c{c}\~ao com as dos
nativos (3,89\% VOT/consoante) (n = 450). Os resultados indicam,
portanto, que brasileiros tendem a realizar {[}p{]}, {[}t{]} e {[}k{]},
na interl\'ingua, com menos aspira\c{c}\~ao que falantes nativos; e {[}b{]},
{[}d{]} e {[}g{]} com maior grau de vozeamento. Ressalta-se, todavia,
que o estudou tamb\'em utilizou um n\'umero reduzido de participantes, dois
brasileiros aprendizes de ingl\^es e dois nativos.

Schwartzhaupt (2012) analisou o impacto que fatores fon\'etico-fonol\'ogicos
podem ter sobre os valores de VOT, seu experimento foi conduzido com dez
brasileiros aprendizes de ingl\^es e cinco nativos. Foram analisados os
seguintes fatores fon\'etico-fonol\'ogicos: (i) lugar de articula\c{c}\~ao da
consoante, (ii) qualidade da vogal adjacente e (iii) o n\'umero de s\'ilabas
da palavra-alvo. Os aprendizes possu\'iam conhecimento
intermedi\'ario-avan\c{c}ado ou avan\c{c}ado de ingl\^es. Apesar de o foco do estudo
n\~ao ser a produ\c{c}\~ao dos aprendizes, mas os efeitos dos contextos
fon\'etico-fonol\'ogicos no VOT, Schwartzhaupt (2012) observou que os
aprendizes foram capazes de realizar, na interl\'ingua, valores de VOT bem
pr\'oximos aos dos nativos. Ele salienta o fato de os aprendizes terem
conseguido partir de um sistema sem distin\c{c}\~ao de VOT entre {[}p{]} e
{[}t{]}, como \'e o caso do PB, e terem alcan\c{c}ado, na interl\'ingua, um
sistema em que tal distin\c{c}\~ao \'e patente, como \'e o caso do ingl\^es. Na
interl\'ingua, as realiza\c{c}\~oes de {[}t?{]} (77ms) dos aprendizes
apresentaram VOT significativamente maior que as de {[}p?{]} (61ms), tal
como ocorre com falantes nativos.

9 Desvozeamento de obstruintes em posi\c{c}\~ao de final de palavra

O m Xxxxx

10 Vocaliza\c{c}\~ao de laterais em final de s\'ilaba

Baratieri (2006) investigou a produ\c{c}\~ao da lateral {[}l{]} em coda por um
grupo de vinte estudantes de ingl\^es como l\'ingua adicional, de
profici\^encia intermedi\'aria e avan\c{c}ada. Os resultados obtidos indicaram
que a vocaliza\c{c}\~ao se trata de um fen\^omeno gradiente, tendo o autor
optado por classificar os dados em tr\^es categoriais: (a) segmento
parcialmente vocalizado; (b) vocalizado {[}w{]}; e (c) n\~ao vocalizado
{[}l{]}. A categoria (a) indica os segmentos para os quais o grau de
vocaliza\c{c}\~ao n\~ao \'e imediatamente percept\'ivel, de maneira que n\~ao h\'a um
s\'imbolo IPA correspondente. As produ\c{c}\~oes dos aprendizes apresentaram a
seguinte distribui\c{c}\~ao: 2,7\% de {[}l{]}, 35,5\% de {[}w{]} e 61,8\% de
segmentos parcialmente vocalizados (n = 2134). Como se observa, a taxa
de produ\c{c}\~ao do padr\~ao esperado {[}l{]} foi extremamente baixa, indicando
que a vocaliza\c{c}\~ao de laterais perdura mesmo em estudantes de
profici\^encia intermedi\'aria ou avan\c{c}ada.

11 Quada da nasal em final de s\'ilaba + nasaliza\c{c}\~ao da vogal precedente

Kluge e Baptista (2008) estudaram a produ\c{c}\~ao das nasais {[}m{]} e
{[}n{]} em posi\c{c}\~ao final de palavra por um grupo de dez aprendizes de
n\'ivel intermedi\'ario de profici\^encia. A tarefa consistiu na leitura de
senten\c{c}as em voz alta. O corpus foi formado por 72 senten\c{c}as, 36
contendo {[}m{]} e 36 contendo {[}n{]}, em posi\c{c}\~ao final de monoss\'ilabo
acentuado do tipo (C)CVC. Os estudantes realizaram a nasal alveolar
{[}n{]} final esperada em 78,6\% dos casos (n = 359), e a nasal bilabial
{[}m{]} esperada em 63,9\% dos casos (n = 357); nos demais, houve
apagamento da nasal e a nasaliza\c{c}\~ao da vogal precedente. Testes
estat\'isticos indicaram que a diferen\c{c}a de acerto nas realiza\c{c}\~oes de
{[}m{]} e {[}n{]} \'e significativa, havendo, portanto, mais dificuldade
na aquisi\c{c}\~ao do {[}m{]} final pelos aprendizes. As autoras justificam
que esse resultado pode se dar em virtude de, no PB, as palavras que
terminam em nasal tenderem a ser escritas com (ex.: ``fim'', ``correm'',
``amam''), havendo poucas palavras com (ex.: ``h\'ifen'', ``p\'olen'',
``abd\^omen''). Sendo assim, o padr\~ao com \'e refor\c{c}ado e o aprendiz
apresenta mais resist\^encia para super\'a-lo na interl\'ingua.

12 Paragoge da consoante oclusiva velar vozeada {[}g{]}

O Xxxxx

13 Assimila\c{c}\~ao voc\'alica

O Xxxxx

14 Ep\^entese interconsonantal (morfema -ed)

O Xxxxx


\section{Why Consider L1 in L2 Teaching?}

Lorem ipsum at nusquam appellantur his, ut eos erant homero
concludaturque. Albucius appellantur deterruisset id eam, vivendum
partiendo dissentiet ei ius. Vis melius facilisis ea, sea id convenire
referrentur, takimata adolescens ex duo. Ei harum argumentum per. Eam
vidit exerci appetere ad, ut vel zzril intellegam interpretaris.

Errem omnium ea per, pro \ac{UML} congue populo ornatus cu, ex qui
dicant nemore melius. No pri diam iriure euismod. Graecis eleifend
appellantur quo id. Id corpora inimicus nam, facer nonummy ne pro,
kasd repudiandae ei mei. Mea menandri mediocrem dissentiet cu, ex
nominati imperdiet nec, sea odio duis vocent ei. Tempor everti
appareat cu ius, ridens audiam an qui, aliquid admodum conceptam ne
qui. Vis ea melius nostrum, mel alienum euripidis eu.

\section{Common Mispronunciation}\label{sec:common-mispronunciation}

\subsection{Syllable Simplification}
Lorem ipsum

\subsection{Consonant Change}
Lorem ipsum

\subsection{Deaspiration of Voiceless Plosives in Initial or Stressed Positions}
Lorem ipsum

\subsection{Terminal Devoicing in Word-Final Obstruents}
Lorem ipsum

\subsection{Delateralization and rounding of lateral liquids in final position}
Lorem ipsum

\subsection{Vocalization of final nasals}\label{sec:voc-nasals}
Lorem ipsum

\subsection{Velar consonantal paragoge}
Lorem ipsum

\subsection{Vowel assimilation}\label{sec:voc-assimilation}
Lorem ipsum

\subsection{Interconsonantal epenthesis (-ed and -s morphemes)}
Lorem ipsum

