% ******************************* Thesis Appendix A ****************************
\chapter{Aeiouad\^o: Syllabification Rules} 

The syllabification rules developed for Aeiouad\^o are based on the premise that postvocalic sounds in Portuguese are limited to a very small set due to its parataxis. The parataxis of Portuguese allows for syllables with up to five phonemes $C_1C_2VC_3C_4$, as in ``transporte'' /traNs.p\ipa{O}R.te/, whereas $C_3$ can be /\ipa{l, N, R, s}/ and $C_4$ is always /\ipa{s}/. As in Portuguese the sound-letter relation is mostly transparent and rules for translineation are deeply influenced by phonological syllables, the algorithm analyses graphemes trying to define which are the prevocalic elements, i.e. the syllable onset. \autoref{pseudocode-syll-rules} contains the pseudocode with the 10 cases that are taken into consideration\footnote{The code can be found in the followign github repository: \url{http://github.com/gustavoauma/aeiouado_g2p}}.


\begin{figure}[!ht]
  \caption{Syllabification Rules.}
      \label{pseudocode-syll-rules}
\footnotesize

\begin{verbatim}

word  # Input word.
hyphen_pos = []  # Positions to append hyphens.

# Regex with relevant contexts
all_vowels = [aáàâãeéêiíoóôõuúü]
vowels_wo_diacritics = [aãeioõuü]
vowels_with_diacritics = [áàâéêíóôúãõ]
low_vowels = [aáàâeéêoóô]  # No nasal.
high_vowels = [iíuú]
consonants = [bcçdfghjklmnpqrstvwxyz]
plosives = [ptckbdgfv]
digraphs = nh|lh|ch|qu|gu
liquids = [lr]

def get_hyphen_pos(word):

  for i in word:
  
    # window1: Cases 1st-4th
    window1 = <previous 2 and current char>
    
    # window2: Cases 5th-7th
    window2 = <previous 1, current and next 1 chars>
    
    # window3: Cases 8th-10th
    window3 = <previous 1, current and next 2 chars>
    
    # 1st Case: complex onsets
    # cabrita -> cabr{i}ta -> ca[br]ita -> ca[-br]ita 
    if window1 has (plosives + liquids):
      hyphen_pos.append(i-2)]
      continue
    
    # 2nd Case: Digraphs 'nh|lh|ch|qu|gu'
    # linhagem -> linh{a}gem -> li[nh]agem -> li[-nh]agem  
    if window1 has digraphs:
      hyphen_pos.append(i-2)
      continue
      
    # 3rd Case: Simple onset (after vowel)
    # batata -> bat{a}ta -> b[at]ata -> b[a-t]ata
    if window1 has (all_vowes + consonants):
      hyphen_pos.append(i-1)
      continue
    
    # 4th Case: Simple onset (after consonant)
    # perspectiva -> persp{e}ctiva -> per[sp]ectiva -> per[s-p]ectiva
    if window1 has (consonants + consonants):
      hyphen_pos.append(i-1)
      continue

\end{verbatim}
\end{figure}
\clearpage

\footnotesize
\begin{verbatim}
  
    # 5th Case: vowel without diacritic + vowel with diacritic
    # paraíso -> para{í}so -> par[aí]so -> par[a-í]so
    if window2 has (vowels_wo_diacritics + vowels_with_diacritics):
      hyphen_pos.append(i)
      continue
    
    # 6th Case: low vowel + low vowel
    # coador -> co{a}dor -> c[oa]dor -> c[o-a]dor
    if window2 has (low_vowels + low_vowels):
      hyphen_pos.append(i)
      continue
    
    # 7th Case: high vowel + low vowel
    # dicionario -> dici{o}nario -> dic[io]nario -> dic[i-o]nario 
    if window2 has (high_vowels + low_vowels):
      hyphen_pos.append(i)
      continue
    
    # 8th Case: hyathus before 'nh'
    # rainha -> ra{i}nha -> r[ainh]a -> r[a-inh]a
    # Exception: words with 'gu|qu'
    # pouquinho -> pouqu{i}nho -> pouq[uinh]o -> pouq[uinh]o
    if window3 has (all_vowels + high_vowels + 'nh'):
      hyphen_pos.append(i)
      continue

    # 9th Case: vowel + high vowel + liquid + (cons|#)
    # abstrair -> abstra{i}r -> abstr[air#] -> abstr[a-ir#]
    if window3 has (all_vowels + high_vowels + [rl] + consonants|#):
      hyphen_pos.append(i)
      continue
    
    # 10th Case: vowel + high vowel + [mn]$
    # amendoim -> amendo{i}m ->  amend[oim$] ->  amend[oim$]    
    if window3 has (all_vowels + high_vowels + [mn]$):
      hyphen_pos.append(i)
      continue
    # abstraindo -> abstra{i}ndo -> abstr[aind]o -> abstr[a-ind]o 
    if window3 has (all_vowels + high_vowels + [mn] + consonants):
      hyphen_pos.append(i)
      continue
    
  return hyphen_pos
\end{verbatim}
