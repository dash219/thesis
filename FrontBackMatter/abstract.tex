% ************************** Thesis Abstract *****************************
% Use `abstract' as an option in the document class to print only the titlepage and the abstract.
\begin{abstract}
Esta disserta\c{c}\~ao apresenta recursos e ferramentas voltadas para o desenvolvimento de aplica\c{c}\~oes para reconhecimento de fala, tanto de nativos de n\~ao-nativos. S\~ao quatro as contribuições aqui discutidas. Primeiro, um conversor grafema-fonema h\'ibrido para o Portugu\^es Brasileiro, chamado \emph{Aeiouad\^o}, o qual utiliza regras de transcri\c{c}\~ao fon\'etica e Classification and Regression Trees (CART) para inferir os fones da fala. Segundo, uma ferramenta de corre\c{c}\~ao autom\'atica baseada em aprendizado de m\'aquina, que leva em conta erros de digita\c{c}\~ao de origem fon\'etica, que \'e capaz de lidar com erros contextuais e emprega as transcri\c{c}\~oes geradas pelo \emph{Aeiouad\^o}. Terceiro, um m\'etodo para a extra\c{c}\~ao de senten\c{c}as foneticamente-ricas, tendo em vista a cria\c{c}\~ao de corpora de fala, baseado em algoritmos gulosos. Quarto, um prot\'otipo de um sistema de reconhecimento e corre\c{c}\~ao de fala n\~ao-nativa, voltado para o Ingl\^es falado por aprendizes brasileiros.
\end{abstract}