% ************************** Thesis Abstract *****************************
% Use `abstract' as an option in the document class to print only the titlepage and the abstract.
\begin{abstract}
Esta dissertação apresenta recursos voltados para o desenvolvimento de aplicações de reconhecimento de fala e avaliação de pronúncia. São quatro as contribuições aqui discutidas. Primeiro, um conversor grafema-fonema híbrido para o Português Brasileiro, chamado Aeiouadô, o qual utiliza regras de transcrição fonética e Classification and Regression Trees (CART) para inferir os fones da fala. Segundo, uma ferramenta de correção automática baseada em aprendizado de máquina, que leva em conta erros de digitação de origem fonética, que é capaz de lidar com erros contextuais e emprega as transcrições geradas pelo Aeiouadô. Terceiro, um método para a extração de sentenças foneticamente-ricas, tendo em vista a criação de corpora de fala, baseado em algoritmos gulosos. Quarto, um protótipo de um sistema de reconhecimento e correção de fala não-nativa, voltado para o Inglês falado por aprendizes brasileiros.

\section*{\centering Resumo}
\medskip \medskip
Esta dissertação apresenta recursos voltados para o desenvolvimento de aplicações de reconhecimento de fala e avaliação de pronúncia. São quatro as contribuições aqui discutidas. Primeiro, um conversor grafema-fonema híbrido para o Português Brasileiro, chamado Aeiouadô, o qual utiliza regras de transcrição fonética e Classification and Regression Trees (CART) para inferir os fones da fala. Segundo, uma ferramenta de correção automática baseada em aprendizado de máquina, que leva em conta erros de digitação de origem fonética, que é capaz de lidar com erros contextuais e emprega as transcrições geradas pelo Aeiouadô. Terceiro, um método para a extração de sentenças foneticamente-ricas, tendo em vista a criação de corpora de fala, baseado em algoritmos gulosos. Quarto, um protótipo de um sistema de reconhecimento e correção de fala não-nativa, voltado para o Inglês falado por aprendizes brasileiros.
\end{abstract}