% ************************** Thesis Abstract *****************************
% Use `abstract' as an option in the document class to print only the titlepage and the abstract.
\begin{abstract}
This thesis presents tools and resources for the development of applications in Natural Language Processing and Pronunciation Training. There are four main contributions. First, a hybrid grapheme-to-phoneme converter for Brazilian Portuguese, named Aeiouad\^o, which makes use of both manual transcription rules and Classification and Regression Trees (CART) to infer the phone transcription. Second, a spelling correction system based on machine learning, which uses the trascriptions produced by Aeiouad\^o and is capable of handling phonologically-motivated errors, as well as contextual errors. Third, a method for the extraction of phonetically-rich sentences, which is based on greedy algorithms.  Fourth, a prototype system for automatic pronunciation assessment, especially designed for Brazilian-accented English.
\small
\paragraph{Palavras-chaves} natural language processing; pronunciation training; text-to-speech; spelling correction; corpus balancing; automatic pronunciation assessment.
\normalsize
\section*{\centering Resumo}
\smallskip
Esta disserta\c{c}\~ao apresenta recursos voltados para o desenvolvimento de aplicações de reconhecimento de fala e avaliação de pron\'uncia. S\~ao quatro as contribui\c{c}\~oes aqui discutidas. Primeiro, um conversor grafema-fonema h\'ibrido para o Portugu\^es Brasileiro, chamado Aeiouad\^o, o qual utiliza regras de transcri\c{c}\~ao fon\'etica e Classification and Regression Trees (CART) para inferir os fones da fala. Segundo, uma ferramenta de corre\c{c}\~ao autom\'atica baseada em aprendizado de m\'aquina, que leva em conta erros de digita\c{c}\~ao de origem fon\'etica, que \'e capaz de lidar com erros contextuais e emprega as transcri\c{c}\~oes geradas pelo Aeiouad\^o. Terceiro, um m\'etodo para a extração de sentenças foneticamente-ricas, tendo em vista a criação de corpora de fala, baseado em algoritmos gulosos. Quarto, um protótipo de um sistema de reconhecimento e correção de fala não-nativa, voltado para o Inglês falado por aprendizes brasileiros.
\small
\paragraph{Palavras-chaves} processamento de l\'ingua natural; treino de pron\'uncia; convers\~ao grafema-fonema; corretor ortogr\'afico; balanceamento de corpus; avalia\c{c}\~ao de pron\'uncia autom\'atica.

\end{abstract}