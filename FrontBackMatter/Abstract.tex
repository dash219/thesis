%*******************************************************
% Abstract
%*******************************************************
%\renewcommand{\abstractname}{Abstract}
\pdfbookmark[1]{Abstract}{Abstract}
\begingroup
\let\clearpage\relax
\let\cleardoublepage\relax
\let\cleardoublepage\relax

\chapter*{Abstract}

Pesquisas recentes t\^em avaliado o Brasil entre os pa\'ises com menor n\'ivel
de profici\^encia em l\'ingua inglesa. Este projeto busca criar um recurso
que possa contribuir para a melhoria desse cen\'ario. O objetivo \'e
desenvolver um reconhecedor de pron\'uncia para falantes do portugu\^es
brasileiro (PB) aprendizes de ingl\^es, chamado Listener, que seja capaz
de fornecer ao usu\'ario feedback sobre sua pron\'uncia. Recursos
semelhantes j\'a foram desenvolvidos para outras l\'inguas, no entanto, para
o PB, h\'a ainda uma lacuna a ser explorada. A hip\'otese de pesquisa \'e que
\'e poss\'ivel construir tal reconhecedor de pron\'uncia atrav\'es de: (i) uma
classifica\c{c}\~ao de erros de pron\'uncia que leve em conta a transfer\^encia de
padr\~oes de L1 para L2; (ii) um modelo ac\'ustico que agregue dados de fala
do ingl\^es tanto de nativos, quanto de aprendizes; (iii) um dicion\'ario de
pron\'uncia que contenha a transcri\c{c}\~ao das pron\'uncias desviantes do
aprendiz; e (iv) um modelo de l\'ingua que condiga com a sintaxe do
aprendiz. Nove erros de pron\'uncia foram selecionados para serem tratados
pelo Listener, assumindo-se, como pron\'uncia padr\~ao, o General American
(GA). A engine Julius ser\'a empregada como base do reconhecedor. O modelo
ac\'ustico ser\'a compilado a partir de um corpus de fala de nativos de
ingl\^es: TIMIT Acoustic-Phonetic Continuous Speech Corpus{[}1{]}; e outro
de aprendizes: COBAI - Corpus Oral Brasileiro de Aprendizes de
Ingl\^es{[}2{]}. O dicion\'ario a ser empregado \'e o CMU Pronouncing
Dictionary, ao qual ser\~ao acrescentaremos as hip\'oteses de pron\'uncia dos
aprendizes, por meio de regras. O modelo de l\'ingua ser\'a gerado a partir
da Simple English Wikipedia em conjunto com um corpus de textos escritos
por aprendizes de ingl\^es, o COMAprend{[}3{]}, um dos tr\^es corpus do
projeto COMET da Faculdade de Filosofia, Letras e Ci\^encias Humanas da
Universidade de S\~ao Paulo. A efici\^encia do reconhecedor ser\'a avaliada
por meio de medidas de Word Error Rate (WER), Character Error Rate (CER)
e matrizes de confus\~ao. O reconhecedor proposto visa a propiciar a
cria\c{c}\~ao de sistemas de treino de pron\'uncia mediado por computador. De
modo a verificar a viabilidade do m\'etodo ora proposto, um prot\'otipo do
reconhecedor foi elaborado e avaliado intrinsecamente. Um excerto do
COBAI e de um corpus de erros induzidos especialmente coletado para este
prot\'otipo foram utilizados para alimentar o modelo ac\'ustico
(\textasciitilde{}3h50min de fala). O prot\'otipo foi desenvolvido para
reconhecer erros relacionados à simplifica\c{c}\~ao sil\'abica. As variantes de
pron\'uncia foram adicionadas no dicion\'ario atrav\'es de um conjunto de 20
regras. As taxas de WER obtidas foram de 61\% para um l\'exico de 5.768
entradas, e de 78\% para alinhamento for\c{c}ado, indicando, portanto, que o
m\'etodo \'e promissor.\dots


\vfill

\pdfbookmark[1]{Zusammenfassung}{Zusammenfassung}
\chapter*{Zusammenfassung}
Kurze Zusammenfassung des Inhaltes in deutscher Sprache\dots


\endgroup			

\vfill